% --- Contenuto LaTeX autogenerato da conclusione.md (sezione 9) ---

\section{Conclusione: Un Ecosistema per la Composizione}
L'analisi ha dimostrato come Gamma realizzi con successo il suo paradigma centrale: quello delle maschere di tendenza. Il compositore definisce i confini, le probabilità, le traiettorie, e il sistema esplora lo spazio creativo così delineato.

Tuttavia l'analisi ha evidenziato anche le sfide intrinseche di questo approccio. La sintassi YAML, pur essendo leggibile, può diventare complessa e verbosa quando si definiscono evoluzioni parametriche sofisticate. Inoltre, il tempo di rendering di Csound rimane il principale collo di bottiglia, suggerendo che per un lavoro più agile potrebbero essere esplorate soluzioni di caching più avanzate o motori di sintesi alternativi per le anteprime.

Queste considerazioni non sminuiscono i risultati, ma anzi tracciano una rotta per il futuro. Il percorso da Gamma a un ipotetico sistema \textit{Delta} emerge quasi naturalmente dall'architettura esistente. Se Gamma è un sistema dove i layer operano in un elegante isolamento, Delta potrebbe esplorare l'interazione tra gli elementi all'interno dei layer, trasformando la composizione da un insieme di processi paralleli a un vero e proprio sistema di agenti sonori che si ascoltano e si influenzano a vicenda. L'introduzione di una memoria a lungo termine e di metriche di valutazione del materiale generato potrebbe trasformare il sistema da un esecutore di istruzioni a un agente capace di auto-organizzazione.