% --- Contenuto LaTeX autogenerato da capitolo2.md (sezione 3) ---

\section{CONFIGURAZIONE E PARAMETRIZZAZIONE}
Il sistema Gamma implementa una separazione netta tra logica di sintesi e configurazione dei parametri, permettendo al compositore di modificare profondamente il comportamento del sistema senza toccare il codice Csound. Questo approccio modulare facilita la sperimentazione e l'estensione del sistema.
\subsection{Il File tables.yaml}
Il file \texttt{tables.yaml} rappresenta il cuore configurabile del sistema, definendo tutte le tabelle di forma d'onda e inviluppo utilizzate nella sintesi. La sua struttura gerarchica separa chiaramente gli inviluppi per eventi singoli da quelli per sezioni intere.
\subsubsection{Struttura delle Definizioni di Tabella}
Ogni tabella è definita attraverso quattro parametri fondamentali:

\begin{lstlisting}[language=Python]
nome_simbolico:
  number: [numero della f-table in Csound]
  size: [dimensione in campioni]
  gen_routine: [numero della GEN routine]
  parameters: [lista dei parametri per la GEN]
\end{lstlisting}

Vediamo un esempio concreto:

\begin{lstlisting}[language=Python]
event_envelopes:
  lineare:
    number: 2
    size: 4096
    gen_routine: 6
    parameters: [0.001, 2048, 0.5, 2048, 1]  # Linea retta da 0 a 1
\end{lstlisting}

Questa definizione genera in Csound:
\begin{lstlisting}[language=C]
f 2 0 4096 6 0.001 2048 0.5 2048 1
\end{lstlisting}

La scelta della GEN routine 6 (segmenti cubici) invece della più comune GEN 7 (segmenti lineari) permette transizioni più morbide tra i punti di controllo, essenziale per inviluppi naturali.
\subsubsection{Inviluppi Evento vs Inviluppi Sezione}
Il sistema distingue due categorie di inviluppi con funzioni distinte:

\texttt{Event Envelopes} Applicati ai singoli eventi sonori:

\begin{lstlisting}[language=Python]
event_envelopes:
  impulsivo:
    number: 5
    size: 4096
    gen_routine: 5
    parameters: [0.001, 512, 1, 3584, 0.0001]

lento:
    number: 6
    size: 4096
    gen_routine: 7
    parameters: [0, 3072, 1, 1024, 0]

sostenuto:
    number: 7
    size: 4096
    gen_routine: 7
    parameters: [0, 512, 1, 3072, 1, 512, 0]
\end{lstlisting}

L'inviluppo \texttt{impulsivo} utilizza GEN 5 (segmenti esponenziali) per creare un attacco rapidissimo (512 campioni su 4096, circa 1/8 della durata) seguito da un decadimento esponenziale. Il valore finale di 0.0001 invece di 0 evita discontinuità nell'interpolazione esponenziale.

L'inviluppo \texttt{lento} con GEN 7 crea un attacco graduale per 3/4 della durata, ideale per tessiture ambient o crescendi graduali.

\texttt{Section Envelopes} Modulano interi gruppi di eventi:

\begin{lstlisting}[language=Python]
section_envelopes:
  crescendo_diminuendo:
    number: 24
    size: 4096
    gen_routine: 7
    parameters: [0, 2048, 1, 2048, 0]

impulso:
    number: 25
    size: 4096
    gen_routine: 6
    parameters: [1, 4096, 0.001]
\end{lstlisting}

Gli inviluppi di sezione operano su una scala temporale maggiore. Il numero di tabella parte da 20 per convenzione, distinguendoli chiaramente dagli inviluppi evento nel codice Csound:

\begin{lstlisting}[language=C]
if i_ifn_section_env > 20 && i_section_duration > 0 then
    ; Applica inviluppo di sezione
endif
\end{lstlisting}
\subsubsection{GEN Routines e Loro Parametri}
Le GEN routines utilizzate nel sistema sono scelte per le loro caratteristiche specifiche:

\texttt{GEN 5} Segmenti esponenziali:
\begin{itemize}
    \item Ideale per decadimenti naturali e attacchi percussivi
    \item Richiede valori non-zero per evitare singolarità matematiche
    \item Parametri: [valore, durata, valore, durata, \ldots]
\end{itemize}

\texttt{GEN 6} Segmenti cubici:
\begin{itemize}
    \item Transizioni morbide senza discontinuità nelle derivate
    \item Perfetta per inviluppi che devono suonare \textit{organici}
    \item Stessa sintassi di GEN 5 ma interpolazione diversa
\end{itemize}

\texttt{GEN 7} Segmenti lineari:
\begin{itemize}
    \item Controllo preciso e prevedibile
    \item Efficiente computazionalmente
    \item Ideale per forme geometriche precise
\end{itemize}

\texttt{GEN 10} Sintesi additiva:
\begin{lstlisting}[language=Python]
plateau_forte:
  number: 23
  size: 4096
  gen_routine: 10
  parameters: [1]
\end{lstlisting}
Usata qui in modo non convenzionale per creare un valore costante.
\subsection{Sistema di Macro e Costanti Globali}
Il template CSD di Gamma definisce un sistema di macro che parametrizza l'intero spazio sonoro:

\begin{lstlisting}[language=C]
#define SQRT2 #1.4142135623730951#
#define MAX_AMP #0.999#
#define FONDAMENTALE #32#
#define OTTAVE #10#
#define INTERVALLI #200#
#define REGISTRI #50#
#define M_PI #3.141592653589793#
\end{lstlisting}
\subsubsection{Parametri dello Spazio Frequenziale}
Le macro \texttt{OTTAVE}, \texttt{INTERVALLI} e \texttt{REGISTRI} definiscono la risoluzione del sistema di intonazione:

\begin{itemize}
    \item \texttt{OTTAVE} (10): Copre l'intero range udibile da 32 Hz a \textasciitilde{}32 kHz
    \item \texttt{INTERVALLI} (200): Numero di divisioni per ottava nel sistema pitagorico
    \item \texttt{REGISTRI} (50): Suddivisioni fini all'interno di ogni ottava
\end{itemize}

La relazione tra questi parametri determina la granularità frequenziale:
\begin{lstlisting}
Totale frequenze = OTTAVE * INTERVALLI = 2000
Risoluzione per registro = INTERVALLI / REGISTRI = 4 intervalli
\end{lstlisting}
\subsubsection{Tabelle Globali e Allocazione Dinamica}
Il sistema utilizza due tabelle globali principali:

\begin{lstlisting}[language=C]
gi_Index init 1
gi_eve_attacco ftgen 0, 0, 2^20, -2, 0
gi_Intonazione ftgen 0, 0, $OTTAVE*$INTERVALLI+1, -2, 0
\end{lstlisting}

\texttt{gi\_eve\_attacco}: Una tabella enorme ($2^{20}$ = 1.048.576 elementi) che memorizza i tempi di attacco di tutti gli eventi generati. La dimensione generosa permette composizioni estremamente lunghe senza rischio di overflow.

\texttt{gi\_Intonazione}: Contiene tutte le frequenze del sistema pitagorico. La dimensione è calcolata dinamicamente come \texttt{OTTAVE * INTERVALLI + 1}, dove il +1 gestisce casi limite di indicizzazione.

L'allocazione avviene con \texttt{ftgen}:
\begin{itemize}
    \item Primo parametro (0): Numero di tabella assegnato automaticamente
    \item Secondo parametro (0): Creazione a init-time
    \item Terzo parametro: Dimensione
    \item Quarto parametro (-2): GEN routine per dati arbitrari
\end{itemize}
\subsubsection{Gestione della Memoria}
Il sistema implementa strategie per ottimizzare l'uso della memoria:

\begin{enumerate}
    \item \textbf{Tabelle Temporanee}: Lo strumento Voce crea tabelle estese che vengono deallocate alla fine:
\end{enumerate}
\begin{lstlisting}[language=C]
i_TempRitmiTab ftgen 0, 0, i_LenRitmiTab + 10000, -2, 0
; ... uso della tabella ...
ftfree i_TempRitmiTab, 0
\end{lstlisting}

\begin{enumerate}
    \item \textbf{Condivisione delle Tabelle Ritmiche}: Pattern identici condividono le stesse tabelle attraverso il sistema di mappatura in Python.
    \item \textbf{Inizializzazione Lazy}: Le tabelle vengono popolate solo quando necessario, come \texttt{gi\_Intonazione} che viene riempita dallo strumento Init.
\end{enumerate}
\subsubsection{Integrazione con Python}
Il file \texttt{tables.yaml} viene letto dal generatore Python che:
\begin{enumerate}
    \item Carica le configurazioni all'inizializzazione
    \item Genera automaticamente gli f-statement nel CSD
    \item Mantiene mappe nome→numero per riferimenti simbolici
\end{enumerate}
Questo permette di riferirsi agli inviluppi per nome nel YAML compositivo:
\begin{lstlisting}[language=Python]
inviluppo_attacco: { value: 'impulsivo' }
\end{lstlisting}

Invece di numeri magici:
\begin{lstlisting}[language=Python]
inviluppo_attacco: { value: 5 }  # Meno leggibile e manutenibile
\end{lstlisting}

Il sistema di configurazione di Gamma dimostra come una buona architettura software possa rendere un sistema complesso sia potente che accessibile, permettendo estensioni e modifiche senza richiedere modifiche al core del sistema.