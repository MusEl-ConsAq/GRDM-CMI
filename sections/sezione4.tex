% --- Contenuto LaTeX autogenerato da capitolo4.md (sezione 5) ---

\section{DALL'INTENZIONE ALLA NOTA: MASCHERE DI TENDENZA E GENERAZIONE PARAMETRICA}
La generazione parametrica è il motore alchemico di Gamma, il processo centrale attraverso cui le intenzioni del compositore, espresse come  maschere di tendenza  nel file YAML, vengono trasformate in valori numerici concreti per ogni singolo evento sonoro. Analizzeremo come il sistema traduce l'astrazione in suono, con particolare attenzione alle tecniche di generazione, interpolazione e gestione della coerenza strutturale.
\subsection{Il Motore Generativo: \texttt{\_generate\_params\_from\_mask()}}
Il metodo \texttt{\_generate\_params\_from\_mask()} è il punto di convergenza tra l'astrazione compositiva e la concretezza numerica. Implementa la logica che trasforma una singola maschera di tendenza nei parametri specifici per un evento sonoro, gestendo una varietà di strategie generative per rispondere a diverse necessità espressive.
\subsubsection{Architettura e Gestione Specializzata}
Il processo non è monolitico. Il metodo prima isola un insieme di chiavi che richiedono una gestione specializzata, poiché non rappresentano parametri diretti o necessitano di logiche di trasformazione complesse.

\begin{lstlisting}[language=Python]
SKIPPED_KEYS = {
    'choices', 'weights', 'distribution',  # Metadati per la generazione
    'dynamic_index', 'dinamica',           # Logica di dinamica complessa
    'nonlinear_mode', 'senso_movimento',   # Parametri di controllo per Csound
    'inviluppo_attacco', 'tipo_ritmi',      # Richiedono traduzione o generazione complessa
    'densita_cluster'                      
}
\end{lstlisting}
Questa separazione permette a un loop generico di gestire i parametri puramente numerici, mentre logiche dedicate si occupano di tradurre concetti come \texttt{'dinamica': 'f'} nell'indice numerico richiesto da Csound o di generare intere sequenze ritmiche da una categoria come \texttt{'medi'}.
\subsubsection{Le Modalità di Generazione}
Il cuore del metodo itera sui parametri, applicando la modalità di generazione più appropriata in base alla struttura della maschera.

\begin{lstlisting}[language=Python]
for key, p_mask in mask.items():
    if key in SKIPPED_KEYS: continue
\section{1. Distribuzione Normale (Gaussiana)}
    if 'mean' in p_mask and 'std' in p_mask:
        val = np.random.normal(loc=p_mask['mean'], scale=p_mask['std'])
\section{2. Distribuzione Uniforme (Range)}
    elif 'range' in p_mask:
        min_val, max_val = p_mask['range']
        val = random.randint(min_val, max_val) if isinstance(min_val, int) else random.uniform(min_val, max_val)
\section{3. Scelta Discreta Pesata}
    elif 'choices' in p_mask:
        val = random.choices(p_mask['choices'], weights=p_mask.get('weights'), k=1)[0]
\section{4. Valore Fisso}
    elif 'value' in p_mask:
        val = p_mask['value']
\end{lstlisting}

Questo toolkit di quattro modalità offre al compositore un controllo granulare sul grado di determinismo e casualità:

\begin{enumerate}
    \item Distribuzione Normale : Ideale per creare una \textit{massa} sonora attorno a un centro tonale o timbrico. Un'ottava definita come \texttt{\{mean: 5, std: 0.5\}} tenderà a rimanere nell'ottava 5, ma con occasionali e naturali \textit{fughe} verso le ottave vicine. La deviazione standard diventa un parametro espressivo che controlla la \textit{disciplina} del materiale.
    \item Range Uniforme \texttt{range: [1, 10]}: Utile quando tutti i valori in un intervallo sono ugualmente possibili.
    \item Scelta Pesata : Permette di definire il \textit{colore} statistico di una sezione. Una dinamica specificata come \texttt{\{choices: ['p', 'mf', 'f'], weights: [0.6, 0.3, 0.1]\}} assicura una predominanza di eventi piano, pur mantenendo la varietà.
    \item Valore Fisso : Garantisce il determinismo assoluto, essenziale per parametri strutturali come il senso di movimento spaziale.
\end{enumerate}
\subsubsection{La Logica Gerarchica dei Ritmi}
La generazione dei ritmi è un esempio emblematico di come il sistema supporti molteplici livelli di astrazione, dal controllo totale alla delega generativa.

\begin{lstlisting}[language=Python]
if 'explicit_values' in rhythm_mask:
\section{Modalità 1: Controllo totale con una lista esplicita}
    params['ritmi'] = rhythm_mask['explicit_values']
elif 'choices' in rhythm_mask:
    choice = random.choices(rhythm_mask['choices'], ...)[0]
    if isinstance(choice, list):
\section{Modalità 2: Scelta tra pattern pre-composti}
        params['ritmi'] = choice
    else:
\section{Modalità 3: Astrazione massima tramite categorie ('piccoli', 'medi'...)}
        params['ritmi'] = self._generate_rhythm_pattern(choice)
\end{lstlisting}
Questa architettura permette al compositore di scegliere il livello di dettaglio più consono: specificare un pattern esatto, scegliere da una libreria di pattern, o semplicemente indicare una \textit{qualità} ritmica desiderata.
\subsection{L'Evoluzione nel Tempo: Interpolazione delle Maschere}
Se la generazione da una singola maschera crea eventi statici, l'interpolazione tra due maschere (\texttt{stato\_iniziale} e \texttt{stato\_finale}) dà vita a processi dinamici e trasformativi. Il metodo \texttt{\_interpolate\_mask()} implementa questa logica.
\subsubsection{Gestione delle Transizioni Asimmetriche}
Un problema chiave nell'interpolazione è come gestire parametri che compaiono solo nello stato finale. Gamma adotta una strategia di \textit{riempimento} che ne aumenta la flessibilità.

\begin{lstlisting}[language=Python]
all_keys = set(start_mask.keys()) | set(end_mask.keys())
for key in all_keys:
    s_mask = start_mask.get(key)
    e_mask = end_mask.get(key)

if s_mask is None: s_mask = e_mask
    if e_mask is None: e_mask = s_mask
\end{lstlisting}
Se un parametro è definito solo alla fine, il sistema assume che fosse presente fin dall'inizio con lo stesso valore finale. Questo permette di \textit{introdurre} un nuovo processo (es. un glissando) senza doverne specificare un valore nullo all'inizio, semplificando la scrittura delle partiture YAML.
\subsubsection{Strategie di Interpolazione Differenziate}
L'interpolazione si adatta al tipo di parametro, producendo transizioni musicalmente significative:

\begin{itemize}
 \item Parametri Numerici (Range, Mean/Std) : I loro valori vengono interpolati linearmente. Questo permette di creare effetti come un \textit{restringimento} del campo sonoro (interpolando verso un range più piccolo) o una \textit{focalizzazione} (interpolando verso una deviazione standard minore).
\end{itemize}

\begin{itemize}
 \item Scelte Discrete (Choices) : Il sistema tenta un  cross-fade probabilistico . Se le scelte sono le stesse, i loro pesi (\texttt{weights}) vengono interpolati. Questo rea una transizione graduale nella probabilità di occorrenza, ad esempio passando da una predominanza di dinamiche piano a una di forte. Se le scelte sono diverse, il sistema effettua una transizione a scalino a metà del percorso.
\end{itemize}

È importante notare che questo processo di interpolazione non solo guida la generazione degli eventi sonori, ma fornisce anche i dati per la visualizzazione grafica. Le \textit{maschere di tendenza} visibili nel PDF generato da \texttt{CompositionDebugger} sono la rappresentazione visiva diretta dei valori interpolati in ogni punto del tempo. Questo crea una coerenza totale tra ciò che il compositore specifica, ciò che il sistema visualizza e ciò che l'orchestra Csound suona.
\subsection{Il Sistema Gerarchico del Glissando}
Il glissando in Gamma non è una semplice transizione di frequenza, ma un sistema gerarchico che illustra l'approccio progettuale del compositore: fornire opzioni potenti con priorità chiare.

\begin{enumerate}
    \item Modalità Offset (Priorità Massima): Specifica un intervallo di glissando relativo alla nota di partenza (es. \texttt{offset\_ottava: 2}). È ideale per creare pattern di movimento che mantengono la loro coerenza intervallare a diverse altezze.
    \item Modalità Assoluta (Priorità Media) : Specifica una destinazione fissa (es. \texttt{ottava\_arrivo: 8}). È utile per creare convergenze armoniche, dove più voci, partendo da punti diversi, si dirigono verso un'unica regione tonale.
    \item Default (Nessun Glissando) : In assenza di specifiche, la frequenza rimane statica.
\end{enumerate}
Questa gerarchia viene risolta in \texttt{\_generate\_params\_from\_mask()}, che calcola i parametri \texttt{ottava\_arrivo} e \texttt{registro\_arrivo} finali. Questi vengono poi passati a Csound, dove la funzione \texttt{calcFrequenza} è chiamata due volte, una per la frequenza di partenza e una per quella di arrivo, garantendo che entrambe rispettino la logica dell'intonazione pitagorica del sistema.