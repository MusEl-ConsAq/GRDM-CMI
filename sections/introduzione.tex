% --- Contenuto LaTeX autogenerato da introduzione.md (sezione 1) ---

\section{INTRODUZIONE}
Il presente lavoro documenta lo sviluppo di ''Gamma'', un sistema compositivo algoritmico che rappresenta una tappa fondamentale nel più ampio progetto del ciclo ''Delta''. Questa tesina nasce dall'esigenza di formalizzare e analizzare un percorso compositivo che, partito con ambizioni di complessità adattiva, ha rivelato la necessità di un passaggio intermedio attraverso un sistema deterministico controllato.
\subsection{Il Ciclo Delta e la Genesi di Gamma}
Il ciclo compositivo ''Delta'' nasce dalla volontà di studiare come modellare un sistema musicale complesso che commistioni tratti caotici e adattivi. Concepito come sistema chiuso e acusmatico, Delta rappresenta un passo preliminare e necessario prima di approcciare lo studio e la realizzazione di ecosistemi performativi aperti, come quelli esplorati da compositori come Agostino Di Scipio. La scelta di lavorare inizialmente con un sistema chiuso non è limitativa, ma strategica: permette di concentrarsi sulla comprensione e modellazione delle dinamiche interne senza le variabili aggiuntive dell'interazione in tempo reale con l'ambiente o con i performer.

La sfida principale che ha portato alla nascita di Gamma non risiedeva nel mantenere un controllo compositivo - questione che in un sistema chiuso è per definizione gestibile - ma nella complessità intrinseca di progettare e gestire le relazioni interne di uno strumento compositivo senza avere l'esperienza diretta di ''suonarlo'' e di comporci. È come tentare di costruire uno strumento musicale complesso senza poterlo testare durante la costruzione: la mancanza di un feedback esperienziale rende difficile calibrare le relazioni tra i parametri, prevedere i comportamenti emergenti, e soprattutto sviluppare un'intuizione compositiva per il sistema.

È in questo contesto che nasce ''Gamma'', non come ripiego o semplificazione, ma come banco di prova necessario. Gamma permette di esplorare in modo più deterministico e controllato le stesse tecniche e strutture che poi verranno integrate nel sistema più complesso di Delta. È un laboratorio dove sperimentare, comprendere e affinare gli strumenti compositivi prima di lanciarsi nella complessità delle dinamiche caotiche e adattive.

La scelta del nome ''Gamma'' riflette precisamente questa funzione: rappresenta la gamma di possibilità esperibili dello strumento Delta. Se Delta è la foce dove tutti i flussi compositivi convergono in complessità caotica e adattiva, Gamma è la sorgente - il luogo dove questi flussi nascono chiari e distinguibili, dove è possibile osservare e comprendere ogni singolo rivolo prima che si mescoli con gli altri. In fisica, i raggi gamma rappresentano una forma di radiazione elettromagnetica ad alta energia e frequenza, caratterizzata da comportamenti sia ondulatori che corpuscolari. Questa dualità rispecchia perfettamente la natura del sistema compositivo sviluppato, che oscilla costantemente tra determinismo e stocasticità, tra controllo e casualità.
\subsection{Dalle Note alle Nuvole: Un'Eredità Compositiva}
La tradizione della composizione musicale occidentale si è basata per secoli sul controllo preciso di parametri discreti: altezze definite, durate quantizzate, dinamiche categorizzate. Il compositore operava come un architetto che posiziona mattoni sonori in posizioni predeterminate. L'avvento della musica elettronica e delle tecniche compositive algoritmiche ha inizialmente replicato questo paradigma in ambito digitale, sostituendo la notazione tradizionale con liste di parametri numerici, ma mantenendo la stessa filosofia di controllo deterministico.

Gamma si inserisce consapevolmente nella tradizione della composizione stocastica e generativa, ispirandosi liberamente ai metodi di lavoro sviluppati da pionieri come Iannis Xenakis e Barry Truax. L'approccio delle ''maschere di tendenza'' che caratterizza il sistema non è un'invenzione ex novo, ma piuttosto un'interpretazione personale e un'implementazione specifica di tecniche già consolidate nella letteratura della computer music. Xenakis, con la sua musica stocastica, aveva già negli anni '60 esplorato l'uso di distribuzioni probabilistiche per la generazione di masse sonore. Truax sviluppò l'approccio delle maschere di tendenza per necessità pratiche legate alla sintesi granulare: quando si lavora con tecniche che richiedono la generazione di milioni di parametri per controllare nuvole di grani sonori, diventa impossibile specificare ogni singolo valore. Le maschere di tendenza emergono quindi come soluzione naturale per gestire questa complessità, permettendo di definire comportamenti statistici globali piuttosto che valori individuali.

Ciò che Gamma apporta a questa tradizione è una sistematizzazione particolare di questi concetti, adattandoli alle esigenze specifiche del progetto Delta. Il sistema implementa quattro modalità distinte di generazione parametrica (range, choices, distribuzione normale, valore fisso), organizzate in una gerarchia compositiva chiara (composizione → sezioni → layer → eventi). Questa strutturazione permette di gestire la complessità mantenendo un controllo compositivo significativo, preparando il terreno per l'evoluzione verso il sistema adattivo previsto.

L'uso delle maschere di tendenza in Gamma permette al compositore di lavorare su diversi livelli di astrazione simultaneamente. A livello micro, si possono definire distribuzioni precise per singoli parametri; a livello macro, si possono creare evoluzioni graduali attraverso l'interpolazione tra stati. Questa flessibilità multi-scala è essenziale per gestire la complessità formale richiesta da composizioni di ampio respiro.
\subsection{Struttura e Obiettivi della Tesina}
Il presente lavoro si propone di documentare e analizzare il sistema Gamma sotto molteplici prospettive, fornendo sia una comprensione teorica dei principi sottostanti sia una guida pratica all'implementazione e all'uso del sistema.

Gli obiettivi principali sono:

\begin{enumerate}
    \item \textbf{Documentare l'architettura del sistema}: Fornire una descrizione dettagliata e sistematica di tutti i componenti software che costituiscono Gamma, dalle strutture dati Python agli strumenti Csound, dalla sintassi YAML al sistema di visualizzazione.
    \item \textbf{Contestualizzare il lavoro nella tradizione elettroacustica}: Evidenziare come Gamma si inserisca nel continuum della composizione algoritmica, riconoscendo i debiti verso i predecessori e identificando gli elementi di originalità nell'implementazione.
    \item \textbf{Analizzare le tecniche compositive}: Attraverso esempi concreti tratti dal repertorio creato con Gamma, mostrare come i principi teorici si traducano in pratica compositiva e quali possibilità espressive il sistema offra.
    \item \textbf{Valutare criticamente il sistema}: Identificare punti di forza e limitazioni di Gamma, sia dal punto di vista tecnico che estetico, fornendo spunti per sviluppi futuri.
    \item \textbf{Preparare il terreno per Delta}: Comprendere come l'esperienza di Gamma informi e prepari lo sviluppo del sistema adattivo completo previsto per Delta.
\end{enumerate}

La tesina è strutturata in modo da guidare il lettore attraverso un percorso che parte dai fondamenti teorici, passa attraverso i dettagli implementativi, e culmina nell'analisi di opere concrete. Questo approccio permette sia ai lettori interessati agli aspetti concettuali sia a quelli più orientati alla pratica di trovare contenuti rilevanti e accessibili.