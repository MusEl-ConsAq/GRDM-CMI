\documentclass[a4paper,12pt]{article}
\usepackage{ME_AQ_temp}
\usepackage{tabularx}
\usepackage{listings}
\usepackage{xcolor}

\setmonofont{Fira Code}
% \setmonofont{Source Code Pro}
% \setmonofont{DejaVu Sans Mono}
% \setmonofont{JuliaMono} % Ottimo per caratteri scientifici e matematici

\usepackage{listings}
\usepackage{microtype}
% \sloppy % Prova a rimuoverlo o commentarlo. Usa emergencystretch che è meno aggressivo.
\emergencystretch=2em % Ho aumentato leggermente il valore, puoi regolarlo

\lstset{
    % Il font di base ora sarà quello impostato da \setmonofont
    basicstyle=\ttfamily\footnotesize, 
    breaklines=true,
    frame=single,
    numbers=left,
    numbersep=5pt,
    postbreak=\mbox{\textcolor{red}{$\hookrightarrow$}\space},
    breakindent=1.5em,
    breakautoindent=true,
    % La riga 'literate' non è più necessaria! Il nuovo font ha già i simboli ≈, ✓, ✗, ecc.
    % literate={≈}{{$\approx$}}1 
}
% Applica questo stile a tutti gli ambienti lstlisting del documento
%% -------------------------------------- %%
%  - impostare il titolo della tesina in entrambe le righe 
% 
%% -------------------------------------- %%

\newcommand{\mycustomtitle}{Gamma}
% Definisci un titolo personalizzato
\newcommand{\setmytitle}[1]{\renewcommand{\mycustomtitle}{#1}}

% Definizione del titolo e dell'autore
\title{Corsi Accademici di Musica Elettronica DCPL34 Conservatorio A. Casella, L'Aquila \\ \fontsize{14}{17}\bfseries\uppercase{Gamma}}
\author{Giulio Romano De Mattia \\ esame di \bfseries{Composizione Musicale Informatica} }
\date{27/06/2025}

% Sovrascrivi le impostazioni di hyperref per l'indice
\hypersetup{
    linkcolor=black, % Imposta il colore dei link dell'indice a nero
}

\begin{document}

% Pagina 1: Titolo e riassunto
\maketitle
\thispagestyle{empty}

\begin{center}
    \vspace{1cm}
    \textbf{\fontsize{12}{15}\selectfont{Sommario}}
\end{center}

Il presente scritto documenta il lavoro svolto nell'arco di un anno e mezzo in merito al brano di composizione algoritmica \textit{Gamma}. Non volendomi fermare 
alla contemplazione del risultato musicale in quanto tale, in questa tesina porrò l'attenzione sugli strumenti compositivi scritti per la realizzazione 
del brano poiché reputo lo strumento stesso e l'ambiente di sviluppo digitale creato come fondamenta della composizione, se non composizione essa stessa. 
Verrà così esplorato il motore di csound, ultimo elemento della catena, per risalire poi al codice python, per arrivare fin sù la sorgente, il dizionario YAML che rappresenta la partitura della composizione: il \textit{sorgente} di Gamma.



\newpage
% Genera l'indice
\tableofcontents  

% Pagina 2: Introduzione e resto del testo
\newpage

% --- File di inclusione generato automaticamente ---
% --- Contenuto LaTeX autogenerato da introduzione.md (sezione 1) ---

\section{INTRODUZIONE}
Il presente lavoro documenta lo sviluppo di ''Gamma'', un sistema compositivo algoritmico che rappresenta una tappa fondamentale nel più ampio progetto del ciclo ''Delta''. Questa tesina nasce dall'esigenza di formalizzare e analizzare un percorso compositivo che, partito con ambizioni di complessità adattiva, ha rivelato la necessità di un passaggio intermedio attraverso un sistema deterministico controllato.
\subsection{Il Ciclo Delta e la Genesi di Gamma}
Il ciclo compositivo ''Delta'' nasce dalla volontà di studiare come modellare un sistema musicale complesso che commistioni tratti caotici e adattivi. Concepito come sistema chiuso e acusmatico, Delta rappresenta un passo preliminare e necessario prima di approcciare lo studio e la realizzazione di ecosistemi performativi aperti, come quelli esplorati da compositori come Agostino Di Scipio. La scelta di lavorare inizialmente con un sistema chiuso non è limitativa, ma strategica: permette di concentrarsi sulla comprensione e modellazione delle dinamiche interne senza le variabili aggiuntive dell'interazione in tempo reale con l'ambiente o con i performer.

La sfida principale che ha portato alla nascita di Gamma non risiedeva nel mantenere un controllo compositivo - questione che in un sistema chiuso è per definizione gestibile - ma nella complessità intrinseca di progettare e gestire le relazioni interne di uno strumento compositivo senza avere l'esperienza diretta di ''suonarlo'' e di comporci. È come tentare di costruire uno strumento musicale complesso senza poterlo testare durante la costruzione: la mancanza di un feedback esperienziale rende difficile calibrare le relazioni tra i parametri, prevedere i comportamenti emergenti, e soprattutto sviluppare un'intuizione compositiva per il sistema.

È in questo contesto che nasce ''Gamma'', non come ripiego o semplificazione, ma come banco di prova necessario. Gamma permette di esplorare in modo più deterministico e controllato le stesse tecniche e strutture che poi verranno integrate nel sistema più complesso di Delta. È un laboratorio dove sperimentare, comprendere e affinare gli strumenti compositivi prima di lanciarsi nella complessità delle dinamiche caotiche e adattive.

La scelta del nome ''Gamma'' riflette precisamente questa funzione: rappresenta la gamma di possibilità esperibili dello strumento Delta. Se Delta è la foce dove tutti i flussi compositivi convergono in complessità caotica e adattiva, Gamma è la sorgente - il luogo dove questi flussi nascono chiari e distinguibili, dove è possibile osservare e comprendere ogni singolo rivolo prima che si mescoli con gli altri. In fisica, i raggi gamma rappresentano una forma di radiazione elettromagnetica ad alta energia e frequenza, caratterizzata da comportamenti sia ondulatori che corpuscolari. Questa dualità rispecchia perfettamente la natura del sistema compositivo sviluppato, che oscilla costantemente tra determinismo e stocasticità, tra controllo e casualità.
\subsection{Dalle Note alle Nuvole: Un'Eredità Compositiva}
La tradizione della composizione musicale occidentale si è basata per secoli sul controllo preciso di parametri discreti: altezze definite, durate quantizzate, dinamiche categorizzate. Il compositore operava come un architetto che posiziona mattoni sonori in posizioni predeterminate. L'avvento della musica elettronica e delle tecniche compositive algoritmiche ha inizialmente replicato questo paradigma in ambito digitale, sostituendo la notazione tradizionale con liste di parametri numerici, ma mantenendo la stessa filosofia di controllo deterministico.

Gamma si inserisce consapevolmente nella tradizione della composizione stocastica e generativa, ispirandosi liberamente ai metodi di lavoro sviluppati da pionieri come Iannis Xenakis e Barry Truax. L'approccio delle ''maschere di tendenza'' che caratterizza il sistema non è un'invenzione ex novo, ma piuttosto un'interpretazione personale e un'implementazione specifica di tecniche già consolidate nella letteratura della computer music. Xenakis, con la sua musica stocastica, aveva già negli anni '60 esplorato l'uso di distribuzioni probabilistiche per la generazione di masse sonore. Truax sviluppò l'approccio delle maschere di tendenza per necessità pratiche legate alla sintesi granulare: quando si lavora con tecniche che richiedono la generazione di milioni di parametri per controllare nuvole di grani sonori, diventa impossibile specificare ogni singolo valore. Le maschere di tendenza emergono quindi come soluzione naturale per gestire questa complessità, permettendo di definire comportamenti statistici globali piuttosto che valori individuali.

Ciò che Gamma apporta a questa tradizione è una sistematizzazione particolare di questi concetti, adattandoli alle esigenze specifiche del progetto Delta. Il sistema implementa quattro modalità distinte di generazione parametrica (range, choices, distribuzione normale, valore fisso), organizzate in una gerarchia compositiva chiara (composizione → sezioni → layer → eventi). Questa strutturazione permette di gestire la complessità mantenendo un controllo compositivo significativo, preparando il terreno per l'evoluzione verso il sistema adattivo previsto.

L'uso delle maschere di tendenza in Gamma permette al compositore di lavorare su diversi livelli di astrazione simultaneamente. A livello micro, si possono definire distribuzioni precise per singoli parametri; a livello macro, si possono creare evoluzioni graduali attraverso l'interpolazione tra stati. Questa flessibilità multi-scala è essenziale per gestire la complessità formale richiesta da composizioni di ampio respiro.
\subsection{Struttura e Obiettivi della Tesina}
Il presente lavoro si propone di documentare e analizzare il sistema Gamma sotto molteplici prospettive, fornendo sia una comprensione teorica dei principi sottostanti sia una guida pratica all'implementazione e all'uso del sistema.

Gli obiettivi principali sono:

\begin{enumerate}
    \item \textbf{Documentare l'architettura del sistema}: Fornire una descrizione dettagliata e sistematica di tutti i componenti software che costituiscono Gamma, dalle strutture dati Python agli strumenti Csound, dalla sintassi YAML al sistema di visualizzazione.
    \item \textbf{Contestualizzare il lavoro nella tradizione elettroacustica}: Evidenziare come Gamma si inserisca nel continuum della composizione algoritmica, riconoscendo i debiti verso i predecessori e identificando gli elementi di originalità nell'implementazione.
    \item \textbf{Analizzare le tecniche compositive}: Attraverso esempi concreti tratti dal repertorio creato con Gamma, mostrare come i principi teorici si traducano in pratica compositiva e quali possibilità espressive il sistema offra.
    \item \textbf{Valutare criticamente il sistema}: Identificare punti di forza e limitazioni di Gamma, sia dal punto di vista tecnico che estetico, fornendo spunti per sviluppi futuri.
    \item \textbf{Preparare il terreno per Delta}: Comprendere come l'esperienza di Gamma informi e prepari lo sviluppo del sistema adattivo completo previsto per Delta.
\end{enumerate}

La tesina è strutturata in modo da guidare il lettore attraverso un percorso che parte dai fondamenti teorici, passa attraverso i dettagli implementativi, e culmina nell'analisi di opere concrete. Questo approccio permette sia ai lettori interessati agli aspetti concettuali sia a quelli più orientati alla pratica di trovare contenuti rilevanti e accessibili.  % Auto-generated: include introduzione.tex
% --- Contenuto LaTeX autogenerato da capitolo1.md (sezione 2) ---

\section{L'ORCHESTRA GAMMA}
L'orchestra Csound di Gamma non è un sistema autonomo, ma il motore di sintesi e di esecuzione progettato specificamente per interpretare le strutture musicali complesse generate dallo script Python \texttt{generative\_composer.py}. Ogni strumento e opcode è stato creato per tradurre in suono un parametro o un comportamento definito nel file YAML di input. Lo strumento Voce funge da \textit{ponte} principale, ricevendo un intero \textit{comportamento} (un cluster di eventi) da Python e orchestrandone la micro-temporalità e la sintesi. In questo capitolo, analizzeremo come questa traduzione avviene, partendo dal livello macroscopico (Voce) fino al dettaglio del singolo campione audio (eventoSonoro).
\subsection{Lo Strumento Voce: Generatore di Comportamenti}
Lo strumento \texttt{Voce} costituisce il livello più alto della gerarchia di sintesi in Gamma. Non genera direttamente suoni, ma orchestra la creazione di sequenze di eventi sonori secondo logiche compositive complesse. La sua definizione inizia con una ricca parametrizzazione:

\begin{lstlisting}[language=C]
instr Voce
    ; -----------------------------------------------------------------------
    ; 1. INIZIALIZZAZIONE E ACQUISIZIONE PARAMETRI
    ; -----------------------------------------------------------------------
    i_CAttacco       = p2             ; Tempo di attacco del comportamento
    i_Durata         = p3             ; Durata complessiva
    i_RitmiTab       = p4             ; Tabella dei ritmi
    i_DurataArmonica = p5             ; Durata armonica di riferimento
    i_DynamicIndex   = p6         
    i_Ottava         = p7             
    i_Registro       = p8             
    i_ottava_arrivo = p9
    i_registro_arrivo = p10
    i_PosTab         = p11             ; Tabella delle posizioni
    i_IdComp         = p12            ; ID del comportamento
    i_NonlinearMode  = (p13 == 0 ? 3 : p13)
    i_SensoMovimento = (p14 == 0 ? 1 : p14) 
    i_ifnAttacco     = (p15 == 0 ? 10 : p15)
    i_ifn_section_env = p16 
    i_section_start_time = p17
    i_duration_leeway = p19
    i_section_duration = p18 + p19
    i_section_end = i_section_start_time + i_section_duration
    iSafetyBuffer = p20
\end{lstlisting}

Ogni parametro ha un significato musicale preciso:

\begin{itemize}
    \item \texttt{ i\_CAttacco}  e \texttt{i\_Durata}: definiscono la finestra temporale in cui il comportamento è attivo
    \item \texttt{i\_RitmiTab}: punta a una tabella contenente la sequenza di valori ritmici che determinano sia la temporalità che le frequenze degli eventi
    \item \texttt{i\_DurataArmonica}: il valore di riferimento per il calcolo delle durate reali degli eventi
    \item \texttt{i\_Ottava} e \texttt{i\_Registro}: coordinate nello spazio delle altezze di partenza
    \item \texttt{i\_ottava\_arrivo} e \texttt{i\_registro\_arrivo}: destinazione per eventuali glissandi
    \item \texttt{i\_NonlinearMode}: seleziona l'algoritmo di generazione per nuovi ritmi
\end{itemize}
\subsubsection{Gestione Adattiva della Durata e dei Confini di Sezione}
Un aspetto cruciale per la coerenza musicale è la gestione degli eventi che superano i confini della loro sezione. Il parametro iSafetyBuffer attiva una logica di controllo fondamentale:

\begin{lstlisting}[language=C]
    if i_EventAttack + i_EventDuration > i_section_end then 
        if iSafetyBuffer == 1 then
            i_EventDuration = i_section_end - i_EventAttack + random:i(0, i_duration_leeway)
        endif
    endif
\end{lstlisting}

Quando un evento sta per \textit{sforare} la fine della sezione (definita da \texttt{i\_section\_end}), la sua durata viene troncata per terminare esattamente al confine. Inoltre, viene aggiunto un piccolo tempo casuale (\texttt{i\_duration\_leeway}) per evitare che tutti gli eventi terminino bruscamente allo stesso istante, creando una fine più organica e meno artificiale. Questa logica, controllata da Python, è essenziale per assemblare sezioni consecutive senza creare sovrapposizioni o troncamenti sonori indesiderati (qualora si manteng il safety buffer attivo).
\subsubsection{Il Loop Generativo Principale}
Il cuore dello strumento Voce è un loop while che genera eventi fino al raggiungimento della durata specificata:

\begin{lstlisting}[language=C]
i_EventIdx = 0
i_whileTime = 0

while i_whileTime < i_Durata do
    ; -------- 3.1 GESTIONE RITMI --------
    if i_EventIdx < i_LenRitmiTab then
        i_RitmoCorrente tab_i i_EventIdx, i_TempRitmiTab
        if i_RitmoCorrente == 0 then
            goto generateNewRhythm
        endif
        i_Vecchio_Ritmo = (i_EventIdx == 0) ? 1 : tab_i(i_EventIdx - 1, i_TempRitmiTab)
    else
        generateNewRhythm:
        i_Vecchio_Ritmo tab_i i_EventIdx - 1, i_TempRitmiTab
        i_RitmoCorrente NonlinearFunc i_Vecchio_Ritmo, i_NonlinearMode
        tabw_i i_RitmoCorrente, i_EventIdx, i_TempRitmiTab
    endif
\end{lstlisting}

Questo codice implementa una logica sofisticata: inizialmente legge i ritmi dalla tabella fornita, ma quando questa si esaurisce, genera nuovi valori usando l'opcode \texttt{NonlinearFunc}, creando potenzialmente sequenze infinite che evolvono secondo regole caotiche o deterministiche.
\subsubsection{Calcolo Temporale degli Eventi}
Il timing di ogni evento dipende dal ritmo precedente secondo la formula:

\begin{lstlisting}[language=C]
if i_EventIdx == 0 then
    i_EventAttack = i_CAttacco
else
    i_RitmoNormalizzato = 1 / i_Vecchio_Ritmo
    i_PreviousAttack tab_i gi_Index - 1, gi_eve_attacco
    i_EventAttack = i_DurataArmonica * i_RitmoNormalizzato + i_PreviousAttack
endif
\end{lstlisting}

Questa relazione inversamente proporzionale significa che valori ritmici più alti producono eventi più ravvicinati, creando accelerazioni, mentre valori bassi generano rarefazioni temporali.
\subsubsection{Gestione della Tabella Ritmi Temporanea}
Una caratteristica importante è la creazione di una tabella temporanea estesa per i ritmi:

\begin{lstlisting}[language=C]
i_LenRitmiTab = ftlen(i_RitmiTab)
i_TempRitmiTab ftgen 0, 0, i_LenRitmiTab + 10000, -2, 0

; Copia i ritmi iniziali nella tabella temporanea
i_IndexCopy = 0
while i_IndexCopy < i_LenRitmiTab do
    i_ValRitmo tab_i i_IndexCopy, i_RitmiTab
    tabw_i i_ValRitmo, i_IndexCopy, i_TempRitmiTab
    i_IndexCopy += 1
od
\end{lstlisting}

Questo approccio permette di estendere dinamicamente la sequenza ritmica oltre i valori iniziali senza modificare la tabella originale, mantenendo la purezza dei dati di input mentre si esplora lo spazio generativo.
\subsubsection{Sistema di Scheduling degli Eventi}
La creazione effettiva degli eventi sonori avviene attraverso la chiamata a \texttt{schedule}:

\begin{lstlisting}[language=C]
schedule "eventoSonoro", i_EventAttack - p2, i_EventDuration, i_DynamicIndex, i_Freq1, i_Pos, i_RitmoCorrente, i_Freq2, i_ifnAttacco, gi_Index, i_IdComp, i_SensoMovimento, i_ifn_section_env, i_section_start_time, i_section_duration
\end{lstlisting}

Il parametro \texttt{i\_RitmoCorrente} viene passato come \texttt{p7} allo strumento eventoSonoro, dove viene letto come \texttt{iHR} che spiegherò in seguito.
\subsection{EventoSonoro: Dal Parametro al Suono}
Lo strumento \texttt{eventoSonoro} è responsabile della generazione effettiva del suono. Riceve i parametri calcolati da Voce e li trasforma in segnale audio attraverso sintesi e processamento.
\subsubsection{Sistema di Compensazione Isofonica dell'Ampiezza}
Una delle caratteristiche più sofisticate di Gamma è l'implementazione di un sistema di calibrazione dell'ampiezza basato sulle curve isofoniche ISO 226:2003. Per comprendere l'importanza di questa implementazione, è necessario esaminare il fenomeno psicoacustico che la motiva.

L'orecchio umano non percepisce tutte le frequenze con la stessa sensibilità. Un tono puro a 100 Hz deve avere un'intensità fisica significativamente maggiore di un tono a 3000 Hz per essere percepito con la stessa loudness. Le curve isofoniche mappano questa non-linearità percettiva, mostrando quali livelli di pressione sonora (SPL) sono necessari a diverse frequenze per produrre la stessa sensazione di loudness.

Lo standard ISO 226:2003 rappresenta la revisione più recente di queste curve, basata su estesi studi psicoacustici internazionali. Ogni curva rappresenta un livello di loudness costante misurato in phon, dove per definizione:
\begin{itemize}
    \item A 1000 Hz, il livello in phon equivale al livello in dB SPL
    \item A tutte le altre frequenze, il livello in dB SPL necessario varia secondo la curva
\end{itemize}

Il sistema utilizza tre tabelle fondamentali derivate dallo standard ISO:

\begin{lstlisting}[language=C]
giIsoFreqs ftgen 0, 0, 32, -2, 20, 25, 31.5, 40, 50, 63, 80, 100, 125, 160, 200, 250, 315, 400, 500, 630, 800, 1000, 1250, 1600, 2000, 2500, 3150, 4000, 5000, 6300, 8000, 10000, 12500
giAf       ftgen 0, 0, 32, -2, 0.532, 0.506, 0.480, 0.455, 0.432, 0.409, 0.387, 0.367, 0.349, 0.330, 0.315, 0.301, 0.288, 0.276, 0.267, 0.259, 0.253, 0.250, 0.246, 0.244, 0.243, 0.243, 0.243, 0.242, 0.242, 0.245, 0.254, 0.271, 0.301
giLu       ftgen 0, 0, 32, -2, -31.6, -27.2, -23.0, -19.1, -15.9, -13.0, -10.3, -8.1, -6.2, -4.5, -3.1, -2.0, -1.1, -0.4, 0.0, 0.3, 0.5, 0.0, -2.7, -4.1, -1.0, 1.7, 2.5, 1.2, -2.1, -7.1, -11.2, -10.7, -3.1
giTf       ftgen 0, 0, 32, -2, 78.5, 68.7, 59.5, 51.1, 44.0, 37.5, 31.5, 26.5, 22.1, 17.9, 14.4, 11.4, 8.6, 6.2, 4.4, 3.0, 2.2, 2.4, 3.5, 1.7, -1.3, -4.2, -6.0, -5.4, -1.5, 6.0, 12.6, 13.9, 12.3
\end{lstlisting}

Questi parametri rappresentano:
\begin{itemize}
    \item \textbf{giAf}: Esponente di loudness, determina la pendenza della funzione di trasferimento
    \item \textbf{giLu}: Livello di loudness alla soglia, rappresenta la correzione per la soglia uditiva
    \item \textbf{giTf}: Soglia uditiva in campo libero, il livello minimo udibile in condizioni ideali
\end{itemize}

Il calcolo dell'ampiezza compensata avviene in più fasi:

\begin{lstlisting}[language=C]
kamp GetIsoAmp_k i_DynamicIndex, ifreq1, ifreq2
\end{lstlisting}

Questo UDO k-rate gestisce la compensazione durante i glissandi. Per frequenze statiche, il calcolo è più diretto:

\begin{lstlisting}[language=C]
opcode GetIsoAmp, i, ii
    iFrequency, iDynamicIndex xin
    iSafeFrequency = limit(iFrequency, 20, 12500)

; 1. Recupera i parametri di base per la dinamica data
    iPhonLevel, iDbfsRef1kHz GetDynamicParams iDynamicIndex

; 2. Calcola il dB SPL target per la frequenza e il livello phon dati
    iDbSplTarget    PhonToSpl_i     iPhonLevel, iSafeFrequency

; 3. Il dB SPL di riferimento a 1kHz è per definizione uguale al livello Phon
    iDbSplRef1kHz   =               iPhonLevel

; 4. Calcola l'offset di compensazione
    iFrequencyOffset = iDbSplTarget - iDbSplRef1kHz

; 5. Applica l'offset al livello dBFS di riferimento
    iFinalDbfs      = iDbfsRef1kHz + iFrequencyOffset

; 6. Converti il dBFS finale in ampiezza lineare
    iFinalAmp       = ampdbfs(iFinalDbfs)

xout iFinalAmp
endop
\end{lstlisting}

Invece di applicare curve di equalizzazione complesse, il sistema calcola quanto la frequenza target si discosta dal riferimento a 1kHz e applica questa differenza al livello dBFS desiderato.

La conversione da phon a SPL implementa la formula matematica dello standard:

\begin{lstlisting}[language=C]
opcode PhonToSpl_i, i, ii
    iphon, ifreq    xin

; Interpolazione lineare dalle tabelle ISO
    iaf             Interp  ifreq, giIsoFreqs, giAf
    ilu             Interp  ifreq, giIsoFreqs, giLu
    itf             Interp  ifreq, giIsoFreqs, giTf

; Formula ISO 226:2003
    iterm1          =       4.47e-3 * (pow(10, 0.025 * iphon) - 1.15)
    iterm2_exp      =       (itf + ilu) / 10.0 - 9
    iterm2          =       pow(0.4 * pow(10, iterm2_exp), iaf)
    iaf_value       =       iterm1 + iterm2

if iaf_value <= 0 then
        ispl        =       itf + (iphon / 40.0) * 20
    else
        ispl        =       (10.0 / iaf) * log10(iaf_value) - ilu + 94.0
    endif

if abs(ifreq - 1000) < 0.1 then
        ispl = iphon
    endif

xout            ispl
endop
\end{lstlisting}

La formula si divide in due termini:
\begin{itemize}
    \item \textbf{iterm1}: Rappresenta la componente lineare della loudness, dominante a livelli alti
    \item \textbf{iterm2}: Cattura la non-linearità vicino alla soglia uditiva
\end{itemize}

Il caso speciale \texttt{if iaf\_value <= 0} gestisce situazioni vicine o sotto la soglia uditiva, dove la formula principale potrebbe produrre valori matematicamente indefiniti.

Questa implementazione garantisce che:

\begin{enumerate}
    \item \textbf{Coerenza Percettiva}: Un evento marcato come \texttt{mf} (mezzoforte) mantiene la stessa loudness percepita indipendentemente dalla sua frequenza
    \item \textbf{Glissandi Naturali}: Durante un glissando, l'ampiezza viene continuamente aggiustata per compensare i cambiamenti di sensibilità dell'orecchio
    \item \textbf{Bilanciamento Automatico}: In texture polifoniche, eventi in registri diversi mantengono bilanciamento percettivo senza intervento manuale
\end{enumerate}
Per esempio, un evento a 100 Hz marcato come \texttt{f} (forte) riceverà automaticamente più energia di uno a 3000 Hz con la stessa dinamica, compensando la minore sensibilità dell'orecchio alle basse frequenze. Questa compensazione è particolarmente critica nel sistema pitagorico di Gamma, dove le frequenze generate possono spaziare su tutto lo spettro udibile.

L'implementazione k-rate per i glissandi assicura che questa compensazione avvenga continuamente:

\begin{lstlisting}[language=C]
if iAmpStart > iAmpEnd then
    kf expseg  1, p3, 0.0001
    kFinalAmp = (kf * (iAmpStart-iAmpEnd))+iAmpEnd
elseif iAmpStart < iAmpEnd then
    kf expseg  0.0001, p3, 1
    kFinalAmp = (kf * (iAmpEnd-iAmpStart))+iAmpStart
\end{lstlisting}

L'uso di segmenti esponenziali invece che lineari è per la natura logaritmica della percezione dell'ampiezza, creando transizioni che appaiono lineari all'ascolto.
\subsubsection{Sistema di Spazializzazione Mid-Side e Armoniche Spaziali}
La spazializzazione in Gamma va oltre il semplice panning stereofonico, implementando un sistema basato su \textit{armoniche spaziali} di mia ideazione che deriva dalla teoria delle armoniche ritmiche. Il concetto chiave è che i valori ritmici non solo organizzano il tempo e selezionano le frequenze, ma definiscono anche il movimento nello spazio stereofonico.

Vediamo come si sviluppa questo sistema partendo dai parametri di base:

\begin{lstlisting}[language=C]
; Parametri di base per la spazializzazione
iwhichZero = abs(p6)    ; quale "zero" della funzione trigonometrica usare
iHR = max(1, abs(p7))   ; Harmonic Ratio - il numero di "spicchi" della circonferenza

; Calcolo del periodo e della posizione iniziale
iPeriod = $M_PI * 2 / iHR
iradi = (iwhichZero > 0 ? (iwhichZero - 1) * iPeriod : 0)
\end{lstlisting}

Il parametro \texttt{iHR} (Harmonic Ratio) determina in quanti \textit{spicchi} viene suddivisa la circonferenza. Ad esempio:
\begin{itemize}
    \item \texttt{iHR = 1}: un solo periodo, movimento completo 0-360°
    \item \texttt{iHR = 4}: quattro periodi, la circonferenza è divisa in quadranti
    \item \texttt{iHR = 7}: sette spicchi, creando una suddivisione asimmetrica
\end{itemize}

Il parametro \texttt{iwhichZero} determina da quale zero della funzione trigonometrica iniziare il movimento:

\begin{lstlisting}[language=C]
; Evoluzione temporale della posizione angolare
kndx_local line 0, p3, 1
ktab tab kndx_local, ifn_shape, 1
krad = iradi + (ktab * iPeriod * i_senso)
\end{lstlisting}

Qui \texttt{krad} evolve nel tempo secondo l'inviluppo specificato da \texttt{ifn\_shape}, modulato dal senso di movimento (\texttt{i\_senso} = 1 o -1 per movimento orario/antiorario).

La generazione dell'inviluppo locale usa una modifica della funzione seno quando \texttt{ifn\_shape == 2}:

\begin{lstlisting}[language=C]
if ifn_shape == 2 then
    kEnv_local = abs(sin(krad * iHR / 2))
else
    kEnv_local tab kndx_local, ifn_shape, 1
endif
\end{lstlisting}

La formula \texttt{abs(sin(krad * iHR / 2))} genera curve polari modificate. Questa trasformazione:
\begin{itemize}
    \item Prende il valore assoluto, creando lobi sempre positivi
    \item Moltiplica per \texttt{iHR / 2}, dimezzando il numero di lobi rispetto agli spicchi spaziali
    \item Crea una correlazione diretta tra movimento spaziale e ampiezza
\end{itemize}

Per comprendere meglio, consideriamo il codice Python fornito che visualizza queste funzioni:

\begin{lstlisting}[language=Python]
def genera_e_plotta_polare_sine(self):
    theta = np.linspace(0, 2 * np.pi, 500)
    num_funzioni = 10
\section{Base delle funzioni sinusoidali}
    r_base = [np.abs(np.sin(theta * i / 2)) for i in range(1, num_funzioni + 1)]
\end{lstlisting}

Questo mostra come per \texttt{i} crescenti si ottengono curve polari con sempre più lobi, che in Csound diventano pattern di inviluppo sempre più complessi.

La conversione finale da coordinate polari a stereo avviene con:

\begin{lstlisting}[language=C]
; Calcolo delle componenti Mid-Side
kMid = cos(krad)
kSide = sin(krad)

; Applicazione dell'inviluppo al segnale
aMid = kMid * asigEnv 
aSide = kSide * asigEnv

; Conversione a Left-Right con matrice di rotazione
aL = (aMid + aSide) / $SQRT2
aR = (aMid - aSide) / $SQRT2
\end{lstlisting}
\subsubsection{Gestione degli Inviluppi Multipli}
Il sistema gestisce due livelli di inviluppo che interagiscono moltiplicativamente:

\begin{lstlisting}[language=C]
; Inviluppo locale dell'evento (derivato dalle armoniche spaziali)
asigLocalEnv = asig * kEnv_local

; Inviluppo di sezione (se presente)
kEnv_section = 1
if i_ifn_section_env > 20 && i_section_duration > 0 then
    k_time_absolute times      
    k_time_since_section_start = k_time_absolute - i_section_start_time
    kndx_section = limit(k_time_since_section_start / i_section_duration, 0, 1)
    kEnv_section tablei kndx_section, i_ifn_section_env, 1
endif

; Combinazione degli inviluppi
asigEnvPre = asigLocalEnv * kEnv_section
asigEnv dcblock asigEnvPre
\end{lstlisting}

L'inviluppo di sezione permette modulazioni globali su tutti gli eventi di una sezione, mentre l'inviluppo locale (potenzialmente derivato dalle armoniche spaziali) definisce la forma del singolo evento.
\subsection{Il Sistema di Intonazione Pitagorica}
Il sistema di altezze in Gamma si basa su una implementazione personalizzata dell'intonazione pitagorica, gestita dall'opcode \texttt{GenPythagFreqs}:

\begin{lstlisting}[language=C]
opcode GenPythagFreqs, i, iiii
  iFund, iNumIntervals, iNumOctaves, iTblNum xin
  iTotalLen = iNumIntervals * iNumOctaves
  iFreqs[] init iTotalLen

iOctave = 0
  iBaseIndex = 0

while iOctave < iNumOctaves do
    iFifth = 3/2
    iFreqs[iBaseIndex] = iFund * (2^iOctave)

; Genera la serie di quinte per questa ottava
    indx = 1
    iLastRatio = 1
    while (indx < iNumIntervals) do
      iRatio = iLastRatio * iFifth
      ; Riduci all'ottava di riferimento
      while (iRatio >= 2) do
        iRatio = iRatio / 2
      od
      iFreqs[iBaseIndex + indx] = iFund * iRatio * (2^iOctave)
      iLastRatio = iRatio
      indx += 1
    od
\end{lstlisting}

Il sistema genera una tabella bidimensionale concettuale dove:
\begin{itemize}
    \item Ogni ottava contiene \texttt{iNumIntervals} frequenze (200 nel nostro caso)
    \item Le frequenze sono generate attraverso iterazioni della quinta perfetta (3/2)
    \item Ogni quinta che supera l'ottava viene riportata all'interno tramite divisione per 2
\end{itemize}

Dopo la generazione, le frequenze vengono ordinate all'interno di ogni ottava:

\begin{lstlisting}[language=C]
; Ordina le frequenze per questa ottava
indx = iBaseIndex
while (indx < (iBaseIndex + iNumIntervals - 1)) do
  indx2 = indx + 1
  while (indx2 < (iBaseIndex + iNumIntervals)) do
    if (iFreqs[indx2] < iFreqs[indx]) then
      iTemp = iFreqs[indx]
      iFreqs[indx] = iFreqs[indx2]
      iFreqs[indx2] = iTemp
    endif
    indx2 += 1
  od
  indx += 1
od
\end{lstlisting}

Questo bubble sort garantisce che le frequenze siano accessibili in ordine crescente all'interno di ogni ottava.
\subsubsection{Mappatura Ottava-Registro-Ritmo}
L'accesso alle frequenze avviene attraverso la funzione \texttt{calcFrequenza}:

\begin{lstlisting}[language=C]
opcode calcFrequenza, i, iii
    i_Ottava, i_Registro, i_RitmoCorrente xin

; Calculate octave register
    i_Indice_Ottava = int(i_Ottava * $INTERVALLI)
    ; Calculate interval offset within the octave
    i_OffsetIntervallo = i_Indice_Ottava + int(((i_Registro * $INTERVALLI) / $REGISTRI))

; Get the frequency from the table using the calculated offset
    i_Freq table max(1, i_OffsetIntervallo + i_RitmoCorrente), gi_Intonazione
    ifreq = min(i_Freq, sr/2-1)
    xout ifreq
endop
\end{lstlisting}

La formula di indicizzazione \texttt{i\_OffsetIntervallo + i\_RitmoCorrente} crea una relazione diretta tra il valore ritmico e l'altezza selezionata. Questo significa che:

\begin{itemize}
    \item Ritmi identici in registri diversi producono intervalli correlati
    \item La sequenza ritmica diventa una sequenza melodica
    \item Valori ritmici alti tendono verso frequenze più acute all'interno del registro
\end{itemize}
\subsubsection{Implicazioni Compositive}
Questa architettura crea una profonda interconnessione tra dimensione temporale, frequenziale e spaziale. Un pattern ritmico [3, 5, 8, 13] non solo definisce:
\begin{itemize}
    \item Le durate primarie relative degli eventi (durataArmonica/3, durataArmonica/5, etc.) ( primarie poiché trasfigurate successivamente da un moltiplicatore di durata).
    \item Le altezze selezionate dalla tabella pitagorica
    \item Il numero di suddivisioni spaziali e il pattern di movimento stereofonico
    \item La forma dell'inviluppo di ampiezza quando si usano le armoniche spaziali
\end{itemize}

L'uso dell'intonazione pitagorica invece del temperamento equabile aggiunge ulteriore ricchezza armonica: le quinte sono pure (rapporto 3:2), ma questo genera comma pitagorici e intervalli microtonali che colorano il risultato sonoro con battimenti e risonanze particolari.

La gerarchia Voce → eventoSonoro, supportata dal sistema di intonazione pitagorica, dalle tecniche di compensazione isofonica e dal sistema di armoniche spaziali, fornisce al compositore uno strumento di straordinaria flessibilità espressiva, capace di generare texture complesse da specifiche relativamente semplici.
\subsection{NonlinearFunc: Il Generatore di Ritmi Caotici}
L'opcode \texttt{NonlinearFunc} rappresenta un sistema per la generazione di sequenze ritmiche che evolvono nel tempo secondo principi deterministici, periodici o caotici. Questo UDO (User Defined Opcode) estende le possibilità compositive oltre i pattern ritmici predefiniti, permettendo l'esplorazione di territori ritmici emergenti.
\subsubsection{Struttura e Parametri dell'Opcode}
\begin{lstlisting}[language=C]
opcode NonlinearFunc, i, ippo
  iX, iMode, iMinVal, iMaxVal xin

; Valori di default per min/max se non specificati
  iMinVal = (iMinVal == 0) ? 1 : iMinVal
  iMaxVal = (iMaxVal == 0) ? 35 : iMaxVal

; Assicurati che iX sia entro limiti sensati
  iX = limit(iX, 1, 100)

iPI = 4 * taninv(1.0)  
  iTemp = 0
\end{lstlisting}

L'opcode accetta quattro parametri:
\begin{itemize}
    \item \texttt{iX}: Il valore di input, tipicamente il ritmo precedente nella sequenza
    \item \texttt{iMode}: Selettore della modalità operativa (0-3)
    \item \texttt{iMinVal}: Valore minimo del range di output (default: 1)
    \item \texttt{iMaxVal}: Valore massimo del range di output (default: 35)
\end{itemize}

La prima operazione importante è la normalizzazione e limitazione dei valori di input per garantire stabilità numerica. Il valore di iX viene limitato tra 1 e 100 per evitare overflow o comportamenti indefiniti nelle funzioni matematiche successive.
\subsubsection{Modalità 0: Convergente}
\begin{lstlisting}[language=C]
if iMode == 0 then
    ; --- MODALITÀ 0: CONVERGENTE ---
    iR = 2.8
    iTemp = iR * iX * (1 - iX/40)
\end{lstlisting}

Questa modalità implementa una variante della mappa logistica con comportamento convergente. Il parametro \texttt{iR = 2.8} è scelto specificamente per rimanere nella regione stabile del diagramma di biforcazione della mappa logistica, dove il sistema converge verso un punto fisso.

La formula \texttt{iR * iX * (1 {-} iX/40)} differisce dalla classica mappa logistica \texttt{r * x * (1 {-} x)} per il fattore di scala 40. Questo adattamento:
\begin{itemize}
    \item Permette di lavorare con valori di input nell'intervallo 1-100 invece di 0-1
    \item Rallenta la convergenza, rendendo l'evoluzione ritmica più graduale
    \item Crea una traiettoria prevedibile verso un valore stabile
\end{itemize}

Matematicamente, per \texttt{iR = 2.8}, il sistema convergerà verso il punto fisso:
\begin{lstlisting}
x* = 40 * (1 - 1/iR) ≈ 25.71
\end{lstlisting}

Questo significa che sequenze ritmiche in modalità convergente tenderanno gradualmente verso valori intorno a 26, creando un effetto di stabilizzazione ritmica.
\subsubsection{Modalità 1: Periodica}
\begin{lstlisting}[language=C]
elseif iMode == 1 then
    ; --- MODALITÀ 1: PERIODICA ---
    iP1 = sin(iX * iPI/18)
    iP2 = cos(iX * iPI/10)
    iTemp = abs(iP1 * iP2) * 20 + 10
\end{lstlisting}

La modalità periodica utilizza l'interferenza di due funzioni trigonometriche con periodi incommensurabili per generare pattern complessi ma deterministici.

L'analisi matematica rivela:
\begin{itemize}
    \item \texttt{sin(iX * \textbackslash\{\}texttt\{\{\$\textbackslash\{\}pi\$/18)}: periodo di 36 unità
    \item \texttt{cos(iX * \textbackslash\{\}texttt\{\{\$\textbackslash\{\}pi\$/10)}: periodo di 20 unità
    \item Il minimo comune multiplo è 180, creando un super-periodo
\end{itemize}

Il prodotto \texttt{iP1 * iP2} genera un'interferenza costruttiva e distruttiva tra le due onde:
\begin{itemize}
    \item Quando entrambe le funzioni sono vicine ai loro massimi/minimi, il prodotto è grande
    \item Quando una è vicina a zero, il prodotto si annulla
    \item Il valore assoluto garantisce output positivi
\end{itemize}

La trasformazione finale \texttt{abs(iP1 * iP2) * 20 + 10}:
\begin{itemize}
    \item Scala il range da [0, 1] a [0, 20]
    \item Aggiunge un offset di 10, risultando in valori tra 10 e 30
    \item Garantisce che i ritmi generati rimangano in un range musicalmente utile
\end{itemize}

Questa modalità produce sequenze che si ripetono dopo 180 iterazioni ma con una struttura interna ricca di variazioni locali.
\subsubsection{Modalità 2: Caotica Deterministica}
\begin{lstlisting}[language=C]
elseif iMode == 2 then
    ; --- MODALITÀ 2: CAOTICA DETERMINISTICA ---
    iR = 3.99
    iNormX = (iX % 100) / 100
    iNormX = limit(iNormX, 0.01, 0.99)
    iLogistic = iR * iNormX * (1 - iNormX)
    iNoise = random:i(-0.05, 0.05)
    iLogistic = limit(iLogistic + iNoise, 0, 1)
    iRange = iMaxVal - iMinVal + 1
    iTemp = iMinVal + (iLogistic * iRange)
\end{lstlisting}

Questa modalità implementa la mappa logistica nella sua regione caotica con l'aggiunta di una piccola perturbazione stocastica.

Il parametro \texttt{iR = 3.99} posiziona il sistema al limite del caos:
\begin{itemize}
    \item Per r > 3.57, la mappa logistica entra nel regime caotico
    \item A r = 3.99, siamo nella regione di caos sviluppato
    \item Piccole variazioni nell'input producono grandi divergenze nell'output
\end{itemize}

Il processo di normalizzazione \texttt{(iX \% 100) / 100}:
\begin{itemize}
    \item Utilizza l'operatore modulo per mantenere i valori ciclici
    \item Normalizza nell'intervallo [0, 1] richiesto dalla mappa logistica
    \item Il limite \texttt{[0.01, 0.99]} evita i punti fissi instabili a 0 e 1
\end{itemize}

L'aggiunta di rumore \texttt{random:i({-}0.05, 0.05)}:
\begin{itemize}
    \item Introduce una componente stocastica del 5\%
    \item Previene cicli perfetti che potrebbero emergere anche nel caos deterministico
    \item Simula le imperfezioni del mondo reale
\end{itemize}
\subsubsection{Modalità 3: Caos Vero (Default)}
\begin{lstlisting}[language=C]
else
    ; --- MODALITÀ 3: CAOS VERO (DEFAULT) ---
    ; 1. Componente deterministica (60%)
    iSeed1 = (iX * 1.3) % 10
    iSeed2 = (iX * 0.7) % 10
    iSeed3 = (iX * 2.5) % 10
    iNonlinear1 = abs(sin(iSeed1 * iPI/5 + iSeed2))
    iNonlinear2 = abs(cos(iSeed2 * iPI/3 + iSeed3))
    iNonlinear3 = abs(tan(iSeed3 * iPI/7 + iSeed1) % 1)
    iDeterministic = (iNonlinear1 + iNonlinear2 + iNonlinear3) / 3
\end{lstlisting}

La modalità \textit{Caos Vero} rappresenta l'approccio più interessante, combinando molteplici generatori non lineari con componenti stocastiche.

La generazione dei seed utilizza moltiplicatori irrazionali approssimati:
\begin{itemize}
    \item 1.3 $\approx$ √1.69 
    \item 0.7 $\approx$ 1/√2
    \item 2.5 $\approx$ √6.25
\end{itemize}

Questi valori garantiscono che i tre seed evolvano a velocità diverse e incommensurabili, massimizzando la complessità dell'output.

Le tre funzioni non lineari utilizzano:
\begin{itemize}
    \item \texttt{sin} con accoppiamento additivo: sensibile alle fasi relative
    \item \texttt{cos} con accoppiamento additivo: sfasato di $\pi$/2 rispetto a sin
    \item \texttt{tan} con modulo: introduce discontinuità controllate
\end{itemize}

\begin{lstlisting}[language=C]
    ; 2. Componente casuale (40%)
    iRandom = random:i(0, 1)

; 3. Combina le componenti
    iMixRatio = 0.6
    iCombined = (iDeterministic * iMixRatio) + (iRandom * (1 - iMixRatio))
\end{lstlisting}

Il bilanciamento 60/40 tra deterministico e stocastico è calibrato per:
\begin{itemize}
    \item Mantenere una struttura riconoscibile (componente deterministica)
    \item Introdurre sufficiente imprevedibilità (componente random)
    \item Evitare sia la monotonia che il rumore bianco
\end{itemize}

\begin{lstlisting}[language=C]
    ; 4. Perturbazione periodica
    iPerturbation = 0
    if (iX % 7 == 0) then 
      iPerturbation = random:i(-0.3, 0.3)
    endif

; 5. Mappa al range finale
    iRange = iMaxVal - iMinVal + 1
    iTemp = iMinVal + (iCombined * iRange) + (iPerturbation * iRange)
\end{lstlisting}

La perturbazione periodica ogni 7 iterazioni:
\begin{itemize}
    \item Introduce eventi rari ma significativi
    \item Il numero 7 (primo) evita risonanze con altri periodi nel sistema
    \item L'ampiezza ±30\% può causare salti drammatici nel ritmo
\end{itemize}
\subsubsection{Integrazione con il Sistema Gamma}
Nel contesto dello strumento Voce, NonlinearFunc viene chiamato quando la tabella dei ritmi predefiniti si esaurisce:

\begin{lstlisting}[language=C]
i_RitmoCorrente NonlinearFunc i_Vecchio_Ritmo, i_NonlinearMode
\end{lstlisting}

Questo crea una transizione fluida da:
\begin{enumerate}
    \item \textbf{Fase deterministica}: Ritmi composti e memorizzati in tabella
    \item \textbf{Fase generativa}: Ritmi creati algoritmicamente
\end{enumerate}
L'output di NonlinearFunc influenza direttamente:
\begin{itemize}
    \item \textbf{Temporalità}: Attraverso la formula \texttt{i\_DurataArmonica / i\_RitmoCorrente}
    \item \textbf{Altezza}: Il ritmo viene usato come indice nella tabella delle frequenze
    \item \textbf{Spazializzazione}: Determina il parametro iHR per le armoniche spaziali
\end{itemize}  % Auto-generated: include sezione1.tex
% sezione2.tex

\section{Seconda Sezione}
Questo è il testo della seconda sezione del documento.
  % Auto-generated: include sezione2.tex
% --- Contenuto LaTeX autogenerato da capitolo3.md (sezione 4) ---

\section{CAPITOLO 3: ARCHITETTURA DEL COMPOSITORE GENERATIVO}
Il motore Python di Gamma rappresenta l'intelligenza orchestrativa del sistema, traducendo le specifiche compositive ad alto livello in eventi sonori concreti. Per comprendere come questa trasformazione avvenga, è necessario esplorare l'architettura software sottostante, un'architettura che riflette anni di raffinamento iterativo e bilancia sapientemente requisiti apparentemente contraddittori: la necessità di controllo deterministico con la flessibilità generativa, l'efficienza computazionale con la ricchezza espressiva, la complessità interna con la semplicità d'uso.
\section{3.1 Design Pattern e Struttura delle Classi}
L'architettura di Gamma si fonda su tre classi principali, ciascuna incarnando un aspetto fondamentale del processo compositivo. Questa tripartizione non è casuale, ma riflette una profonda comprensione di come la composizione algoritmica si articoli in domini distinti ma interconnessi.
\subsection{La Classe TimeScheduler: Il Tempo come Materiale Compositivo}
Il tempo, in musica, non è semplicemente il contenitore degli eventi, ma un materiale compositivo a pieno titolo. La classe \texttt{TimeScheduler} incarna questa filosofia, trasformando modelli matematici astratti in distribuzioni temporali musicalmente significative:

\begin{lstlisting}[language=Python]
class TimeScheduler:
    def generate_onsets(self, model, duration, num_events):
        if num_events == 0: return []
        if num_events == 1: return [0.0]

base_progress = np.linspace(0, 1, num_events, endpoint=False)
        final_progress = np.zeros_like(base_progress)
        model_type = model.get('type', 'linear')
\end{lstlisting}

La decisione di lavorare con una progressione normalizzata nell'intervallo [0, 1] è particolarmente significativa. Questo approccio, che potrebbe sembrare una semplice scelta implementativa, rivela in realtà una comprensione profonda della natura scalare del tempo musicale. Una frase che accelera dal pianissimo al fortissimo in 10 secondi segue la stessa curva di una che lo fa in 60 secondi - cambia la scala temporale, non la forma del gesto. Normalizzando la progressione, TimeScheduler cattura questa invarianza gestaltica.

L'uso di \texttt{endpoint=False} merita particolare attenzione. Questa scelta apparentemente minore previene un problema sottile ma critico: se l'ultimo evento di una sezione coincidesse esattamente con l'inizio della successiva, si creerebbero sovrapposizioni non intenzionali. È un esempio di come l'architettura di Gamma incorpori la saggezza pratica acquisita attraverso l'uso reale del sistema.
\subsection{La Classe GenerativeComposer: L'Orchestratore Invisibile}
Se TimeScheduler è il cronometrista, \texttt{GenerativeComposer} è il direttore d'orchestra - invisibile ma onnipresente, coordinando ogni aspetto della performance generativa:

\begin{lstlisting}[language=Python]
class GenerativeComposer:
    def __init__(self, output_dir=''composizioni_generate'', tables_config_path=''yaml/tables.yaml''):
        self.base_path = Path(__file__).parent.resolve()
        self.output_path = self.base_path / output_dir
        self.time_scheduler = TimeScheduler()
\section{Stato globale per mappatura tabelle}
        self.rhythm_table_map = {}
        self.next_table_id = 1000
        self.id_comp_counter = 0
\end{lstlisting}

L'inizializzazione della classe rivela immediatamente diverse scelte architetturali cruciali. L'uso di \texttt{Path(\_\_file\_\_).parent.resolve()} non è solo una questione di robustezza del codice - riflette la natura distribuita del sistema Gamma, dove file Python, Csound, YAML e WAV devono coesistere in una struttura gerarchica precisa. Risolvendo i percorsi in modo assoluto fin dall'inizio, il sistema previene un'intera classe di errori legati ai percorsi relativi che potrebbero emergere quando lo script viene eseguito da directory diverse.

La scelta di iniziare gli ID delle tabelle da 1000 rivela una comprensione profonda dell'ecosistema Csound. Le tabelle con numeri bassi sono tradizionalmente riservate per usi speciali o predefiniti. Partendo da 1000, Gamma si assicura uno spazio di numerazione pulito, evitando conflitti anche in orchestrazioni Csound complesse che potrebbero avere le proprie tabelle predefinite.

Ma è nella gestione della mappatura dei ritmi che vediamo l'eleganza dell'architettura:

\begin{lstlisting}[language=Python]
rhythm_tuple = tuple(params['ritmi'])
if rhythm_tuple not in self.rhythm_table_map:
    self.rhythm_table_map[rhythm_tuple] = {
        'ritmi_tab_num': self.next_table_id,
        'pos_tab_num': self.next_table_id + 1
    }
    self.next_table_id += 2
\end{lstlisting}

Questo frammento implementa una forma sofisticata di memoizzazione. In una composizione tipica, certi pattern ritmici tendono a ripetersi - non per mancanza di immaginazione, ma perché la ripetizione e la variazione sono principi compositivi fondamentali. Invece di creare nuove tabelle Csound per ogni istanza di un pattern, il sistema riconosce pattern identici e riutilizza le tabelle esistenti. In una composizione di 30 minuti con centinaia di eventi, questo può ridurre il numero di tabelle da migliaia a poche decine, con conseguenti benefici in termini di memoria e tempo di inizializzazione.
\subsection{La Classe CompositionDebugger: Vedere per Comporre}
La presenza di una classe dedicata alla visualizzazione non è un ripensamento o un'aggiunta tardiva, ma riflette una verità fondamentale della composizione algoritmica: quando i processi generativi creano migliaia di eventi, la visualizzazione diventa essenziale per comprendere e controllare il risultato:

\begin{lstlisting}[language=Python]
class CompositionDebugger:
    def __init__(self, output_dir):
        self.output_path = Path(output_dir)
        self._labels_added = set()
\end{lstlisting}

L'attributo \texttt{\_labels\_added} illustra l'attenzione ai dettagli che permea il sistema. In composizioni con molti layer, ogni layer potrebbe teoricamente aggiungere la propria etichetta alla legenda del grafico. Senza controllo, una composizione con 20 layer che usano tutti ''ottava'' creerebbe 20 voci identiche nella legenda. Il set traccia quali etichette sono già state aggiunte, mantenendo i grafici leggibili anche per le composizioni più complesse.
\subsection{I Pattern Nascosti nell'Architettura}
Analizzando l'architettura nel suo insieme, emergono diversi design pattern classici, implementati non per aderenza dogmatica a best practice, ma perché risolvono naturalmente i problemi specifici del dominio compositivo.

Il **Factory Pattern** emerge nella generazione di parametri dalle maschere. Ogni tipo di maschera (range, choices, mean/std, value) richiede una logica di generazione diversa, ma il codice chiamante non deve preoccuparsene - chiede semplicemente ''dammi un parametro da questa maschera'' e riceve il valore appropriato.

Il **Strategy Pattern** si manifesta nei modelli temporali. Che si tratti di distribuzione lineare, accelerando, ritardando o stocastica, l'interfaccia rimane identica - solo l'algoritmo interno cambia. Questo permette ai compositori di sperimentare con diversi modelli temporali semplicemente cambiando una stringa nel file YAML.

Il **Builder Pattern** appare nella costruzione incrementale dei file CSD. Invece di generare l'intero file in un colpo solo, il sistema lo costruisce pezzo per pezzo - prima le tabelle degli inviluppi, poi quelle dei ritmi, infine gli eventi - permettendo flessibilità e estensibilità.
\section{3.2 Pipeline di Elaborazione}
Il flusso di elaborazione in Gamma non è semplicemente una sequenza di operazioni, ma una coreografia attentamente orchestrata che bilancia parallelismo e sincronizzazione, efficienza e controllo.
\subsection{La Filosofia del Parallelismo Controllato}
Gamma adotta un approccio pragmatico al parallelismo. Invece di parallelizzare tutto il possibile, il sistema identifica i colli di bottiglia reali - il rendering Csound - e concentra lì gli sforzi di ottimizzazione:

\begin{lstlisting}[language=Python]
def execute_layer_rendering_and_collect_data(render_jobs, dirs, veteran_mode_active):
    csound_procs = []
    composer = GenerativeComposer()

for job in render_jobs:
\section{Generazione eventi per questo layer}
        layer_events, layer_onsets = composer._process_layer(...)

if layer_events:
            composer.generate_csd(job['name'], layer_events, job['csd_path'], job['wav_path'])
            proc_data = run_csound_process(job['csd_path'], job['name'], dirs['logs'])
            if proc_data: 
                csound_procs.append(proc_data)
\end{lstlisting}

La generazione degli eventi rimane sequenziale - è veloce e potrebbe creare problemi di sincronizzazione se parallelizzata. Ma il rendering Csound, che può richiedere secondi o minuti per layer complessi, viene eseguito in parallelo. Questo approccio ''parallelize what matters'' massimizza i benefici minimizzando la complessità.

L'uso di \texttt{subprocess.Popen} merita un approfondimento:

\begin{lstlisting}[language=Python]
process = subprocess.Popen(['csound', '--format=float', str(csd_path)], 
                         stdout=log_file, stderr=log_file)
\end{lstlisting}

\texttt{Popen} crea un nuovo processo senza attendere il suo completamento, permettendo al Python di continuare a lanciare altri processi Csound. Il reindirizzamento di stdout e stderr verso file di log individuali permette di diagnosticare problemi specifici di ogni layer senza che i messaggi si mescolino in un output confuso.
\subsection{Il Modello Fork-Join e la Sincronizzazione}
Dopo aver lanciato tutti i processi in parallelo, il sistema deve attendere il loro completamento:

\begin{lstlisting}[language=Python]
for process, name, log_file in csound_procs:
    process.wait()
    log_file.close()
    if process.returncode != 0:
        print(f''✗ ERRORE: Rendering del layer '{name}' fallito!'')
\end{lstlisting}

Questo implementa il classico modello ''fork-join'' della computazione parallela. Il ''fork'' avviene quando lanciamo i processi, il ''join'' quando li attendiamo. La semplicità di questo approccio nasconde la sua efficacia: su un sistema moderno con 8 core, 8 layer possono essere renderizzati simultaneamente, riducendo potenzialmente il tempo totale di un fattore 8.
\subsection{L'Innovazione del Veteran Mode}
Il veteran mode rappresenta una delle innovazioni più pratiche di Gamma, nata dall'esperienza diretta del workflow compositivo:

\begin{lstlisting}[language=Python]
veteran_mode_active = any(
    layer.get('veteranMode', False)
    for part in all_composition_structures
    for section in part
    for layer in section.get('layers', [])
)
\end{lstlisting}

Durante lo sviluppo di una composizione, è comune modificare ripetutamente un singolo layer mentre gli altri rimangono stabili. Senza veteran mode, ogni modifica richiederebbe il re-rendering dell'intera composizione. Con veteran mode, marcando i layer ''veterani'' (quelli già finalizzati), il sistema li salta durante il rendering, riutilizzando i file WAV esistenti. In una composizione di 20 minuti con 10 layer, modificare un singolo layer passa da 20 minuti di attesa a 2 minuti - un miglioramento 10x che trasforma il workflow da frustrante a fluido.
\subsection{La Cache Intelligente per la Visualizzazione}
Il sistema di cache rappresenta un altro esempio di ottimizzazione nata dalla pratica:

\begin{lstlisting}[language=Python]
if fresh_data.get('render_jobs_info'):
    fresh_job_keys = set(
        (job['section']['nome_sezione'], job['layer_idx']) 
        for job in fresh_jobs_info
    )
\section{Rimuovi vecchi dati per layer aggiornati}
    final_data['events'] = [
        e for e in final_data['events']
        if (e['params']['section_name'], e['params'].get('layer_idx_ref')) 
           not in fresh_job_keys
    ]
\end{lstlisting}

Questo meccanismo di merge selettivo permette di mantenere i dati di visualizzazione per i layer non modificati mentre aggiorna solo quelli cambiati. Il risultato è che anche composizioni massive possono essere ri-visualizzate in secondi invece che minuti.
\section{3.3 Gestione dello Stato Globale}
La gestione dello stato in un sistema generativo presenta sfide uniche. Da un lato, lo stato globale facilita la coordinazione tra componenti; dall'altro, può creare accoppiamenti indesiderati e complicare il testing. Gamma adotta un approccio pragmatico che bilancia questi concern.
\subsection{La Gerarchia dello Stato}
Lo stato in Gamma è organizzato gerarchicamente, riflettendo la struttura della composizione stessa:

\begin{enumerate}
    \item \textbf{Stato Globale del Sistema}: Configurazioni, mappature, contatori che persistono per l'intera sessione
    \item \textbf{Stato della Composizione}: Parametri specifici di una particolare composizione
    \item \textbf{Stato della Sezione}: Durata, tempo di inizio, envelope globale
    \item \textbf{Stato del Layer}: Maschere di tendenza, modelli temporali
    \item \textbf{Stato dell'Evento}: Parametri individuali di ogni suono generato
\end{enumerate}

Questa gerarchia non è solo concettuale ma si riflette nell'implementazione:

\begin{lstlisting}[language=Python]
\section{Stato globale}
self.rhythm_table_map = {}
self.next_table_id = 1000
\section{Stato compositivo}
absolute_section_onset = max(0.0, end_time_of_last_section + offset)
\section{Stato del layer}
layer_start_time_abs = current_time_offset + (start_ratio * scaled_section_duration)
\section{Stato dell'evento}
params['time'] = absolute_onset_time + jitter
\end{lstlisting}
\subsection{Le Mappature come Ponti Semantici}
Le varie mappature nel sistema non sono semplici dizionari, ma ponti semantici tra il mondo dei concetti musicali e quello dei valori numerici:

\begin{lstlisting}[language=Python]
\section{Mappatura simbolica degli inviluppi}
self.envelope_map = {
    name: config['number'] 
    for name, config in self.event_envelopes_config.items()
}
\section{Mappatura delle dinamiche}
self.dynamic_to_index = {
    'ppp': 0, 'pp': 1, 'p': 2, 'mf': 3, 'f': 4, 'ff': 5, 'fff': 6
}
\end{lstlisting}

La mappatura delle dinamiche, per esempio, traduce le indicazioni tradizionali della notazione musicale in indici numerici. Ma la scelta degli indici non è casuale: sono ordinati dal più piano al più forte, permettendo interpolazioni significative. Un layer che evolve da 'p' (indice 2) a 'f' (indice 4) può generare dinamiche intermedie interpolando gli indici - un 'mf' emergerebbe naturalmente a metà del percorso.
\subsection{Robustezza attraverso la Gestione dei Percorsi}
La gestione dei percorsi in Gamma dimostra come dettagli apparentemente minori possano fare la differenza tra un prototipo fragile e un sistema robusto:

\begin{lstlisting}[language=Python]
dirs = {
    'base': base_output_dir,
    'layers_wav': base_output_dir / ''wav'' / ''layers'',
    'sections_wav': base_output_dir / ''wav'' / ''sections'',
    'layers_csd': base_output_dir / ''csd'' / ''layers'',
    'sections_csd': base_output_dir / ''csd'' / ''sections'',
    'logs': base_output_dir / ''logs''
}
for d in dirs.values():
    d.mkdir(parents=True, exist_ok=True)
\end{lstlisting}

L'uso di \texttt{pathlib} invece di concatenazione di stringhe non è una questione di stile, ma di sostanza. \texttt{pathlib} gestisce automaticamente le differenze tra Windows (che usa backslash) e Unix (che usa slash), previene errori come doppie barre, e fornisce metodi utili come \texttt{mkdir(parents=True)} che crea l'intera gerarchia di directory se necessario.
\subsection{Verso un'Architettura Future-Proof}
Sebbene l'implementazione attuale sia single-threaded, l'architettura è stata progettata con un occhio al futuro. Lo stato minimamente condiviso, la predominanza di operazioni read-only, e i punti di sincronizzazione chiaramente identificati renderebbero relativamente semplice un'eventuale transizione a un'architettura multi-threaded.

L'architettura di Gamma dimostra come un design attento possa servire simultaneamente le esigenze immediate e preparare il terreno per evoluzioni future. Ogni scelta - dalla struttura delle classi alla gestione dello stato, dal modello di parallelismo al sistema di cache - riflette non solo competenza tecnica ma comprensione profonda del dominio compositivo. È questa sintesi di rigore ingegneristico e sensibilità musicale che rende Gamma non solo un sistema funzionante, ma uno strumento che amplifica genuinamente le possibilità creative del compositore.  % Auto-generated: include sezione3.tex
% --- Contenuto LaTeX autogenerato da capitolo4.md (sezione 5) ---

\section{GENERAZIONE PARAMETRICA E MASCHERE DI TENDENZA}
La generazione parametrica costituisce il processo centrale attraverso cui le specifiche compositive astratte si trasformano in valori concreti per la sintesi. Questo capitolo analizza i meccanismi che permettono questa trasformazione, con particolare attenzione al concetto di maschera di tendenza e alle tecniche di interpolazione che permettono l'evoluzione temporale dei parametri.
\subsection{Il Metodo \texttt{\_generate\_params\_from\_mask()}}
Il metodo \texttt{\_generate\_params\_from\_mask()} rappresenta il punto di convergenza tra l'astrazione compositiva e la concretezza numerica. Con oltre 200 righe di codice, questo metodo implementa la logica che trasforma le maschere di tendenza definite in YAML in parametri specifici per ogni evento sonoro.
\subsubsection{Architettura del Processo Generativo}
Il metodo inizia definendo un insieme di chiavi che richiedono gestione specializzata:

\begin{lstlisting}[language=Python]
SKIPPED_KEYS = {
    'choices', 'weights', 'distribution',  # Metadati
    'dynamic_index', 'dinamica', 'nonlinear_mode',  # Gestiti separatamente
    'senso_movimento', 'inviluppo_attacco', 'tipo_ritmi', 'densita_cluster'
}
\end{lstlisting}

Questa distinzione è necessaria perché alcuni parametri non possono essere gestiti dal loop generico di generazione. Le chiavi di metadati come \texttt{choices} e \texttt{weights} non sono parametri in sé, ma descrivono come generare altri parametri. Altri, come \texttt{dinamica} e \texttt{tipo\_ritmi}, richiedono logiche di trasformazione complesse che vanno oltre la semplice generazione numerica.
\subsubsection{Le Quattro Modalità di Generazione}
Il cuore del metodo è un loop che itera su ogni parametro della maschera, applicando la modalità di generazione appropriata:

\begin{lstlisting}[language=Python]
for key, p_mask in mask.items():
    if key in SKIPPED_KEYS: continue
\section{Percorso 1: Distribuzione normale}
    if 'mean' in p_mask and 'std' in p_mask:
        val = np.random.normal(loc=p_mask['mean'], scale=p_mask['std'])
\section{Percorso 2: Range uniforme}
    elif 'range' in p_mask:
        min_val, max_val = p_mask['range']
        if isinstance(min_val, int) and isinstance(max_val, int):
            val = random.randint(min_val, max_val)
        else:
            val = random.uniform(min_val, max_val)
\section{Percorso 3: Scelta pesata}
    elif 'choices' in p_mask:
        val = random.choices(p_mask['choices'], weights=p_mask.get('weights'), k=1)[0]
\section{Percorso 4: Valore fisso (implementato implicitamente)}
    elif 'value' in p_mask:
        val = p_mask['value']
\end{lstlisting}

Ogni modalità risponde a esigenze compositive diverse:

\textbf{Distribuzione Normale}: Utilizzata quando si desidera concentrazione attorno a un valore centrale con occasionali deviazioni. Un'ottava con \texttt{mean: 5, std: 0.5} produrrà principalmente note nell'ottava 5, con occasionali escursioni nelle ottave 4 e 6. La deviazione standard controlla quanto \textit{avventurosi} possono essere questi scostamenti.

\textbf{Range Uniforme}: Appropriata quando tutti i valori in un intervallo sono ugualmente desiderabili. La distinzione tra interi e float non è solo tecnica - un registro definito come \texttt{range: [1, 10]} (interi) produrrà salti discreti, mentre \texttt{range: [1.0, 10.0]} permetterà microtonalità.

\textbf{Scelta Pesata}: Permette distribuzioni non uniformi discrete. Una dinamica definita come \texttt{choices: ['p', 'mf', 'f'], weights: [0.5, 0.3, 0.2]} produrrà piano la metà delle volte, mezzoforte il 30\% e forte il 20\%. Questo controllo statistico permette di definire il \textit{colore dinamico} generale mantenendo varietà.
\subsubsection{Gestione di Parametri Speciali}
Alcuni parametri richiedono logiche di generazione che vanno oltre i quattro percorsi standard. La dinamica, per esempio, richiede una gestione particolare per supportare sia layer statici che dinamici:

\begin{lstlisting}[language=Python]
\section{Priorità 1: L'indice è già stato calcolato dalla funzione di interpolazione}
if 'dynamic_index' in mask:
    params['dynamic_index'] = mask['dynamic_index']
else:
\section{Priorità 2: Generazione dalla maschera 'dinamica'}
    dynamic_mask = mask.get('dinamica')
    if dynamic_mask:
        if 'choices' in dynamic_mask:
            dynamic_str = random.choices(
                dynamic_mask['choices'], 
                weights=dynamic_mask.get('weights'), 
                k=1
            )[0]
            params['dynamic_index'] = self.dynamic_to_index.get(dynamic_str, 3)
        elif 'value' in dynamic_mask:
            dynamic_str = dynamic_mask['value']
            params['dynamic_index'] = self.dynamic_to_index.get(dynamic_str, 3)
        else:
            params['dynamic_index'] = 3  # Default 'mf'
    else:
        params['dynamic_index'] = 3  # Default 'mf'
\end{lstlisting}

Questa logica a cascata gestisce tre casi: layer dinamici dove l'indice è pre-calcolato dall'interpolazione, layer statici con dinamica specificata, e il caso default. La complessità riflette la necessità di supportare diversi workflow compositivi senza sacrificare la coerenza del sistema.
\subsubsection{Sistema di Generazione dei Ritmi}
La generazione dei ritmi dimostra come il sistema supporti molteplici livelli di astrazione:

\begin{lstlisting}[language=Python]
rhythm_mask = mask.get('tipo_ritmi', {'choices': ['medi']})

if 'explicit_values' in rhythm_mask:
\section{Modalità 1: Lista esplicita}
    params['ritmi'] = rhythm_mask['explicit_values']

elif 'choices' in rhythm_mask:
    choice = random.choices(rhythm_mask['choices'], 
                          weights=rhythm_mask.get('weights'), k=1)[0]

if isinstance(choice, list):
\section{Modalità 2a: Scelta tra liste predefinite}
        params['ritmi'] = choice
    else:
\section{Modalità 2b: Scelta tra categorie}
        params['ritmi'] = self._generate_rhythm_pattern(choice)
\end{lstlisting}

Tre modalità permettono diversi gradi di controllo:
\begin{enumerate}
    \item \textbf{Valori Espliciti}: \texttt{[3, 5, 8, 13]} - controllo totale
    \item \textbf{Liste Predefinite}: Scelta tra pattern completi
    \item \textbf{Categorie}: 'piccoli', 'medi', 'grandi' - astrazione massima
\end{enumerate}
\subsubsection{Validazione e Retry}
La generazione include un meccanismo di retry per garantire parametri validi:

\begin{lstlisting}[language=Python]
for attempt in range(10):
    params = self._generate_params_from_mask(event_mask)
    if self._valida_parametri(params):
        break
else:
    print(f"ATTENZIONE: Impossibile generare parametri validi")
\end{lstlisting}

Dieci tentativi bilanciano la probabilità di successo con la necessità di evitare loop infiniti. La validazione verifica vincoli come la durata minima degli eventi e la coerenza delle frequenze.
\subsection{Interpolazione per Layer Dinamici}
L'interpolazione delle maschere permette l'evoluzione graduale dei parametri nel tempo, trasformando specifiche statiche in processi dinamici. Il metodo \texttt{\_interpolate\_mask()} implementa questa trasformazione con attenzione particolare alla gestione di casi limite e parametri eterogenei.
\subsubsection{Gestione delle Maschere Asimmetriche}
Un problema comune nei sistemi di interpolazione è la gestione di parametri presenti solo in uno dei due stati. Gamma risolve questo con una strategia di \textit{riempimento}:

\begin{lstlisting}[language=Python]
def _interpolate_mask(self, start_mask, end_mask, progress):
    interp_mask = {}
    all_keys = set(start_mask.keys()) | set(end_mask.keys())

for key in all_keys:
        s_mask = start_mask.get(key)
        e_mask = end_mask.get(key)
\section{Gestione parametri asimmetrici}
        if s_mask is None:
            s_mask = e_mask
        if e_mask is None:
            e_mask = s_mask
\end{lstlisting}

Se un parametro esiste solo nello stato finale, viene utilizzato per l'intera durata. Questo permette di introdurre gradualmente nuovi parametri senza dover ridefinire l'intero stato iniziale.
\subsubsection{Strategie di Interpolazione per Tipo}
Diversi tipi di parametri richiedono strategie di interpolazione diverse:

\textbf{Parametri Numerici (Range)}:
\begin{lstlisting}[language=Python]
if 'range' in s_mask:
    s_min, s_max = s_mask['range']
    e_min, e_max = e_mask.get('range', s_mask['range'])
    i_min = s_min + (e_min - s_min) * shaped_progress
    i_max = s_max + (e_max - s_max) * shaped_progress
    interp_mask[key]['range'] = [i_min, i_max]
\end{lstlisting}

L'interpolazione lineare dei limiti del range permette transizioni fluide. Un'ottava che evolve da \texttt{range: [3, 4]} a \texttt{range: [6, 8]} vedrà sia il centro che l'ampiezza del range cambiare gradualmente.

\textbf{Distribuzioni Normali}:
\begin{lstlisting}[language=Python]
elif 'mean' in s_mask:
    i_mean = s_mask['mean'] + (e_mean - s_mask['mean']) * shaped_progress
    i_std = s_mask['std'] + (e_std - s_mask['std']) * shaped_progress
    interp_mask[key]['mean'] = i_mean
    interp_mask[key]['std'] = i_std
\end{lstlisting}

Interpolando sia media che deviazione standard, il sistema può creare effetti come un focus progressivo (std decrescente) o una dispersione graduale (std crescente).

\textbf{Scelte Discrete (Choices)}:
\begin{lstlisting}[language=Python]
elif 'choices' in s_mask:
    s_choices = s_mask['choices']
    e_choices = e_mask.get('choices', s_choices)
    s_weights = np.array(s_mask.get('weights', [1]*len(s_choices)))
    e_weights = np.array(e_mask.get('weights', [1]*len(e_choices)))

if s_choices == e_choices and len(s_weights) == len(e_weights):
        i_weights = s_weights * (1 - shaped_progress) + e_weights * shaped_progress
        interp_mask[key] = {'choices': s_choices, 'weights': i_weights.tolist()}
    else:
        interp_mask[key] = s_mask if shaped_progress < 0.5 else e_mask
\end{lstlisting}

Quando le liste di scelte sono identiche, i pesi vengono interpolati creando un cross-fade probabilistico. Altrimenti, si usa una transizione a scalino al punto medio.
\subsubsection{Shaping delle Curve di Interpolazione}
Il sistema supporta curve di interpolazione non lineari attraverso il parametro \texttt{interp\_shape}:

\begin{lstlisting}[language=Python]
shape = e_mask.get('interp_shape', 1.0)
shaped_progress = progress ** shape
\end{lstlisting}

Valori di shape diversi da 1.0 creano curve diverse:
\begin{itemize}
    \item \texttt{shape < 1.0}: Cambiamenti rapidi all'inizio, poi rallentamento
    \item \texttt{shape > 1.0}: Inizio lento, accelerazione verso la fine
    \item \texttt{shape = 1.0}: Interpolazione lineare standard
\end{itemize}

Questo controllo permette di modellare l'evoluzione temporale secondo necessità espressive specifiche.
\subsection{Sistema Gerarchico del Glissando}
Il glissando in Gamma non è semplicemente una transizione tra due frequenze, ma un sistema gerarchico che supporta molteplici modalità di specifica con priorità ben definite.
\subsubsection{La Gerarchia delle Modalità}
Il sistema implementa tre modalità di glissando con priorità decrescente:

\begin{lstlisting}[language=Python]
\section{CASO 1: MODALITÀ OFFSET (priorità massima)}
if 'offset_ottava' in mask or 'offset_registro' in mask:
    offset_ottava = params.get('offset_ottava', 0)
    offset_registro = params.get('offset_registro', 0)

ottava_arrivo_calc = params['ottava'] + offset_ottava
    registro_arrivo_calc = params['registro'] + offset_registro

params['ottava_arrivo'] = ottava_arrivo_calc
    params['registro_arrivo'] = registro_arrivo_calc
\end{lstlisting}

La modalità offset è relativa: specifica il glissando come movimento rispetto alla nota di partenza. Un \texttt{offset\_ottava: 2} produrrà sempre un salto di due ottave, indipendentemente dalla frequenza iniziale. Questo è utile per pattern che devono mantenere relazioni intervallari costanti.

\begin{lstlisting}[language=Python]
\section{CASO 2: MODALITÀ ASSOLUTA}
elif 'ottava_arrivo' in mask or 'registro_arrivo' in mask:
    params['ottava_arrivo'] = params.get('ottava_arrivo', params['ottava'])
    params['registro_arrivo'] = params.get('registro_arrivo', params['registro'])
\end{lstlisting}

La modalità assoluta specifica la destinazione finale indipendentemente dal punto di partenza. Utile quando si vuole convergere verso una specifica altezza target.

\begin{lstlisting}[language=Python]
\section{CASO 3: DEFAULT (nessun glissando)}
else:
    params['ottava_arrivo'] = params['ottava']
    params['registro_arrivo'] = params['registro']
\end{lstlisting}

Se nessuna modalità è specificata, la frequenza rimane costante.
\subsubsection{Clipping e Validazione}
Dopo il calcolo, tutti i parametri vengono validati e limitati:

\begin{lstlisting}[language=Python]
params['ottava_arrivo'] = int(round(np.clip(params['ottava_arrivo'], 
                                           OTTAVE_RANGE[0], OTTAVE_RANGE[1])))
params['registro_arrivo'] = int(np.clip(params['registro_arrivo'], 
                                       REGISTRI_RANGE[0], REGISTRI_RANGE[1]))
\end{lstlisting}

Il clipping previene valori impossibili che potrebbero causare errori in Csound o produrre frequenze fuori dal range udibile. L'arrotondamento a intero mantiene la coerenza con il sistema di indicizzazione delle frequenze.
\subsubsection{Integrazione con il Sistema di Frequenze}
I parametri di glissando si integrano strettamente con il sistema di frequenze pitagoriche. In Csound, la funzione \texttt{calcFrequenza} viene chiamata due volte:

\begin{lstlisting}[language=C]
i_Freq1 = calcFrequenza(i_Ottava, i_Registro, i_RitmoCorrente)
i_Freq2 = calcFrequenza(i_ottava_arrivo, i_registro_arrivo, i_RitmoCorrente)
\end{lstlisting}

Il fatto che \texttt{i\_RitmoCorrente} sia usato per entrambe le frequenze crea una coerenza armonica: il glissando mantiene la stessa posizione relativa all'interno della scala pitagorica, creando intervalli coerenti anche durante il movimento.
\subsubsection{Implicazioni Compositive}
Il sistema gerarchico del glissando permette diversi approcci compositivi:

\begin{enumerate}
    \item \textbf{Glissandi Strutturali}: Usando offset, si possono creare pattern di movimento coerenti attraverso diverse altezze iniziali
    \item \textbf{Convergenze Armoniche}: Con destinazioni assolute, molteplici voci possono convergere verso punti focali comuni
    \item \textbf{Texture Statiche}: L'assenza di parametri di glissando crea tessiture stabili
\end{enumerate}
La combinazione di queste possibilità in layer diversi permette la creazione di texture complesse dove alcuni elementi si muovono mentre altri rimangono fermi, o dove movimenti paralleli e contrari coesistono.

Il sistema di generazione parametrica di Gamma dimostra come la complessità tecnica possa servire la semplicità compositiva. Attraverso un'architettura stratificata che separa la specifica dall'implementazione, il sistema permette ai compositori di lavorare al livello di astrazione più appropriato per le loro necessità espressive, mentre il motore sottostante si occupa di tradurre queste specifiche in parametri concreti per la sintesi.  % Auto-generated: include sezione4.tex
% --- Contenuto LaTeX autogenerato da capitolo5.md (sezione 6) ---

\section{Scolpire il Tempo: Modelli Temporali e Micro-ritmica}
Se i parametri definiscono il \textit{cosa} di un evento sonoro, il controllo temporale ne definisce il \textit{quando}. In un sistema generativo, questa dimensione assume un'importanza cruciale, poiché la distribuzione degli eventi nel tempo è uno dei principali veicoli dell'espressione musicale. Gamma affronta questa sfida con un approccio a due livelli: una macro-gestualità definita da modelli temporali astratti e una micro-ritmica che introduce variazioni organiche e \textit{umanizza} la griglia computazionale.
\subsection{La Macro-gestualità: \texttt{TimeScheduler} e i Modelli Temporali}
Il posizionamento degli eventi (o meglio, dei cluster di eventi) all'interno di un layer non è lasciato al caso. La classe \texttt{TimeScheduler} incapsula la logica per la distribuzione temporale, offrendo un toolkit di modelli che corrispondono a gesti musicali archetipici. La scelta di isolare questa funzionalità in una classe dedicata sottolinea come il tempo musicale non sia un semplice parametro, ma una dimensione fondamentale che richiede un trattamento specializzato.

Il metodo \texttt{generate\_onsets} è il cuore di questa classe. La sua architettura è elegante e flessibile: parte sempre da una progressione lineare di base, che viene poi \textit{deformata} o \textit{rimappata} secondo il modello scelto nel file YAML.

\begin{lstlisting}[language=Python]
def generate_onsets(self, model, duration, num_events):
\section{... (gestione casi base) ...}
    base_progress = np.linspace(0, 1, num_events, endpoint=False)
    final_progress = np.zeros_like(base_progress)
    model_type = model.get('type', 'linear')
\section{... applicazione del modello ...}
    return final_progress * duration
\end{lstlisting}

Questa architettura a due fasi (generazione di una progressione normalizzata e successiva trasformazione) permette di definire gesti temporali indipendentemente dalla durata effettiva, rendendoli riutilizzabili e scalabili.
\subsubsection{I Modelli Archetipici}
\begin{enumerate}
    \item Lineare (\texttt{type: linear}): Il modello di default, che distribuisce gli eventi in modo equidistante. Sebbene semplice, è fondamentale per creare pulsazioni regolari, ostinati o griglie ritmiche stabili su cui altri layer possono costruire complessità.
    \item Ritardando (\texttt{type: ritardando}): Simula un rallentamento progressivo.
\end{enumerate}
\begin{lstlisting}[language=Python]
elif model_type == 'ritardando':
    shape = model.get('shape', 2.0)
    final_progress = base_progress ** shape
\end{lstlisting}
    Applicando una funzione di potenza con esponente maggiore di 1, la curva di progressione si \textit{piega}, concentrando gli eventi all'inizio e diradandoli verso la fine. Il parametro \texttt{shape} controlla l'intensità del gesto: un valore più alto crea un ritardando più drammatico e pronunciato.

\begin{enumerate}
    \item Accelerando (\texttt{type: accelerando}): Speculare al ritardando, crea un aumento di tensione.
\end{enumerate}
\begin{lstlisting}[language=Python]
elif model_type == 'accelerando':
    shape = model.get('shape', 2.0)
    final_progress = 1 - (1 - base_progress) ** shape
\end{lstlisting}
    La formula inverte la progressione, applica la potenza e la inverte nuovamente. Il risultato è una curva che parte lentamente e accelera in modo esponenziale, ideale per creare archi di tensione, approcci a un climax o gesti di \textit{accumulazione} energetica.

\begin{enumerate}
    \item Stocastico (\texttt{type: stochastic}): Per texture organiche e imprevedibili.
\end{enumerate}
\begin{lstlisting}[language=Python]
elif model_type == 'stochastic':
    final_progress = np.sort(np.random.rand(num_events))
\end{lstlisting}
    Il metodo genera un set di istanti casuali e poi li ordina. Questo garantisce che, pur essendo irregolari, gli eventi mantengano una progressione temporale in avanti. È un modello efficace per rompere la rigidità della griglia metrica e imitare processi naturali o \textit{pointillistici}.
\subsubsection{Il Modello \texttt{breakpoint}: Curve Temporali su Misura}
Il modello più potente e flessibile è \texttt{breakpoint}. Permette al compositore di \textit{disegnare} una curva di distribuzione temporale definendo una serie di punti di controllo. Ogni segmento tra due punti può avere la propria curvatura, consentendo la creazione di profili complessi.

Consideriamo un esempio:
\begin{lstlisting}[language=Python]
timing_model:
  type: breakpoint
  points:
\begin{itemize}
    \item [0.0, 0.0]        # Inizio
    \item [0.3, 0.7, 0.5]   # Il 70% degli eventi avviene nel primo 30% del tempo (curva concava, ease-out)
    \item [1.0, 1.0, 3.0]   # Il restante 30% degli eventi si distribuisce nel 70% del tempo rimanente (curva convessa, ease-in)
\end{itemize}
\end{lstlisting}
Questo YAML descrive un gesto di \textit{esplosione e diradamento}: un'alta densità di eventi all'inizio, seguita da una lunga coda rarefatta.

L'implementazione gestisce questa complessità in modo modulare:
\begin{lstlisting}[language=Python]
\section{... (all'interno del loop sui segmenti) ...}
\section{1. Normalizza il tempo all'interno del segmento (da 0 a 1)}
time_in_segment = (segment_times - t_start) / (t_end - t_start)
\section{2. Applica la funzione di shaping (curva)}
shaped_time = time_in_segment ** shape
\section{3. Interpola linearmente tra i VALORI usando il tempo curvato}
interpolated_values = v_start + (v_end - v_start) * shaped_time
final_progress[segment_mask] = interpolated_values
\end{lstlisting}
La logica chiave è la normalizzazione del tempo *all'interno di ogni segmento*. Questo permette di applicare una curva di \texttt{shape} locale senza che questa influenzi gli altri segmenti, rendendo il sistema potente e intuitivo.  % Auto-generated: include sezione5.tex
% --- Contenuto LaTeX autogenerato da capitolo6.md (sezione 7) ---

\section{SINTASSI E SEMANTICA COMPOSITIVA}
In Gamma, la partitura tradizionale lascia il posto a un documento YAML, un formato che diventa un vero e proprio linguaggio per la composizione algoritmica.
La grammatica di questo linguaggio si fonda su una gerarchia chiara e intuitiva, che va dalla macro-struttura della composizione al dettaglio del singolo layer sonoro. Al livello più alto troviamo le sezioni, definite da una durata e da parametri strutturali come il \texttt{ratio\_temporale} o un inviluppo d'ampiezza globale. Ogni sezione funge da contenitore temporale e contestuale per i suoi layer, che rappresentano i veri flussi sonori. È a livello del layer che si definisce l'identità musicale: la densità degli eventi, la loro distribuzione nel tempo tramite un \texttt{timing\_model} e, soprattutto, le maschere di tendenza che ne governano i parametri.

Questa organizzazione non è casuale, ma guida il compositore a pensare in termini di struttura e tessitura. I parametri di sezione definiscono il \textit{contenitore}, mentre i parametri di layer definiscono il \textit{contenuto}. Il sistema, inoltre, è progettato per essere conciso. Grazie a un modello di eredità implicita, molti parametri assumono valori di default sensati, permettendo al compositore di concentrarsi solo sugli aspetti che desidera modificare. Una semplice maschera come dinamica: 'mf' viene automaticamente espansa dal sistema nella sua forma più strutturata (\texttt{\{value: 'mf'\}}), mantenendo il file di partitura pulito e leggibile.

All'interno di questa struttura, la sintassi di Gamma offre strumenti sofisticati per scolpire la forma nel tempo. Il parametro lifespan, ad esempio, permette di orchestrare entrate e uscite scaglionate dei layer, definendo la loro finestra di attività come una porzione relativa della durata della sezione. Un layer con \texttt{lifespan: [0.2, 0.8]} esisterà solo nella parte centrale della sua sezione, permettendo la creazione di forme ad arco e sovrapposizioni complesse che sarebbero macchinose da specificare in modo sequenziale.

La gestione dei confini temporali è ulteriormente raffinata da due meccanismi complementari: il \texttt{safety\_buffer} e il \texttt{leeway}. Il primo è una misura preventiva che accorcia leggermente la finestra di generazione per garantire che nessun evento \textit{sbordi} accidentalmente nella sezione successiva. Il secondo, il \texttt{leeway}, agisce come una valvola di sfogo, concedendo agli ultimi eventi di un layer un piccolo margine di tempo extra per concludersi in modo naturale, evitando troncamenti bruschi. Insieme, questi strumenti offrono un controllo rigoroso ma flessibile sui punti di transizione, un aspetto critico nell'assemblaggio di forme musicali estese.

Infine, la sintassi supporta un flusso di lavoro pratico e iterativo. Modalità speciali come \texttt{solo: true} permettono di isolare un singolo layer per la messa a punto, mentre il \texttt{veteranMode} ottimizza drasticamente i tempi di rendering riutilizzando i file audio dei layer che non sono stati modificati. Queste non sono funzionalità di sintesi, ma strumenti di orchestrazione che rendono il processo compositivo più agile e interattivo.

In definitiva, il linguaggio YAML in Gamma trascende la sua funzione di semplice formato dati. Esso diventa un medium compositivo.   % Auto-generated: include sezione6.tex
% --- Contenuto LaTeX autogenerato da capitolo7.md (sezione 8) ---

\section{GENERAZIONE E RENDERING}
La trasformazione da specifiche YAML a suono coinvolge un processo multi-fase che genera file CSD Csound, orchestra processi di rendering paralleli, e assembla i risultati in una composizione coerente. Questo capitolo esamina i meccanismi tecnici che rendono possibile questa trasformazione, con particolare attenzione all'ottimizzazione delle risorse computazionali e alla gestione della complessità.
\subsection{Generazione File CSD}
Il file CSD (Csound Unified File Format) combina orchestra e partitura in un singolo documento. In Gamma, la generazione di questi file avviene dinamicamente, adattando un template alle specifiche della composizione.
\subsubsection{Template Dinamico e Sistema di Macro}
Il metodo \texttt{generate\_csd()} utilizza un template che viene popolato con valori calcolati:

\begin{lstlisting}[language=Python]
def generate_csd(self, composition_name, events, csd_file_path, wav_file_path):
\section{Costruzione delle tabelle dei ritmi}
    rhythm_tables_str = ""
    for rhythm_tuple, table_ids in self.rhythm_table_map.items():
        ritmi_str = ' '.join(map(str, rhythm_tuple))
        posizioni = [i % r for i, r in enumerate(rhythm_tuple) if r > 0]
        posizioni_str = ' '.join(map(str, posizioni))
        rhythm_tables_str += f"f {table_ids['ritmi_tab_num']} 0 {len(rhythm_tuple)} -2 {ritmi_str}\n"
        rhythm_tables_str += f"f {table_ids['pos_tab_num']} 0 {len(posizioni)} -2 {posizioni_str}\n"
\end{lstlisting}

La generazione delle tabelle dei ritmi dimostra l'integrazione tra Python e Csound. Ogni pattern ritmico unico genera due tabelle: una per i valori dei ritmi stessi, l'altra per le posizioni calcolate. La formula \texttt{i \% r} per le posizioni crea una mappatura ciclica che determina quale elemento del pattern viene usato per la selezione delle frequenze.

Le macro vengono sostituite nel template:

\begin{lstlisting}[language=Python]
csd_content = template.format(
    wav_file_path=wav_file_path,
    includes_path=includes_path,
    envelope_tables=envelope_tables_str,
    rhythm_tables=rhythm_tables_str,
    score_lines=score_lines,
    durata_totale=last_event_time + 10,
    ottave_macro=OTTAVE_RANGE[1],
    registri_macro=REGISTRI_RANGE[1],
    intervalli_macro=INTERVALLI_PER_OTTAVA
)
\end{lstlisting}

L'aggiunta di 10 secondi alla durata totale (\texttt{last\_event\_time + 10}) fornisce un buffer per il decay degli ultimi eventi, prevenendo troncamenti bruschi.
\subsubsection{Inserimento Dinamico degli F-Statements}
Gli f-statements per gli inviluppi vengono generati dalle configurazioni caricate:

\begin{lstlisting}[language=Python]
envelope_tables_str = "; --- TABELLE DEGLI INVILUPPI (generate da tables.yaml) ---\n"
all_envelope_configs = {\textbf{self.event_envelopes_config, }self.section_envelopes_config}

for name, config in all_envelope_configs.items():
    params_str = ' '.join(map(str, config['parameters']))
    envelope_tables_str += f"; {name}\n"
    envelope_tables_str += f"f {config['number']} 0 {config['size']} {config['gen_routine']} {params_str}\n"
\end{lstlisting}

L'unione dei dizionari con \texttt{\{\textbackslash\{\}textbf\{dict1, \}dict2\}} combina inviluppi evento e sezione in un'unica iterazione. I commenti generati (\texttt{; \{name\}}) facilitano il debugging del file CSD risultante.
\subsubsection{Formattazione delle Score Lines}
La generazione delle linee di score richiede particolare attenzione alla formattazione e all'allineamento:

\begin{lstlisting}[language=Python]
score_lines += (f'i "Voce"\t{event_time:.4f}\t{p["durata_totale"]:.3f}\t'
                f'{p["ritmi_tab_num"]}\t{p["durata_armonica"]:.3f}\t\t{p["dynamic_index"]:.6f}\t'
                f'{p["ottava"]}\t\t{p["registro"]}\t\t\t'
                f'{p["ottava_arrivo"]}\t\t{p["registro_arrivo"]}\t\t'
                f'{p["pos_tab_num"]}\t{p["id_comp"]}\t\t{p["nonlinear_mode"]}'
                f'\t\t\t\t{p["senso_movimento"]}\t\t\t{p["ifn_attacco"]}\t\t\t'
                f'{p.get("section_env_table_num", 0)}')
\end{lstlisting}

La precisione decimale varia per parametro: 4 decimali per il tempo di attacco garantiscono precisione al decimo di millisecondo, mentre 6 decimali per l'indice dinamico permettono interpolazioni fluide. L'uso di \texttt{get()} con default per parametri opzionali previene errori mantenendo retrocompatibilità.
\subsection{Rendering Parallelo}
Il rendering rappresenta il collo di bottiglia computazionale del sistema. Gamma affronta questa sfida attraverso parallelizzazione intelligente che massimizza l'utilizzo delle risorse mantenendo la semplicità del modello di programmazione.
\subsubsection{Gestione dei Processi Csound}
Il lancio dei processi avviene in modo non bloccante:

\begin{lstlisting}[language=Python]
def run_csound_process(csd_path, process_name, log_dir):
    log_file_path = log_dir / f"csound_render_{process_name}.log"
    print(f"    - Avvio rendering per '{process_name}' (Log: {log_file_path.name})")

try:
        log_file = open(log_file_path, 'w')
        process = subprocess.Popen(
            ['csound', '--format=float', str(csd_path)], 
            stdout=log_file, 
            stderr=log_file
        )
        return (process, process_name, log_file)
    except Exception as e:
        print(f"    - ERRORE CRITICO nel lanciare Csound per {process_name}: {e}")
        return None
\end{lstlisting}

L'uso di \texttt{Popen} invece di \texttt{run} è cruciale: permette di lanciare multipli processi Csound senza attendere il completamento di ciascuno. Il flag \texttt{{-}{-}format=float} specifica il formato di output, evitando ambiguità. La cattura separata di stdout e stderr in file di log individuali facilita il debugging post-mortem.
\subsubsection{Strategia di Parallelizzazione}
La parallelizzazione avviene a livello di layer, non di evento:

\begin{lstlisting}[language=Python]
for job in render_jobs:
\section{Genera eventi (veloce, sequenziale)}
    layer_events, layer_onsets = composer._process_layer(...)

if layer_events:
\section{Genera CSD (veloce, sequenziale)}
        composer.generate_csd(job['name'], layer_events, job['csd_path'], job['wav_path'])
\section{Lancia rendering (lento, parallelo)}
        proc_data = run_csound_process(job['csd_path'], job['name'], dirs['logs'])
        if proc_data: 
            csound_procs.append(proc_data)
\end{lstlisting}

Questa granularità è ottimale perché:
\begin{itemize}
    \item I layer sono unità logiche indipendenti
    \item Hanno dimensioni comparabili (bilanciamento del carico)
    \item Il numero di layer tipicamente corrisponde al numero di core disponibili
\end{itemize}
\subsubsection{Sincronizzazione e Gestione Errori}
Dopo il lancio parallelo, il sistema deve attendere il completamento:

\begin{lstlisting}[language=Python]
if csound_procs:
    print("\n   Attendendo il completamento del rendering dei layer...")
    for process, name, log_file in csound_procs:
        process.wait()
        log_file.close()

if process.returncode != 0:
            print(f"   ✗ ERRORE: Rendering del layer '{name}' fallito!")
        else:
            print(f"   ✓ Rendering del layer '{name}' completato.")
\end{lstlisting}

Il metodo \texttt{wait()} blocca fino al completamento del processo. Il controllo del \texttt{returncode} permette di identificare fallimenti. La chiusura esplicita del file di log garantisce che tutti i messaggi siano scritti su disco.
\subsubsection{Ottimizzazione attraverso Caching}
Il sistema implementa due livelli di caching per ottimizzare i tempi di rendering:

\textbf{Cache delle tabelle ritmiche}: Pattern identici condividono tabelle, riducendo il tempo di inizializzazione Csound.

\textbf{Cache dei file WAV (Veteran Mode)}: Layer non modificati riutilizzano WAV esistenti:

\begin{lstlisting}[language=Python]
should_render = not veteran_mode_active or layer.get('veteranMode', False)
if should_render:
    section_needs_reassembly = True
    layer_render_jobs.append({...})
\end{lstlisting}

Questa ottimizzazione può ridurre i tempi di rendering del 90\% durante l'iterazione compositiva.
\subsection{Assemblaggio Multi-Livello}
L'assemblaggio segue la gerarchia compositiva dal basso verso l'alto, permettendo parallelizzazione dove possibile e garantendo coerenza strutturale.
\subsubsection{Generazione dei CSD di Assemblaggio}
I file CSD per l'assemblaggio utilizzano uno strumento orchestrator dedicato:

\begin{lstlisting}[language=Python]
def generate_assembler_csd(csd_path, output_wav_path, input_files_with_onsets, title="Assembler"):
    score_lines = ""
    for file_path, onset in input_files_with_onsets:
        score_lines += f'i "orchestrator" {onset:.4f} [60*8-{onset:.4f}] "{file_path}"\n'

template = f"""<CsoundSynthesizer>
<CsOptions>
-o "{output_wav_path}" -W -d -m0
</CsOptions>
<CsInstruments>
sr=96000
ksmps=32
nchnls=2
0dbfs=1

instr orchestrator
    S_file strget p4
    i_dur filelen S_file
    if i_dur > 0 then
        prints "Scheduling '%s' (dur: %.2fs) at time %.2fs\\n", S_file, i_dur, p2
        schedule "playFile", 0, i_dur, S_file
    else
        prints "WARNING: Could not play file '%s'.\\n", S_file
    endif
endin

instr playFile
    a_L, a_R diskin2 p4, 1
    outs a_L, a_R
endin
</CsInstruments>
<CsScore>
{score_lines}
e
</CsScore>
</CsoundSynthesizer>"""
\end{lstlisting}

L'orchestrator utilizza \texttt{filelen} per determinare la durata del file WAV e \texttt{schedule} per lanciare la riproduzione. La durata \texttt{[60*8{-}\{onset:.4f\}]} garantisce che lo strumento rimanga attivo abbastanza a lungo per qualsiasi file.
\subsubsection{Assemblaggio Layer in Sezioni}
Il primo livello di assemblaggio combina i layer di ogni sezione:

\begin{lstlisting}[language=Python]
if not veteran_mode_active or section_needs_reassembly:
    section_assembly_jobs.append({
        'name': section_name_base,
        'csd_path': dirs['sections_csd'] / f"{section_name_base}_assembler.csd",
        'output_wav': section_wav_path,
        'input_layers': layer_files_for_this_section
    })
\end{lstlisting}

L'assemblaggio avviene solo se necessario: se nessun layer della sezione è cambiato, il WAV esistente viene riutilizzato.
\subsubsection{Assemblaggio Finale}
L'assemblaggio finale unisce tutte le sezioni:

\begin{lstlisting}[language=Python]
unique_final_parts = {str(part['wav_path']): part 
                     for part in reversed(final_parts)}.values()

generate_assembler_csd(final_csd_path, final_wav_path, 
                      [(p['wav_path'], p['onset']) for p in unique_final_parts],
                      title="Final Composition Assembler")
\end{lstlisting}

La deduplicazione attraverso dizionario previene inclusioni multiple dello stesso file. La reversione prima della deduplicazione mantiene l'ultima occorrenza di ogni parte, utile quando si sovrascrivono sezioni durante lo sviluppo.
\subsubsection{Gestione dei Silenzi e File Mancanti}
Il sistema gestisce robustamente casi limite:

\begin{lstlisting}[language=Python]
if not layer_events:
    generate_silent_wav(job['wav_path'], scaled_duration)
\end{lstlisting}

La generazione di WAV silenziosi mantiene la consistenza strutturale: ogni layer produce sempre un file, semplificando la logica di assemblaggio.

Il sistema di generazione e rendering di Gamma dimostra come la complessità della produzione audio multi-canale possa essere gestita attraverso astrazione e parallelizzazione intelligente. La separazione tra generazione (veloce, sequenziale) e rendering (lento, parallelo) massimizza l'efficienza computazionale. L'assemblaggio gerarchico permette di lavorare a diversi livelli di granularità mantenendo la coerenza del risultato finale. Le ottimizzazioni attraverso caching e modalità speciali trasformano quello che potrebbe essere un processo tedioso in un workflow fluido e responsivo.  % Auto-generated: include sezione7.tex
% --- Contenuto LaTeX autogenerato da capitolo8.md (sezione 9) ---

\section{VISUALIZZAZIONE E ANALISI}
La visualizzazione in un sistema compositivo generativo non è un accessorio decorativo ma uno strumento epistemologico essenziale. Quando i processi algoritmici generano migliaia di eventi con parametri multidimensionali, la rappresentazione visiva diventa l'unico mezzo pratico per comprendere le strutture emergenti e verificare la corrispondenza tra intenzione compositiva e risultato generato.
\subsection{Sistema di Plotting Multi-Pagina}
La classe \texttt{CompositionDebugger} implementa un sistema di visualizzazione che bilancia completezza informativa con leggibilità, affrontando la sfida di rappresentare simultaneamente molteplici dimensioni parametriche su scale temporali estese.
\subsubsection{Architettura del Piano Roll Esteso}
Il metodo principale \texttt{plot\_piano\_roll()} genera visualizzazioni che vanno oltre il tradizionale piano roll MIDI, integrando informazioni su dinamiche, maschere di tendenza e struttura formale:

\begin{lstlisting}[language=Python]
def plot_piano_roll(self, events, all_onsets, composition_name, composition_structure, 
                   composer, onsets_by_section, title=None, partitura_mode=False, 
                   page_duration_s=60):

with PdfPages(plot_filename) as pdf:
        num_pages = int(np.ceil(total_duration / page_duration_s)) if partitura_mode else 1

for i in range(num_pages):
            fig, (ax_pitch, ax_dyn_linear, ax_dyn_prob) = plt.subplots(
                nrows=3, ncols=1, 
                figsize=(A3_LANDSCAPE_WIDTH_INCHES, A3_LANDSCAPE_HEIGHT_INCHES),
                sharex=True, gridspec_kw={'height_ratios': [3, 1, 1]}
            )
\end{lstlisting}

La scelta di utilizzare formato A3 landscape riflette la necessità di visualizzare dettagli fini su composizioni di lunga durata. La divisione in tre assi con proporzioni 3:1:1 privilegia la visualizzazione delle altezze mantenendo spazio sufficiente per le dinamiche.
\subsubsection{Rappresentazione degli Eventi Sonori}
Gli eventi sono visualizzati come rettangoli colorati dove multiple dimensioni sono mappate a proprietà visive:

\begin{lstlisting}[language=Python]
for item in plot_data_list:
    if item['start'] < page_end_time and (item['start'] + item['duration']) > page_start_time:
        ax_pitch.add_patch(
            plt.Rectangle(
                (item['start'], item['pitch'] - 0.04), 
                item['duration'], 
                0.08, 
                color=plt.cm.viridis(item['amp_norm']), 
                alpha=0.6, 
                zorder=3
            )
        )
\end{lstlisting}

La mappatura utilizza:
\begin{itemize}
    \item \textbf{Posizione X}: Tempo di inizio
    \item \textbf{Larghezza}: Durata
    \item \textbf{Posizione Y}: Altezza (ottava.registro)
    \item \textbf{Colore}: Dinamica (attraverso colormap viridis)
    \item \textbf{Trasparenza}: Permette visualizzazione di sovrapposizioni
\end{itemize}

L'altezza fissa dei rettangoli (0.08) bilancia visibilità e densità informativa.
\subsubsection{Visualizzazione delle Maschere di Tendenza}
Le maschere di tendenza sono rappresentate come aree ombreggiate che mostrano l'evoluzione dei parametri:

\begin{lstlisting}[language=Python]
def _plot_parameter_envelope(self, ax, times, data, color, label_key, label_text):
    if not data:
        return

label = label_text if label_key not in self._labels_added else ""
    ax.fill_between(times, data['lower'], data['upper'], 
                    color=color, alpha=0.2, zorder=1, label=label)
    self._labels_added.add(label_key)
\end{lstlisting}

L'uso di \texttt{fill\_between} crea un'area che rappresenta visivamente il range di valori possibili in ogni momento. La trasparenza (alpha=0.2) permette la sovrapposizione di multiple maschere mantenendo la leggibilità.
\subsubsection{Gestione delle Dinamiche: Lineare vs Probabilistica}
Il sistema distingue tra due modalità di visualizzazione delle dinamiche, riflettendo la differenza tra interpolazione deterministica e evoluzione probabilistica:

\begin{lstlisting}[language=Python]
def _plot_dynamics(self, ax_linear, ax_prob, times, dynamics_data, layer_color, layer_name):
    if dynamics_data['is_probabilistic']:
\section{Stackplot per probabilità}
        if dynamics_data['prob_weights']:
            weights_per_dynamic = np.array(dynamics_data['prob_weights']).T
            ax_prob.stackplot(times, weights_per_dynamic, 
                             labels=stackplot_labels, colors=stackplot_colors,
                             alpha=0.6, zorder=2)
    else:
\section{Linea per interpolazione}
        if dynamics_data['line_trend']:
            ax_linear.plot(valid_times, valid_dynamics, color=layer_color, 
                          linewidth=2.5, label=f"Dinamica: {layer_name}", 
                          zorder=3, alpha=0.8)
\end{lstlisting}

Lo stackplot mostra l'evoluzione delle probabilità per ogni dinamica, mentre il grafico lineare traccia l'interpolazione continua dell'indice dinamico. Questa doppia rappresentazione cattura la differenza fondamentale tra i due approcci compositivi.
\subsubsection{Modalità Partitura per Composizioni Lunghe}
Per composizioni che eccedono una singola pagina, il sistema genera PDF multi-pagina:

\begin{lstlisting}[language=Python]
if partitura_mode:
    page_start_time = i * page_duration_s
    page_end_time = (i + 1) * page_duration_s
    ax_pitch.set_xlim(page_start_time, page_end_time)

page_title = f"{title} (Pagina {i+1}/{num_pages})"
    fig.suptitle(page_title, fontsize=14)
\end{lstlisting}

Ogni pagina mantiene la stessa struttura visiva ma mostra una finestra temporale diversa. I marcatori di sezione e le etichette sono gestiti per evitare duplicazioni tra pagine.
\subsection{Cache e Aggiornamento Incrementale}
Il sistema di cache rappresenta una soluzione al problema dei lunghi tempi di calcolo per la visualizzazione di composizioni complesse.
\subsubsection{Serializzazione e Deserializzazione}
I dati di visualizzazione vengono salvati in formato JSON:

\begin{lstlisting}[language=Python]
def _sanitize_data_for_json(data):
    if isinstance(data, dict):
        return {k: _sanitize_data_for_json(v) for k, v in data.items()}
    elif isinstance(data, list):
        return [_sanitize_data_for_json(v) for v in data]
    elif isinstance(data, Path):
        return str(data)
    elif isinstance(data, np.ndarray):
        return data.tolist()
    elif isinstance(data, (np.float64, np.int64)):
        return float(data)

return data
\end{lstlisting}

La sanitizzazione gestisce tipi Python/NumPy non serializzabili nativamente in JSON. La conversione di Path objects in stringhe e di array NumPy in liste mantiene la portabilità dei dati.
\subsubsection{Meccanismo di Merge Selettivo}
L'aggiornamento incrementale identifica e sostituisce solo i dati modificati:

\begin{lstlisting}[language=Python]
if fresh_data.get('render_jobs_info'):
    fresh_job_keys = set(
        (job['section']['nome_sezione'], job['layer_idx']) 
        for job in fresh_jobs_info
    )
\section{Rimuovi vecchi dati per layer aggiornati}
    final_data['events'] = [
        e for e in final_data['events']
        if (e['params']['section_name'], e['params'].get('layer_idx_ref')) 
           not in fresh_job_keys
    ]
\end{lstlisting}

L'identificazione attraverso tupla (\texttt{section\_name}, \texttt{layer\_idx}) garantisce unicità. La rimozione dei vecchi dati prima dell'inserimento dei nuovi previene duplicazioni.
\subsubsection{Benefici del Sistema di Cache}
La cache trasforma il workflow compositivo in diversi modi:

\begin{enumerate}
    \item \textbf{Iterazione Rapida}: Modificare un singolo layer non richiede ricalcolo dell'intera visualizzazione
    \item \textbf{Persistenza tra Sessioni}: I dati sopravvivono al riavvio del sistema
    \item \textbf{Debugging Storico}: È possibile confrontare visualizzazioni di versioni diverse
\end{enumerate}
\subsubsection{Gestione della Coerenza}
Il sistema mantiene coerenza tra cache e stato attuale attraverso:

\begin{lstlisting}[language=Python]
\section{Allineamento della struttura onsets_by_section}
onsets_map = {}
for job in final_data.get('render_jobs_info', []):
    key = (job['section']['nome_sezione'], job['layer_idx'])
    onsets_map[key] = job.get('adjusted_onsets', [])

for section_yaml in composition_structure:
    section_data = {'section_name': section_yaml['nome_sezione'], 'layers': []}
    for layer_idx_yaml, layer_yaml in enumerate(layers_in_yaml):
        key = (section_yaml['nome_sezione'], layer_idx_yaml)
        onsets_for_this_layer = onsets_map.get(key, [])
        section_data['layers'].append({
            'layer_name': layer_yaml.get('nome_layer', f'Layer {layer_idx_yaml+1}'),
            'onsets': onsets_for_this_layer
        })
\end{lstlisting}

Questo allineamento garantisce che la struttura dei dati di visualizzazione corrisponda sempre alla struttura YAML corrente, anche quando layer sono stati aggiunti o rimossi.

Il sistema di visualizzazione e analisi di Gamma trasforma dati numerici astratti in rappresentazioni visive che rivelano la struttura e l'evoluzione della composizione. L'integrazione di multiple dimensioni parametriche in una singola visualizzazione, combinata con il sistema di cache incrementale, crea uno strumento che non solo documenta il risultato compositivo ma diventa parte integrante del processo creativo stesso, permettendo al compositore di \textit{vedere} la musica emergere dai processi generativi e di intervenire con precisione chirurgica dove necessario.  % Auto-generated: include sezione8.tex
% conclusione.tex

\section{Conclusione}
Questo è il testo della conclusione del documento.
  % Auto-generated: include conclusione.tex


% Aggiungi la bibliografia
%\newpage % ---- Inizia una nuova pagina prima della bibliografia
%\bibliographystyle{plain}
%\bibliography{bibliography}

\end{document}