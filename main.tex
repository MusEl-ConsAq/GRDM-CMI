\documentclass[a4paper,12pt]{article}
\usepackage{ME_AQ_temp}
\usepackage{tabularx}
\usepackage{listings}
\usepackage{xcolor} % Necessario per usare i colori

% Definisci i colori che userai per il codice
\definecolor{codegreen}{rgb}{0,0.6,0}
\definecolor{codegray}{rgb}{0.5,0.5,0.5}
\definecolor{codepurple}{rgb}{0.58,0,0.82}
\definecolor{backcolour}{rgb}{0.98,0.98,0.98} % Un grigio molto chiaro per lo sfondo

% Definisci uno stile personalizzato per tutti i tuoi listati
\lstdefinestyle{mystyle}{
    backgroundcolor=\color{backcolour},   % Sfondo del blocco di codice
    commentstyle=\color{codegreen},       % Colore per i commenti
    keywordstyle=\color{magenta},         % Colore per le parole chiave del linguaggio
    numberstyle=\tiny\color{codegray},    % Stile per i numeri di riga
    stringstyle=\color{codepurple},       % Colore per le stringhe
    basicstyle=\ttfamily\footnotesize,    % Font di base (monospaziato, piccolo)
    
    % --- OPZIONI FONDAMENTALI PER I MARGINI ---
    breakatwhitespace=false,              % Permette di spezzare le righe ovunque, non solo sugli spazi
    breaklines=true,                      % Abilita l'andare a capo automatico
    
    % --- Altre opzioni utili per la leggibilità ---
    captionpos=b,                         % Posizione della didascalia (b=bottom, t=top)
    keepspaces=true,                      % Mantiene gli spazi nel codice, fondamentale per l'indentazione
    numbers=left,                         % Posizione dei numeri di riga
    numbersep=5pt,                        % Distanza tra numeri di riga e codice
    showspaces=false,                     % Non mostrare gli spazi con un simbolo speciale
    showstringspaces=false,               % Non mostrare gli spazi nelle stringhe con un simbolo speciale
    showtabs=false,                       % Non mostrare i tab con un simbolo speciale
    tabsize=2,                            % Dimensione di un carattere di tabulazione
    frame=single,                         % Aggiunge un bordo attorno al blocco di codice
    framerule=0.5pt,                      % Spessore del bordo
    rulecolor=\color{black},              % Colore del bordo
    xleftmargin=1em,                      % Margine sinistro del blocco
    xrightmargin=1em                      % Margine destro del blocco
}

% Applica questo stile a tutti gli ambienti lstlisting del documento
\lstset{style=mystyle}
%% -------------------------------------- %%
%  - impostare il titolo della tesina in entrambe le righe 
% 
%% -------------------------------------- %%

\newcommand{\mycustomtitle}{Titolo della Tesina}
% Definisci un titolo personalizzato
\newcommand{\setmytitle}[1]{\renewcommand{\mycustomtitle}{#1}}

% Definizione del titolo e dell'autore
\title{Corsi Accademici di Musica Elettronica DCPL34 Conservatorio A. Casella, L'Aquila \\ \fontsize{14}{17}\bfseries\uppercase{Titolo della Tesina}}
\author{Nome Cognome \\ esame di \bfseries{Storia della Musica Elettroacustica} e \\ \bfseries{Analisi della Musica Elettroacustica}}
\date{xx/xx/xxxx}

% Sovrascrivi le impostazioni di hyperref per l'indice
\hypersetup{
    linkcolor=black, % Imposta il colore dei link dell'indice a nero
}

\begin{document}

% Pagina 1: Titolo e riassunto
\maketitle
\thispagestyle{empty}

\begin{center}
    \vspace{1cm}
    \textbf{\fontsize{12}{15}\selectfont{Sommario}}
\end{center}

Testo del sommario...

\newpage
% Genera l'indice
\tableofcontents  

% Pagina 2: Introduzione e resto del testo
\newpage

% --- File di inclusione generato automaticamente ---
\input{sections/introduzione.tex}  % Auto-generated: include introduzione.tex
% --- Contenuto LaTeX autogenerato da capitolo1.md (sezione 2) ---

\section{L'ORCHESTRA GAMMA}
L'orchestra Csound di Gamma rappresenta il cuore pulsante del sistema di sintesi, dove i parametri astratti generati dal motore Python si trasformano in eventi sonori concreti. Questo capitolo esplora in dettaglio l'architettura degli strumenti principali e il flusso di elaborazione che porta dal numero al suono.
\subsection{Lo Strumento Voce: Generatore di Comportamenti}
Lo strumento \texttt{Voce} costituisce il livello più alto della gerarchia di sintesi in Gamma. Non genera direttamente suoni, ma orchestra la creazione di sequenze di eventi sonori secondo logiche compositive complesse. La sua definizione inizia con una ricca parametrizzazione:

\begin{lstlisting}[language=C]
instr Voce
    ; -----------------------------------------------------------------------
    ; 1. INIZIALIZZAZIONE E ACQUISIZIONE PARAMETRI
    ; -----------------------------------------------------------------------
    i_CAttacco       = p2             ; Tempo di attacco del comportamento
    i_Durata         = p3             ; Durata complessiva
    i_RitmiTab       = p4             ; Tabella dei ritmi
    i_DurataArmonica = p5             ; Durata armonica di riferimento
    i_DynamicIndex   = p6         
    i_Ottava         = p7             
    i_Registro       = p8             
    i_ottava_arrivo = p9
    i_registro_arrivo = p10
    i_PosTab         = p11             ; Tabella delle posizioni
    i_IdComp         = p12            ; ID del comportamento
    i_NonlinearMode  = (p13 == 0 ? 3 : p13)
    i_SensoMovimento = (p14 == 0 ? 1 : p14) 
    i_ifnAttacco     = (p15 == 0 ? 10 : p15)
    i_ifn_section_env = p16 
    i_section_start_time = p17
    i_duration_leeway = p19
    i_section_duration = p18 + p19
    i_section_end = i_section_start_time + i_section_duration
    iSafetyBuffer = p20
\end{lstlisting}

Ogni parametro ha un significato musicale preciso:

\begin{itemize}
    \item \texttt{ i\_CAttacco}  e \texttt{i\_Durata}: definiscono la finestra temporale in cui il comportamento è attivo
    \item \texttt{i\_RitmiTab}: punta a una tabella contenente la sequenza di valori ritmici che determinano sia la temporalità che le frequenze degli eventi
    \item \texttt{i\_DurataArmonica}: il valore di riferimento per il calcolo delle durate reali degli eventi
    \item \texttt{i\_Ottava} e \texttt{i\_Registro}: coordinate nello spazio delle altezze di partenza
    \item \texttt{i\_ottava\_arrivo} e \texttt{i\_registro\_arrivo}: destinazione per eventuali glissandi
    \item \texttt{i\_NonlinearMode}: seleziona l'algoritmo di generazione per nuovi ritmi
\end{itemize}
\subsubsection{Il Loop Generativo Principale}
Il cuore dello strumento Voce è un loop while che genera eventi fino al raggiungimento della durata specificata:

\begin{lstlisting}[language=C]
i_EventIdx = 0
i_whileTime = 0

while i_whileTime < i_Durata do
    ; -------- 3.1 GESTIONE RITMI --------
    if i_EventIdx < i_LenRitmiTab then
        i_RitmoCorrente tab_i i_EventIdx, i_TempRitmiTab
        if i_RitmoCorrente == 0 then
            goto generateNewRhythm
        endif
        i_Vecchio_Ritmo = (i_EventIdx == 0) ? 1 : tab_i(i_EventIdx - 1, i_TempRitmiTab)
    else
        generateNewRhythm:
        i_Vecchio_Ritmo tab_i i_EventIdx - 1, i_TempRitmiTab
        i_RitmoCorrente NonlinearFunc i_Vecchio_Ritmo, i_NonlinearMode
        tabw_i i_RitmoCorrente, i_EventIdx, i_TempRitmiTab
    endif
\end{lstlisting}

Questo codice implementa una logica sofisticata: inizialmente legge i ritmi dalla tabella fornita, ma quando questa si esaurisce, genera nuovi valori usando l'opcode \texttt{NonlinearFunc}, creando potenzialmente sequenze infinite che evolvono secondo regole caotiche o deterministiche.
\subsubsection{Calcolo Temporale degli Eventi}
Il timing di ogni evento dipende dal ritmo precedente secondo la formula:

\begin{lstlisting}[language=C]
if i_EventIdx == 0 then
    i_EventAttack = i_CAttacco
else
    i_RitmoNormalizzato = 1 / i_Vecchio_Ritmo
    i_PreviousAttack tab_i gi_Index - 1, gi_eve_attacco
    i_EventAttack = i_DurataArmonica * i_RitmoNormalizzato + i_PreviousAttack
endif
\end{lstlisting}

Questa relazione inversamente proporzionale significa che valori ritmici più alti producono eventi più ravvicinati, creando accelerazioni, mentre valori bassi generano rarefazioni temporali.
\subsubsection{Gestione della Tabella Ritmi Temporanea}
Una caratteristica importante è la creazione di una tabella temporanea estesa per i ritmi:

\begin{lstlisting}[language=C]
i_LenRitmiTab = ftlen(i_RitmiTab)
i_TempRitmiTab ftgen 0, 0, i_LenRitmiTab + 10000, -2, 0

; Copia i ritmi iniziali nella tabella temporanea
i_IndexCopy = 0
while i_IndexCopy < i_LenRitmiTab do
    i_ValRitmo tab_i i_IndexCopy, i_RitmiTab
    tabw_i i_ValRitmo, i_IndexCopy, i_TempRitmiTab
    i_IndexCopy += 1
od
\end{lstlisting}

Questo approccio permette di estendere dinamicamente la sequenza ritmica oltre i valori iniziali senza modificare la tabella originale, mantenendo la purezza dei dati di input mentre si esplora lo spazio generativo.
\subsubsection{Sistema di Scheduling degli Eventi}
La creazione effettiva degli eventi sonori avviene attraverso la chiamata a \texttt{schedule}:

\begin{lstlisting}[language=C]
schedule "eventoSonoro", i_EventAttack - p2, i_EventDuration, i_DynamicIndex, i_Freq1, i_Pos,
        i_RitmoCorrente, i_Freq2, i_ifnAttacco, gi_Index, i_IdComp, i_SensoMovimento, 
        i_ifn_section_env, i_section_start_time, i_section_duration
\end{lstlisting}

Notare come \texttt{i\_EventAttack {-} p2} converta il tempo assoluto in tempo relativo all'inizio dello strumento Voce, mantenendo la coerenza temporale nella gerarchia degli strumenti.
\subsection{EventoSonoro: Dal Parametro al Suono}
Lo strumento \texttt{eventoSonoro} è responsabile della generazione effettiva del suono. Riceve i parametri calcolati da Voce e li trasforma in segnale audio attraverso sintesi e processamento.
\subsubsection{Inizializzazione e Validazione Parametri}
\begin{lstlisting}[language=C]
instr eventoSonoro
   i_DynamicIndex = p4
   i_debug=gi_debug
   ifreq1 = limit(p5, 20, sr/2)   
   iwhichZero = abs(p6)    
   iHR = max(1, abs(p7))

iPeriod = $M_PI * 2 / iHR

iradi = (iwhichZero > 0 ? (iwhichZero - 1) * iPeriod : 0)
   ifreq2 = limit(p8, 20, sr/2)

ifn_shape = (p9 == 0 ? 2 : p9)
\end{lstlisting}

I parametri vengono immediatamente validati e limitati per evitare valori che potrebbero causare problemi:
\begin{itemize}
    \item Le frequenze sono limitate tra 20 Hz e la frequenza di Nyquist
    \item \texttt{iHR} (harmonic ratio) è forzato ad essere almeno 1
    \item \texttt{ifn\_shape} ha un default alla tabella 2 se non specificato
\end{itemize}
\subsubsection{Sistema di Compensazione Isofonica dell'Ampiezza}
Una delle caratteristiche più sofisticate di Gamma è l'implementazione di un sistema di calibrazione dell'ampiezza basato sulle curve isofoniche ISO 226:2003. Questo garantisce che la percezione di loudness rimanga costante indipendentemente dalla frequenza.

Il calcolo dell'ampiezza avviene attraverso una catena di UDO specializzati:

\begin{lstlisting}[language=C]
kamp GetIsoAmp_k i_DynamicIndex, ifreq1, ifreq2
\end{lstlisting}

Questo UDO k-rate gestisce glissandi compensando dinamicamente l'ampiezza. Vediamo l'implementazione:

\begin{lstlisting}[language=C]
opcode GetIsoAmp_k, k, iii
    iDynamicIndex, iFreqStart, iFreqEnd xin

; 1. Calcola le ampiezze isofoniche per i punti di inizio e fine a i-rate
    iAmpStart       GetIsoAmp       iFreqStart, iDynamicIndex
    iAmpEnd         GetIsoAmp       iFreqEnd, iDynamicIndex

if iAmpStart > iAmpEnd then
        kf expseg  1, p3, 0.0001
        kFinalAmp = (kf * (iAmpStart-iAmpEnd))+iAmpEnd
    elseif iAmpStart < iAmpEnd then
        kf expseg  0.0001, p3, 1
        kFinalAmp = (kf * (iAmpEnd-iAmpStart))+iAmpStart
    else
        kFinalAmp = iAmpStart
    endif
    xout kFinalAmp
endop
\end{lstlisting}

Il cuore del sistema è l'UDO \texttt{GetIsoAmp}:

\begin{lstlisting}[language=C]
opcode GetIsoAmp, i, ii
    iFrequency, iDynamicIndex xin

iSafeFrequency = limit(iFrequency, 20, 12500)

; 1. Recupera i parametri di base per la dinamica data
    iPhonLevel, iDbfsRef1kHz GetDynamicParams iDynamicIndex

; 2. Calcola il dB SPL target per la frequenza e il livello phon dati
    iDbSplTarget    PhonToSpl_i     iPhonLevel, iSafeFrequency

; 3. Il dB SPL di riferimento a 1kHz è per definizione uguale al livello Phon
    iDbSplRef1kHz   =               iPhonLevel

; 4. Calcola l'offset di compensazione
    iFrequencyOffset = iDbSplTarget - iDbSplRef1kHz

; 5. Applica l'offset al livello dBFS di riferimento
    iFinalDbfs      = iDbfsRef1kHz + iFrequencyOffset

; 6. Converti il dBFS finale in ampiezza lineare
    iFinalAmp       = ampdbfs(iFinalDbfs)

xout iFinalAmp
endop
\end{lstlisting}

La conversione da Phon a SPL utilizza le curve ISO interpolate:

\begin{lstlisting}[language=C]
opcode PhonToSpl_i, i, ii
    iphon, ifreq    xin

; Interpolazione lineare dalle tabelle ISO
    iaf             Interp  ifreq, giIsoFreqs, giAf
    ilu             Interp  ifreq, giIsoFreqs, giLu
    itf             Interp  ifreq, giIsoFreqs, giTf

; Formula ISO 226:2003
    iterm1          =       4.47e-3 * (pow(10, 0.025 * iphon) - 1.15)
    iterm2_exp      =       (itf + ilu) / 10.0 - 9
    iterm2          =       pow(0.4 * pow(10, iterm2_exp), iaf)
    iaf_value       =       iterm1 + iterm2

if iaf_value <= 0 then
        ispl        =       itf + (iphon / 40.0) * 20
    else
        ispl        =       (10.0 / iaf) * log10(iaf_value) - ilu + 94.0
    endif

if abs(ifreq - 1000) < 0.1 then
        ispl = iphon
    endif

xout            ispl
endop
\end{lstlisting}
\subsubsection{Sistema di Spazializzazione Mid-Side e Armoniche Spaziali}
La spazializzazione in Gamma va oltre il semplice panning stereofonico, implementando un sistema basato su \textit{armoniche spaziali} che deriva dalla teoria delle armoniche ritmiche. Il concetto chiave è che i valori ritmici non solo organizzano il tempo e selezionano le frequenze, ma definiscono anche il movimento nello spazio stereofonico.

Vediamo come si sviluppa questo sistema partendo dai parametri di base:

\begin{lstlisting}[language=C]
; Parametri di base per la spazializzazione
iwhichZero = abs(p6)    ; quale "zero" della funzione trigonometrica usare
iHR = max(1, abs(p7))   ; Harmonic Ratio - il numero di "spicchi" della circonferenza

; Calcolo del periodo e della posizione iniziale
iPeriod = $M_PI * 2 / iHR
iradi = (iwhichZero > 0 ? (iwhichZero - 1) * iPeriod : 0)
\end{lstlisting}

Il parametro \texttt{iHR} (Harmonic Ratio) determina in quanti \textit{spicchi} viene suddivisa la circonferenza. Ad esempio:
\begin{itemize}
    \item \texttt{iHR = 1}: un solo periodo, movimento completo 0-360°
    \item \texttt{iHR = 4}: quattro periodi, la circonferenza è divisa in quadranti
    \item \texttt{iHR = 7}: sette spicchi, creando una suddivisione asimmetrica
\end{itemize}

Il parametro \texttt{iwhichZero} determina da quale zero della funzione trigonometrica iniziare il movimento:

\begin{lstlisting}[language=C]
; Evoluzione temporale della posizione angolare
kndx_local line 0, p3, 1
ktab tab kndx_local, ifn_shape, 1
krad = iradi + (ktab * iPeriod * i_senso)
\end{lstlisting}

Qui \texttt{krad} evolve nel tempo secondo l'inviluppo specificato da \texttt{ifn\_shape}, modulato dal senso di movimento (\texttt{i\_senso} = 1 o -1 per movimento orario/antiorario).

La generazione dell'inviluppo locale usa una modifica della funzione seno quando \texttt{ifn\_shape == 2}:

\begin{lstlisting}[language=C]
if ifn_shape == 2 then
    kEnv_local = abs(sin(krad * iHR / 2))
else
    kEnv_local tab kndx_local, ifn_shape, 1
endif
\end{lstlisting}

La formula \texttt{abs(sin(krad * iHR / 2))} genera curve polari modificate. Questa trasformazione:
\begin{itemize}
    \item Prende il valore assoluto, creando lobi sempre positivi
    \item Moltiplica per \texttt{iHR / 2}, dimezzando il numero di lobi rispetto agli spicchi spaziali
    \item Crea una correlazione diretta tra movimento spaziale e ampiezza
\end{itemize}

Per comprendere meglio, consideriamo il codice Python fornito che visualizza queste funzioni:

\begin{lstlisting}[language=Python]
def genera_e_plotta_polare_sine(self):
    theta = np.linspace(0, 2 * np.pi, 500)
    num_funzioni = 10
\section{Base delle funzioni sinusoidali}
    r_base = [np.abs(np.sin(theta * i / 2)) for i in range(1, num_funzioni + 1)]
\end{lstlisting}

Questo mostra come per \texttt{i} crescenti si ottengono curve polari con sempre più lobi, che in Csound diventano pattern di inviluppo sempre più complessi.

La conversione finale da coordinate polari a stereo avviene con:

\begin{lstlisting}[language=C]
; Calcolo delle componenti Mid-Side
kMid = cos(krad)
kSide = sin(krad)

; Applicazione dell'inviluppo al segnale
aMid = kMid * asigEnv 
aSide = kSide * asigEnv

; Conversione a Left-Right con matrice di rotazione
aL = (aMid + aSide) / $SQRT2
aR = (aMid - aSide) / $SQRT2
\end{lstlisting}

Questa matrice di rotazione:
\begin{itemize}
    \item Mantiene potenza costante durante il movimento
    \item Crea un campo stereofonico coerente
    \item Permette movimenti fluidi nello spazio
\end{itemize}
\subsubsection{Gestione degli Inviluppi Multipli}
Il sistema gestisce due livelli di inviluppo che interagiscono moltiplicativamente:

\begin{lstlisting}[language=C]
; Inviluppo locale dell'evento (derivato dalle armoniche spaziali)
asigLocalEnv = asig * kEnv_local

; Inviluppo di sezione (se presente)
kEnv_section = 1
if i_ifn_section_env > 20 && i_section_duration > 0 then
    k_time_absolute times      
    k_time_since_section_start = k_time_absolute - i_section_start_time
    kndx_section = limit(k_time_since_section_start / i_section_duration, 0, 1)
    kEnv_section tablei kndx_section, i_ifn_section_env, 1
endif

; Combinazione degli inviluppi
asigEnvPre = asigLocalEnv * kEnv_section
asigEnv dcblock asigEnvPre
\end{lstlisting}

L'inviluppo di sezione permette modulazioni globali su tutti gli eventi di una sezione, mentre l'inviluppo locale (potenzialmente derivato dalle armoniche spaziali) definisce la forma del singolo evento.
\subsection{Il Sistema di Intonazione Pitagorica}
Il sistema di altezze in Gamma si basa su una implementazione personalizzata dell'intonazione pitagorica, gestita dall'opcode \texttt{GenPythagFreqs}:

\begin{lstlisting}[language=C]
opcode GenPythagFreqs, i, iiii
  iFund, iNumIntervals, iNumOctaves, iTblNum xin
  iTotalLen = iNumIntervals * iNumOctaves
  iFreqs[] init iTotalLen

iOctave = 0
  iBaseIndex = 0

while iOctave < iNumOctaves do
    iFifth = 3/2
    iFreqs[iBaseIndex] = iFund * (2^iOctave)

; Genera la serie di quinte per questa ottava
    indx = 1
    iLastRatio = 1
    while (indx < iNumIntervals) do
      iRatio = iLastRatio * iFifth
      ; Riduci all'ottava di riferimento
      while (iRatio >= 2) do
        iRatio = iRatio / 2
      od
      iFreqs[iBaseIndex + indx] = iFund * iRatio * (2^iOctave)
      iLastRatio = iRatio
      indx += 1
    od
\end{lstlisting}
\subsubsection{Costruzione della Tabella Frequenze}
Il sistema genera una tabella bidimensionale concettuale dove:
\begin{itemize}
    \item Ogni ottava contiene \texttt{iNumIntervals} frequenze (200 nel nostro caso)
    \item Le frequenze sono generate attraverso iterazioni della quinta perfetta (3/2)
    \item Ogni quinta che supera l'ottava viene riportata all'interno tramite divisione per 2
\end{itemize}
\subsubsection{Ordinamento e Memorizzazione}
Dopo la generazione, le frequenze vengono ordinate all'interno di ogni ottava:

\begin{lstlisting}[language=C]
; Ordina le frequenze per questa ottava
indx = iBaseIndex
while (indx < (iBaseIndex + iNumIntervals - 1)) do
  indx2 = indx + 1
  while (indx2 < (iBaseIndex + iNumIntervals)) do
    if (iFreqs[indx2] < iFreqs[indx]) then
      iTemp = iFreqs[indx]
      iFreqs[indx] = iFreqs[indx2]
      iFreqs[indx2] = iTemp
    endif
    indx2 += 1
  od
  indx += 1
od
\end{lstlisting}

Questo bubble sort garantisce che le frequenze siano accessibili in ordine crescente all'interno di ogni ottava.
\subsubsection{Mappatura Ottava-Registro-Ritmo}
L'accesso alle frequenze avviene attraverso la funzione \texttt{calcFrequenza}:

\begin{lstlisting}[language=C]
opcode calcFrequenza, i, iii
    i_Ottava, i_Registro, i_RitmoCorrente xin

; Calculate octave register
    i_Indice_Ottava = int(i_Ottava * $INTERVALLI)
    ; Calculate interval offset within the octave
    i_OffsetIntervallo = i_Indice_Ottava + int(((i_Registro * $INTERVALLI) / $REGISTRI))

; Get the frequency from the table using the calculated offset
    i_Freq table max(1, i_OffsetIntervallo + i_RitmoCorrente), gi_Intonazione
    ifreq = min(i_Freq, sr/2-1)
    xout ifreq
endop
\end{lstlisting}

La formula di indicizzazione \texttt{i\_OffsetIntervallo + i\_RitmoCorrente} crea una relazione diretta tra il valore ritmico e l'altezza selezionata. Questo significa che:

\begin{itemize}
    \item Ritmi identici in registri diversi producono intervalli correlati
    \item La sequenza ritmica diventa una sequenza melodica
    \item Valori ritmici alti tendono verso frequenze più acute all'interno del registro
\end{itemize}
\subsubsection{Implicazioni Compositive}
Questa architettura crea una profonda interconnessione tra dimensione temporale, frequenziale e spaziale. Un pattern ritmico [3, 5, 8, 13] non solo definisce:
\begin{itemize}
    \item Le durate relative degli eventi (durataArmonica/3, durataArmonica/5, etc.)
    \item Le altezze selezionate dalla tabella pitagorica
    \item Il numero di suddivisioni spaziali e il pattern di movimento stereofonico
    \item La forma dell'inviluppo di ampiezza quando si usano le armoniche spaziali
\end{itemize}

L'uso dell'intonazione pitagorica invece del temperamento equabile aggiunge ulteriore ricchezza armonica: le quinte sono pure (rapporto 3:2), ma questo genera comma pitagorici e intervalli microtonali che colorano il risultato sonoro con battimenti e risonanze particolari.

L'orchestra Gamma dimostra come un'architettura ben progettata possa creare connessioni profonde tra parametri apparentemente indipendenti, trasformando relazioni numeriche astratte in strutture musicali percettivamente significative. La gerarchia Voce → eventoSonoro, supportata dal sistema di intonazione pitagorica, dalle sofisticate tecniche di compensazione isofonica e dal sistema di armoniche spaziali, fornisce al compositore uno strumento di straordinaria flessibilità espressiva, capace di generare tessiture sonore complesse da specifiche relativamente semplici.
\subsection{NonlinearFunc: Il Generatore di Ritmi Caotici}
L'opcode \texttt{NonlinearFunc} rappresenta uno degli elementi più innovativi di Gamma, fornendo un sistema sofisticato per la generazione di sequenze ritmiche che evolvono nel tempo secondo principi deterministici, periodici o caotici. Questo UDO (User Defined Opcode) estende le possibilità compositive oltre i pattern ritmici predefiniti, permettendo l'esplorazione di territori ritmici emergenti.
\subsubsection{Struttura e Parametri dell'Opcode}
\begin{lstlisting}[language=C]
opcode NonlinearFunc, i, ippo
  iX, iMode, iMinVal, iMaxVal xin

; Valori di default per min/max se non specificati
  iMinVal = (iMinVal == 0) ? 1 : iMinVal
  iMaxVal = (iMaxVal == 0) ? 35 : iMaxVal

; Assicurati che iX sia entro limiti sensati
  iX = limit(iX, 1, 100)

iPI = 4 * taninv(1.0)  
  iTemp = 0
\end{lstlisting}

L'opcode accetta quattro parametri:
\begin{itemize}
    \item \texttt{iX}: Il valore di input, tipicamente il ritmo precedente nella sequenza
    \item \texttt{iMode}: Selettore della modalità operativa (0-3)
    \item \texttt{iMinVal}: Valore minimo del range di output (default: 1)
    \item \texttt{iMaxVal}: Valore massimo del range di output (default: 35)
\end{itemize}

La prima operazione importante è la normalizzazione e limitazione dei valori di input per garantire stabilità numerica. Il valore di iX viene limitato tra 1 e 100 per evitare overflow o comportamenti indefiniti nelle funzioni matematiche successive.
\subsubsection{Modalità 0: Convergente}
\begin{lstlisting}[language=C]
if iMode == 0 then
    ; --- MODALITÀ 0: CONVERGENTE ---
    iR = 2.8
    iTemp = iR * iX * (1 - iX/40)
\end{lstlisting}

Questa modalità implementa una variante della mappa logistica con comportamento convergente. Il parametro \texttt{iR = 2.8} è scelto specificamente per rimanere nella regione stabile del diagramma di biforcazione della mappa logistica, dove il sistema converge verso un punto fisso.

La formula \texttt{iR * iX * (1 {-} iX/40)} differisce dalla classica mappa logistica \texttt{r * x * (1 {-} x)} per il fattore di scala 40. Questo adattamento:
\begin{itemize}
    \item Permette di lavorare con valori di input nell'intervallo 1-100 invece di 0-1
    \item Rallenta la convergenza, rendendo l'evoluzione ritmica più graduale
    \item Crea una traiettoria prevedibile verso un valore stabile
\end{itemize}

Matematicamente, per \texttt{iR = 2.8}, il sistema convergerà verso il punto fisso:
\begin{lstlisting}
x* = 40 * (1 - 1/iR) ≈ 25.71
\end{lstlisting}

Questo significa che sequenze ritmiche in modalità convergente tenderanno gradualmente verso valori intorno a 26, creando un effetto di stabilizzazione ritmica.
\subsubsection{Modalità 1: Periodica}
\begin{lstlisting}[language=C]
elseif iMode == 1 then
    ; --- MODALITÀ 1: PERIODICA ---
    iP1 = sin(iX * iPI/18)
    iP2 = cos(iX * iPI/10)
    iTemp = abs(iP1 * iP2) * 20 + 10
\end{lstlisting}

La modalità periodica utilizza l'interferenza di due funzioni trigonometriche con periodi incommensurabili per generare pattern complessi ma deterministici.

L'analisi matematica rivela:
\begin{itemize}
    \item \texttt{sin(iX * \}\}\}\$\textbackslash\{\}pi\$\textbackslash\{\}texttt\{\{/18)}: periodo di 36 unità
    \item \texttt{cos(iX * \}\}\}\$\textbackslash\{\}pi\$\textbackslash\{\}texttt\{\{/10)}: periodo di 20 unità
    \item Il minimo comune multiplo è 180, creando un super-periodo
\end{itemize}

Il prodotto \texttt{iP1 * iP2} genera un'interferenza costruttiva e distruttiva tra le due onde:
\begin{itemize}
    \item Quando entrambe le funzioni sono vicine ai loro massimi/minimi, il prodotto è grande
    \item Quando una è vicina a zero, il prodotto si annulla
    \item Il valore assoluto garantisce output positivi
\end{itemize}

La trasformazione finale \texttt{abs(iP1 * iP2) * 20 + 10}:
\begin{itemize}
    \item Scala il range da [0, 1] a [0, 20]
    \item Aggiunge un offset di 10, risultando in valori tra 10 e 30
    \item Garantisce che i ritmi generati rimangano in un range musicalmente utile
\end{itemize}

Questa modalità produce sequenze che si ripetono dopo 180 iterazioni ma con una struttura interna ricca di variazioni locali.
\subsubsection{Modalità 2: Caotica Deterministica}
\begin{lstlisting}[language=C]
elseif iMode == 2 then
    ; --- MODALITÀ 2: CAOTICA DETERMINISTICA ---
    iR = 3.99
    iNormX = (iX % 100) / 100
    iNormX = limit(iNormX, 0.01, 0.99)
    iLogistic = iR * iNormX * (1 - iNormX)
    iNoise = random:i(-0.05, 0.05)
    iLogistic = limit(iLogistic + iNoise, 0, 1)
    iRange = iMaxVal - iMinVal + 1
    iTemp = iMinVal + (iLogistic * iRange)
\end{lstlisting}

Questa modalità implementa la mappa logistica nella sua regione caotica con l'aggiunta di una piccola perturbazione stocastica.

Il parametro \texttt{iR = 3.99} posiziona il sistema al limite del caos:
\begin{itemize}
    \item Per r > 3.57, la mappa logistica entra nel regime caotico
    \item A r = 3.99, siamo nella regione di caos sviluppato
    \item Piccole variazioni nell'input producono grandi divergenze nell'output
\end{itemize}

Il processo di normalizzazione \texttt{(iX \% 100) / 100}:
\begin{itemize}
    \item Utilizza l'operatore modulo per mantenere i valori ciclici
    \item Normalizza nell'intervallo [0, 1] richiesto dalla mappa logistica
    \item Il limite \texttt{[0.01, 0.99]} evita i punti fissi instabili a 0 e 1
\end{itemize}

L'aggiunta di rumore \texttt{random:i({-}0.05, 0.05)}:
\begin{itemize}
    \item Introduce una componente stocastica del 5\%
    \item Previene cicli perfetti che potrebbero emergere anche nel caos deterministico
    \item Simula le imperfezioni del mondo reale
\end{itemize}
\subsubsection{Modalità 3: Caos Vero (Default)}
\begin{lstlisting}[language=C]
else
    ; --- MODALITÀ 3: CAOS VERO (DEFAULT) ---
    ; 1. Componente deterministica (60%)
    iSeed1 = (iX * 1.3) % 10
    iSeed2 = (iX * 0.7) % 10
    iSeed3 = (iX * 2.5) % 10
    iNonlinear1 = abs(sin(iSeed1 * iPI/5 + iSeed2))
    iNonlinear2 = abs(cos(iSeed2 * iPI/3 + iSeed3))
    iNonlinear3 = abs(tan(iSeed3 * iPI/7 + iSeed1) % 1)
    iDeterministic = (iNonlinear1 + iNonlinear2 + iNonlinear3) / 3
\end{lstlisting}

La modalità \textit{Caos Vero} rappresenta l'approccio più sofisticato, combinando molteplici generatori non lineari con componenti stocastiche.

La generazione dei seed utilizza moltiplicatori irrazionali approssimati:
\begin{itemize}
    \item 1.3 $\approx$ √1.69 
    \item 0.7 $\approx$ 1/√2
    \item 2.5 $\approx$ √6.25
\end{itemize}

Questi valori garantiscono che i tre seed evolvano a velocità diverse e incommensurabili, massimizzando la complessità dell'output.

Le tre funzioni non lineari utilizzano:
\begin{itemize}
    \item \texttt{sin} con accoppiamento additivo: sensibile alle fasi relative
    \item \texttt{cos} con accoppiamento additivo: sfasato di $\pi$/2 rispetto a sin
    \item \texttt{tan} con modulo: introduce discontinuità controllate
\end{itemize}

\begin{lstlisting}[language=C]
    ; 2. Componente casuale (40%)
    iRandom = random:i(0, 1)

; 3. Combina le componenti
    iMixRatio = 0.6
    iCombined = (iDeterministic * iMixRatio) + (iRandom * (1 - iMixRatio))
\end{lstlisting}

Il bilanciamento 60/40 tra deterministico e stocastico è calibrato per:
\begin{itemize}
    \item Mantenere una struttura riconoscibile (componente deterministica)
    \item Introdurre sufficiente imprevedibilità (componente random)
    \item Evitare sia la monotonia che il rumore bianco
\end{itemize}

\begin{lstlisting}[language=C]
    ; 4. Perturbazione periodica
    iPerturbation = 0
    if (iX % 7 == 0) then 
      iPerturbation = random:i(-0.3, 0.3)
    endif

; 5. Mappa al range finale
    iRange = iMaxVal - iMinVal + 1
    iTemp = iMinVal + (iCombined * iRange) + (iPerturbation * iRange)
\end{lstlisting}

La perturbazione periodica ogni 7 iterazioni:
\begin{itemize}
    \item Introduce eventi rari ma significativi
    \item Il numero 7 (primo) evita risonanze con altri periodi nel sistema
    \item L'ampiezza ±30\% può causare salti drammatici nel ritmo
\end{itemize}
\subsubsection{Integrazione con il Sistema Gamma}
Nel contesto dello strumento Voce, NonlinearFunc viene chiamato quando la tabella dei ritmi predefiniti si esaurisce:

\begin{lstlisting}[language=C]
i_RitmoCorrente NonlinearFunc i_Vecchio_Ritmo, i_NonlinearMode
\end{lstlisting}

Questo crea una transizione fluida da:
\begin{enumerate}
    \item \textbf{Fase deterministica}: Ritmi composti e memorizzati in tabella
    \item \textbf{Fase generativa}: Ritmi creati algoritmicamente
\end{enumerate}
L'output di NonlinearFunc influenza direttamente:
\begin{itemize}
    \item \textbf{Temporalità}: Attraverso la formula \texttt{i\_DurataArmonica / i\_RitmoCorrente}
    \item \textbf{Altezza}: Il ritmo viene usato come indice nella tabella delle frequenze
    \item \textbf{Spazializzazione}: Determina il parametro iHR per le armoniche spaziali
\end{itemize}
\subsubsection{Implicazioni Compositive e Estetiche}
L'uso di NonlinearFunc permette di esplorare diverse estetiche ritmiche:

\begin{itemize}
    \item \textbf{Modalità 0 (Convergente)}: Crea un senso di \textit{arrivo} o \textit{risoluzione} ritmica, utile per conclusioni o punti di stasi
    \item \textbf{Modalità 1 (Periodica)}: Genera groove complessi ma ripetitivi, ideale per sezioni di sviluppo
    \item \textbf{Modalità 2 (Caotica Deterministica)}: Produce variazioni continue senza ripetizioni, perfetta per tessiture in evoluzione
    \item \textbf{Modalità 3 (Caos Vero)}: Bilancia imprevedibilità e coerenza, creando interesse sostenuto
\end{itemize}

La possibilità di cambiare modalità durante la composizione (attraverso il parametro YAML \texttt{nonlinear\_mode}) permette di modulare il grado di prevedibilità/caos nel flusso ritmico, creando archi formali che vanno dall'ordine al disordine e viceversa.

L'implementazione di NonlinearFunc dimostra come principi matematici complessi possano essere tradotti in strumenti compositivi pratici, offrendo al compositore un controllo parametrico su processi generativi sofisticati senza richiedere una comprensione profonda della matematica sottostante.  % Auto-generated: include sezione1.tex
% --- Contenuto LaTeX autogenerato da capitolo2.md (sezione 3) ---

\section{CAPITOLO 2: CONFIGURAZIONE E PARAMETRIZZAZIONE}
Il sistema Gamma implementa una separazione netta tra logica di sintesi e configurazione dei parametri, permettendo al compositore di modificare profondamente il comportamento del sistema senza toccare il codice Csound. Questo approccio modulare facilita la sperimentazione e l'estensione del sistema.
\section{2.1 Il File tables.yaml}
Il file \texttt{tables.yaml} rappresenta il cuore configurabile del sistema, definendo tutte le tabelle di forma d'onda e inviluppo utilizzate nella sintesi. La sua struttura gerarchica separa chiaramente gli inviluppi per eventi singoli da quelli per sezioni intere.
\subsection{Struttura delle Definizioni di Tabella}
Ogni tabella è definita attraverso quattro parametri fondamentali:

\begin{lstlisting}[language=Yaml]
nome_simbolico:
  number: [numero della f-table in Csound]
  size: [dimensione in campioni]
  gen_routine: [numero della GEN routine]
  parameters: [lista dei parametri per la GEN]
\end{lstlisting}

Vediamo un esempio concreto:

\begin{lstlisting}[language=Yaml]
event_envelopes:
  lineare:
    number: 2
    size: 4096
    gen_routine: 6
    parameters: [0.001, 2048, 0.5, 2048, 1]  # Linea retta da 0 a 1
\end{lstlisting}

Questa definizione genera in Csound:
\begin{lstlisting}[language=Csound]
f 2 0 4096 6 0.001 2048 0.5 2048 1
\end{lstlisting}

La scelta della GEN routine 6 (segmenti cubici) invece della più comune GEN 7 (segmenti lineari) permette transizioni più morbide tra i punti di controllo, essenziale per inviluppi naturali.
\subsection{Inviluppi Evento vs Inviluppi Sezione}
Il sistema distingue due categorie di inviluppi con funzioni distinte:

**Event Envelopes** - Applicati ai singoli eventi sonori:

\begin{lstlisting}[language=Yaml]
event_envelopes:
  impulsivo:
    number: 5
    size: 4096
    gen_routine: 5
    parameters: [0.001, 512, 1, 3584, 0.0001]

lento:
    number: 6
    size: 4096
    gen_routine: 7
    parameters: [0, 3072, 1, 1024, 0]

sostenuto:
    number: 7
    size: 4096
    gen_routine: 7
    parameters: [0, 512, 1, 3072, 1, 512, 0]
\end{lstlisting}

L'inviluppo \texttt{impulsivo} utilizza GEN 5 (segmenti esponenziali) per creare un attacco rapidissimo (512 campioni su 4096, circa 1/8 della durata) seguito da un decadimento esponenziale. Il valore finale di 0.0001 invece di 0 evita discontinuità nell'interpolazione esponenziale.

L'inviluppo \texttt{lento} con GEN 7 crea un attacco graduale per 3/4 della durata, ideale per tessiture ambient o crescendi graduali.

**Section Envelopes** - Modulano interi gruppi di eventi:

\begin{lstlisting}[language=Yaml]
section_envelopes:
  crescendo_diminuendo:
    number: 24
    size: 4096
    gen_routine: 7
    parameters: [0, 2048, 1, 2048, 0]

impulso:
    number: 25
    size: 4096
    gen_routine: 6
    parameters: [1, 4096, 0.001]
\end{lstlisting}

Gli inviluppi di sezione operano su una scala temporale maggiore. Il numero di tabella parte da 20 per convenzione, distinguendoli chiaramente dagli inviluppi evento nel codice Csound:

\begin{lstlisting}[language=Csound]
if i_ifn_section_env > 20 && i_section_duration > 0 then
    ; Applica inviluppo di sezione
endif
\end{lstlisting}
\subsection{GEN Routines e Loro Parametri}
Le GEN routines utilizzate nel sistema sono scelte per le loro caratteristiche specifiche:

**GEN 5** - Segmenti esponenziali:
- Ideale per decadimenti naturali e attacchi percussivi
- Richiede valori non-zero per evitare singolarità matematiche
- Parametri: [valore, durata, valore, durata, ...]

**GEN 6** - Segmenti cubici:
- Transizioni morbide senza discontinuità nelle derivate
- Perfetta per inviluppi che devono suonare ''organici''
- Stessa sintassi di GEN 5 ma interpolazione diversa

**GEN 7** - Segmenti lineari:
- Controllo preciso e prevedibile
- Efficiente computazionalmente
- Ideale per forme geometriche precise

**GEN 10** - Sintesi additiva:
\begin{lstlisting}[language=Yaml]
plateau_forte:
  number: 23
  size: 4096
  gen_routine: 10
  parameters: [1]
\end{lstlisting}
Usata qui in modo non convenzionale per creare un valore costante.
\section{2.2 Sistema di Macro e Costanti Globali}
Il template CSD di Gamma definisce un sistema di macro che parametrizza l'intero spazio sonoro:

\begin{lstlisting}[language=Csound]
#define SQRT2 #1.4142135623730951#
#define MAX_AMP #0.999#
#define FONDAMENTALE #32#
#define OTTAVE #10#
#define INTERVALLI #200#
#define REGISTRI #50#
#define M_PI #3.141592653589793#
\end{lstlisting}
\subsection{Parametri dello Spazio Frequenziale}
Le macro \texttt{OTTAVE}, \texttt{INTERVALLI} e \texttt{REGISTRI} definiscono la risoluzione del sistema di intonazione:

- **OTTAVE** (10): Copre l'intero range udibile da 32 Hz a ~32 kHz
- **INTERVALLI** (200): Numero di divisioni per ottava nel sistema pitagorico
- **REGISTRI** (50): Suddivisioni fini all'interno di ogni ottava

La relazione tra questi parametri determina la granularità frequenziale:
\begin{lstlisting}
Totale frequenze = OTTAVE * INTERVALLI = 2000
Risoluzione per registro = INTERVALLI / REGISTRI = 4 intervalli
\end{lstlisting}
\subsection{Tabelle Globali e Allocazione Dinamica}
Il sistema utilizza due tabelle globali principali:

\begin{lstlisting}[language=Csound]
gi_Index init 1
gi_eve_attacco ftgen 0, 0, 2^20, -2, 0
gi_Intonazione ftgen 0, 0, $OTTAVE*$INTERVALLI+1, -2, 0
\end{lstlisting}

**gi_eve_attacco**: Una tabella enorme (2^20 = 1.048.576 elementi) che memorizza i tempi di attacco di tutti gli eventi generati. La dimensione generosa permette composizioni estremamente lunghe senza rischio di overflow.

**gi_Intonazione**: Contiene tutte le frequenze del sistema pitagorico. La dimensione è calcolata dinamicamente come \texttt{OTTAVE * INTERVALLI + 1}, dove il +1 gestisce casi limite di indicizzazione.

L'allocazione avviene con \texttt{ftgen}:
- Primo parametro (0): Numero di tabella assegnato automaticamente
- Secondo parametro (0): Creazione a init-time
- Terzo parametro: Dimensione
- Quarto parametro (-2): GEN routine per dati arbitrari
\subsection{Gestione della Memoria}
Il sistema implementa strategie per ottimizzare l'uso della memoria:

\begin{enumerate}
    \item \textbf{Tabelle Temporanee}: Lo strumento Voce crea tabelle estese che vengono deallocate alla fine:
\end{enumerate}
\begin{lstlisting}[language=Csound]
i_TempRitmiTab ftgen 0, 0, i_LenRitmiTab + 10000, -2, 0
; ... uso della tabella ...
ftfree i_TempRitmiTab, 0
\end{lstlisting}

\begin{enumerate}
    \item \textbf{Condivisione delle Tabelle Ritmiche}: Pattern identici condividono le stesse tabelle attraverso il sistema di mappatura in Python.
\end{enumerate}

\begin{enumerate}
    \item \textbf{Inizializzazione Lazy}: Le tabelle vengono popolate solo quando necessario, come gi_Intonazione che viene riempita dallo strumento Init.
\end{enumerate}
\subsection{Integrazione con Python}
Il file \texttt{tables.yaml} viene letto dal generatore Python che:
\begin{enumerate}
    \item Carica le configurazioni all'inizializzazione
    \item Genera automaticamente gli f-statement nel CSD
    \item Mantiene mappe nome→numero per riferimenti simbolici
\end{enumerate}

Questo permette di riferirsi agli inviluppi per nome nel YAML compositivo:
\begin{lstlisting}[language=Yaml]
inviluppo_attacco: { value: 'impulsivo' }
\end{lstlisting}

Invece di numeri magici:
\begin{lstlisting}[language=Yaml]
inviluppo_attacco: { value: 5 }  # Meno leggibile e manutenibile
\end{lstlisting}

Il sistema di configurazione di Gamma dimostra come una buona architettura software possa rendere un sistema complesso sia potente che accessibile, permettendo estensioni e modifiche senza richiedere modifiche al core del sistema.  % Auto-generated: include sezione2.tex
% --- Contenuto LaTeX autogenerato da capitolo3.md (sezione 4) ---

\section{INTRODUZIONE E ARCHITETTURA GENERALE}
Il \texttt{generative\_composerYaml2.py} è un sistema di composizione algoritmica che traduce una descrizione astratta di una struttura musicale, definita in formato YAML, in un file audio (WAV). Lo fa generando uno score per il software di sintesi sonora Csound.

L'architettura del programma è basata su tre componenti principali e un'esecuzione a fasi:

\begin{enumerate}
    \item \textbf{\texttt{GenerativeComposer}}: La classe principale che contiene la logica per interpretare la partitura YAML, generare i parametri stocastici degli eventi sonori e creare i file di score \texttt{.csd} per Csound.
    \item \textbf{\texttt{CompositionDebugger}}: Una classe di utilità dedicata esclusivamente alla creazione di una visualizzazione grafica (in formato PDF) della composizione generata, simile a un \textit{piano roll} arricchito con informazioni sulle tendenze parametriche.
    \item \textbf{\texttt{TimeScheduler}}: Una classe specializzata nella generazione di sequenze temporali (gli *onset*, o istanti di inizio) degli eventi, secondo diversi modelli (lineare, accelerando, ritardando, etc.).
\end{enumerate}
Il processo, orchestrato nel blocco \texttt{if \_\_name\_\_ == "\_\_main\_\_":}, non è monolitico ma suddiviso in fasi distinte e sequenziali, che permettono di separare la generazione, il rendering e la visualizzazione.
\subsection{Fase di Input: Caricamento della Struttura della Composizione}
Il punto di partenza è un file YAML. Il programma supporta la definizione di composizioni complesse, articolate in più parti, utilizzando la sintassi multi-documento di YAML (documenti separati da \texttt{{-}{-}{-}}).

La funzione \texttt{load\_all\_compositions\_from\_yaml} si occupa di questo compito:

\begin{lstlisting}[language=Python]
def load_all_compositions_from_yaml(file_path):
    """
    Carica una o più composizioni da un singolo file YAML.
    I documenti multipli devono essere separati da '---'.
    Restituisce una lista di strutture di composizione.
    """
    print(f"Caricamento partiture dal file multi-documento: {file_path}")
    composizioni = []
    try:
        with open(file_path, 'r') as f:
            docs = list(yaml.safe_load_all(f))
        for i, composition in enumerate(docs):
            if composition is None: continue 
            composizioni.append(composition)
        return composizioni
\end{lstlisting}

Ogni documento YAML caricato rappresenta una \textit{Parte} della composizione. Ogni parte è una lista di \textit{Sezioni}, e ogni sezione può contenere uno o più \textit{Layer}. Questa struttura gerarchica (Parte -> Sezione -> Layer -> Evento) è il modello concettuale su cui si basa tutta la logica successiva.

In aggiunta, viene caricato un file \texttt{tables.yaml} che definisce le caratteristiche degli inviluppi (es. attacco, rilascio) che verranno usati da Csound.
\subsection{Il Nucleo Generativo: Dal Concetto ai Parametri}
Il cuore del sistema risiede nella classe \texttt{GenerativeComposer} e nella sua capacità di trasformare le \textit{maschere} parametriche definite nel YAML in valori numerici concreti per ogni evento sonoro.

La generazione avviene all'interno di un \textit{layer}. Un layer può essere:
\begin{itemize}
 \item \textbf{Statico}: Definito da uno \texttt{stato\_unico}. Tutti gli eventi generati in questo layer attingeranno da un'unica maschera di parametri.
 \item \textbf{Dinamico}: Definito da uno \texttt{stato\_iniziale} e uno \texttt{stato\_finale}. I parametri degli eventi evolvono nel tempo, interpolando tra queste due maschere.
\end{itemize}

La funzione \texttt{\_process\_layer} gestisce un singolo layer. I suoi passaggi chiave sono:
\begin{enumerate}
    \item \textbf{Calcolo del Timing}: Determina la durata effettiva del layer basandosi sul \texttt{lifespan} (una finestra temporale relativa alla sezione, es. \texttt{[0.0, 0.5]} per la prima metà).
    \item \textbf{Generazione degli Onset}: Utilizza \texttt{TimeScheduler} per calcolare gli istanti di attivazione dei cluster di eventi all'interno della durata del layer.
    \item \textbf{Generazione degli Eventi}: Per ogni onset, determina la maschera parametrica (statica o interpolata) e genera un \textit{cluster} di eventi sonori.
\end{enumerate}
\subsubsection{Generazione Stocastica dei Parametri (\texttt{\_generate\_params\_from\_mask})}
Questa funzione è il motore stocastico. Prende una \textit{maschera} (un dizionario che descrive un parametro) e produce un valore numerico. Supporta diversi tipi di generazione:

\begin{itemize}
 \item \textbf{Distribuzione Uniforme}: Se la maschera contiene una chiave \texttt{range}.
\end{itemize}

\begin{lstlisting}[language=Python]
elif 'range' in p_mask:
    min_val, max_val = p_mask['range']
\section{...}
    val = random.uniform(min_val, max_val)
\end{lstlisting}
\begin{itemize}
 \item \textbf{Distribuzione Normale}: Se la maschera contiene \texttt{mean} e \texttt{std}.
\end{itemize}

\begin{lstlisting}[language=Python]
if 'mean' in p_mask and 'std' in p_mask:
    mean = p_mask['mean']
    std = p_mask['std']
    val = np.random.normal(loc=mean, scale=std)
\end{lstlisting}
\begin{itemize}
 \item \textbf{Scelta Pesata}: Se la maschera contiene \texttt{choices} ed opzionalmente \texttt{weights}.
\end{itemize}

\begin{lstlisting}[language=Python]
elif 'choices' in p_mask:
    val = random.choices(p_mask['choices'], weights=p_mask.get('weights'), k=1)[0]
\end{lstlisting}

Questa logica viene applicata a tutti i parametri (altezza, durata, etc.), rendendo il sistema flessibile.
\subsubsection{Interpolazione dei Parametri (\texttt{\_interpolate\_mask})}
Per i layer dinamici, questa funzione calcola una maschera intermedia tra \texttt{start\_mask} e \texttt{end\_mask} in base a un valore di \texttt{progress} (da 0 a 1). L'interpolazione è intelligente e si adatta al tipo di parametro:
\begin{itemize}
 \item I parametri numerici (come \texttt{mean}, \texttt{std}, \texttt{range}) vengono interpolati linearmente.
 \item I parametri basati su scelte (\texttt{choices}) subiscono un *cross-fade* dei loro pesi (\texttt{weights}), creando una transizione probabilistica graduale da un set di scelte a un altro.
\end{itemize}

\begin{lstlisting}[language=Python]
\section{Esempio di interpolazione di un range}
if 'range' in s_mask:
    s_min, s_max = s_mask['range']
    e_min, e_max = e_mask.get('range', s_mask['range'])
    i_min = s_min + (e_min - s_min) * shaped_progress
    i_max = s_max + (e_max - s_max) * shaped_progress
    interp_mask[key]['range'] = [i_min, i_max]
\end{lstlisting}
\subsubsection{Generazione dello Score Csound (\texttt{generate\_csd})}
Una volta generata la sequenza completa di eventi, la funzione \texttt{generate\_csd} assembla il file \texttt{.csd}. Non scrive codice Csound complesso, ma piuttosto popola un template.

\begin{enumerate}
    \item \textbf{Tabelle Dinamiche (\texttt{f{-}statements})}: Crea le tabelle per i pattern ritmici e gli inviluppi.
    \item \textbf{Eventi (\texttt{i{-}statements})}: Itera su ogni evento generato e scrive una riga di score (\texttt{i "Voce" ...}) con tutti i parametri calcolati.
\end{enumerate}
\begin{lstlisting}[language=Python]
\section{Frammento della riga di score generata}
score_lines += (f'i "Voce"\t{event_time:.4f}\t{p["durata_totale"]:.3f}\t'
                f'{p["ritmi_tab_num"]}\t{p["durata_armonica"]:.3f}\t\t{p["dynamic_index"]:.6f}\t'
\section{... altri parametri ...}
               )
\end{lstlisting}
Questo file \texttt{.csd} è un output intermedio, pronto per essere processato da Csound per generare un file audio.
\subsection{L'Orchestrazione del Rendering (Blocco \texttt{\_\_main\_\_})}
L'approccio del compositore al rendering è granulare e mira a ottimizzare i tempi di lavoro, specialmente su composizioni complesse. Questo avviene attraverso una sequenza di fasi ben definita.
\subsubsection{Fase 1: \texttt{plan\_render\_jobs}}
Questa funzione analizza l'intera struttura della partitura e crea un \textit{piano di lavoro}. Non esegue alcun rendering, ma definisce *cosa-deve essere renderizzato. Per ogni layer che necessita di rendering, crea un \textit{job}, ovvero un dizionario contenente:
\begin{itemize}
 \item La definizione del layer e della sezione a cui appartiene.
 \item I percorsi per i file \texttt{.csd} e \texttt{.wav} di output per quel singolo layer.
 \item Il tempo di inizio assoluto della sezione, calcolato tenendo conto del parametro \texttt{offset\_inizio}.
\end{itemize}
\subsubsection{Fase 2: \texttt{execute\_layer\_rendering\_and\_collect\_data}}
Questa fase esegue i \textit{job} di rendering dei layer.
\begin{enumerate}
    \item Per ogni job, invoca la logica di \texttt{GenerativeComposer} per generare gli eventi solo per quel layer.
    \item Genera un file \texttt{.csd} specifico per il layer.
    \item Lancia un processo Csound (\texttt{subprocess.Popen}) per renderizzare il \texttt{.csd} del layer in un file \texttt{.wav}.
    \item \textbf{Crucialmente}, raccoglie tutti i dati degli eventi generati in una struttura dati (\texttt{plot\_data}). Questi dati sono essenziali per la visualizzazione.
\end{enumerate}
Questo approccio permette di renderizzare solo i layer modificati se si utilizza la \texttt{veteranMode}, una modalità che salta il rendering dei layer non contrassegnati come \texttt{veteranMode: True}.
\subsubsection{Fase 3: \texttt{generate\_composition\_plot} e la Cache di Visualizzazione}
Questa fase si occupa della visualizzazione. La sua caratteristica principale è l'uso di un file cache, \texttt{visual\_cache.json}:
\begin{enumerate}
    \item \textbf{Lettura della Cache}: Carica i dati di visualizzazione da esecuzioni precedenti, se disponibili.
    \item \textbf{Merge}: Se sono stati generati nuovi dati (da \texttt{execute\_layer\_rendering...}), questi vengono uniti alla cache, sostituendo i dati vecchi per i layer che sono stati ri-renderizzati.
    \item \textbf{Plotting}: Usa la classe \texttt{CompositionDebugger} per creare un PDF multi-pagina che mostra: -  Gli eventi sonori come rettangoli. -  Le \textit{maschere di tendenza} (le buste grigie/arancioni) che mostrano l'evoluzione dei range parametrici. -  L'evoluzione delle dinamiche. -  Marcatori per sezioni e layer.
    \item \textbf{Scrittura della Cache}: Salva lo stato aggiornato dei dati di visualizzazione nel file JSON. Questo garantisce che, alla prossima esecuzione in \texttt{veteranMode}, il grafico mostri comunque l'intera composizione (parti vecchie e nuove). La funzione \texttt{\_sanitize\_data\_for\_json} è un helper fondamentale qui, poiché converte tipi di dati specifici di NumPy e \texttt{pathlib} in formati compatibili con JSON.
\end{enumerate}
\subsubsection{Fase 4 e 5: Assemblaggio}
Il rendering finale non avviene generando un unico, enorme file CSD. Avviene invece tramite un processo di assemblaggio gerarchico:

\begin{enumerate}
    \item \texttt{execute\_section\_assembly}: Per ogni sezione, genera un CSD \textit{assembler}. Questo CSD non produce suono, ma si limita a leggere e mixare i file \texttt{.wav} dei singoli layer (generati nella Fase 2) per creare un unico file \texttt{.wav} per l'intera sezione.
    \item \texttt{execute\_final\_assembly}: Genera un ultimo CSD \textit{assembler} che prende i file \texttt{.wav} di tutte le sezioni e li posiziona in sequenza (rispettando gli \texttt{offset\_inizio}) per creare il file \texttt{.wav} finale e completo della composizione.
\end{enumerate}
Questo approccio a \textit{render per layer, poi assembla} ha il vantaggio di essere più gestibile e di non richiedere la rigenerazione dell'intera composizione per piccole modifiche.

Il \texttt{generative\_composerYaml2.py} implementa un flusso di lavoro completo e disaccoppiato per la composizione algoritmica. Le sue caratteristiche tecniche salienti sono:

\begin{itemize}
 \item \textbf{Configurazione Esterna (YAML)}: Offre un'interfaccia di alto livello per la descrizione musicale, separando la logica del codice dai dati della composizione.
 \item \textbf{Generazione Stocastica Multi-modello}: Fornisce un set flessibile di strumenti per definire il comportamento dei parametri sonori.
 \item \textbf{Rendering Granulare e a Fasi}: Scompone il problema del rendering in sotto-problemi più piccoli (layer, sezioni), ottimizzando il processo di lavoro iterativo.
 \item \textbf{Caching della Visualizzazione}: Garantisce che il feedback visivo sia sempre coerente e completo, anche quando si lavora solo su parti della composizione.
 \item \textbf{Assemblaggio Gerarchico}: Utilizza Csound non solo per la sintesi ma anche come uno strumento di montaggio audio per assemblare i componenti finali.
\end{itemize}  % Auto-generated: include sezione3.tex
% --- Contenuto LaTeX autogenerato da capitolo4.md (sezione 5) ---

\section{MASCHERE DI TENDENZA E GENERAZIONE PARAMETRICA}
La generazione parametrica è il motore alchemico di Gamma, il processo centrale attraverso cui le intenzioni del compositore, espresse come  maschere di tendenza  nel file YAML, vengono trasformate in valori numerici concreti per ogni singolo evento sonoro. Analizzeremo come il sistema traduce l'astrazione in suono, con particolare attenzione alle tecniche di generazione, interpolazione e gestione della coerenza strutturale.
\subsection{Il Motore Generativo: \texttt{\_generate\_params\_from\_mask()}}
Il metodo \texttt{\_generate\_params\_from\_mask()} è il punto di convergenza tra l'astrazione compositiva e la concretezza numerica. Implementa la logica che trasforma una singola maschera di tendenza nei parametri specifici per un evento sonoro, gestendo una varietà di strategie generative per rispondere a diverse necessità espressive.
\subsubsection{Architettura e Gestione Specializzata}
Il processo non è monolitico. Il metodo prima isola un insieme di chiavi che richiedono una gestione specializzata, poiché non rappresentano parametri diretti o necessitano di logiche di trasformazione complesse.

\begin{lstlisting}[language=Python]
SKIPPED_KEYS = {
    'choices', 'weights', 'distribution',  # Metadati per la generazione
    'dynamic_index', 'dinamica',           # Logica di dinamica complessa
    'nonlinear_mode', 'senso_movimento',   # Parametri di controllo per Csound
    'inviluppo_attacco', 'tipo_ritmi',      # Richiedono traduzione o generazione complessa
    'densita_cluster'                      
}
\end{lstlisting}
Questa separazione permette a un loop generico di gestire i parametri puramente numerici, mentre logiche dedicate si occupano di tradurre concetti come \texttt{'dinamica': 'f'} nell'indice numerico richiesto da Csound o di generare intere sequenze ritmiche da una categoria come \texttt{'medi'}.
\subsubsection{Le Modalità di Generazione}
Il cuore del metodo itera sui parametri, applicando la modalità di generazione più appropriata in base alla struttura della maschera.

\begin{lstlisting}[language=Python]
for key, p_mask in mask.items():
    if key in SKIPPED_KEYS: continue
\section{1. Distribuzione Normale (Gaussiana)}
    if 'mean' in p_mask and 'std' in p_mask:
        val = np.random.normal(loc=p_mask['mean'], scale=p_mask['std'])
\section{2. Distribuzione Uniforme (Range)}
    elif 'range' in p_mask:
        min_val, max_val = p_mask['range']
        val = random.randint(min_val, max_val) if isinstance(min_val, int) else random.uniform(min_val, max_val)
\section{3. Scelta Discreta Pesata}
    elif 'choices' in p_mask:
        val = random.choices(p_mask['choices'], weights=p_mask.get('weights'), k=1)[0]
\section{4. Valore Fisso}
    elif 'value' in p_mask:
        val = p_mask['value']
\end{lstlisting}

Questo toolkit di quattro modalità offre al compositore un controllo granulare sul grado di determinismo e casualità:

\begin{enumerate}
    \item Distribuzione Normale : Ideale per creare una \textit{massa} sonora attorno a un centro tonale o timbrico. Un'ottava definita come \texttt{\{mean: 5, std: 0.5\}} tenderà a rimanere nell'ottava 5, ma con occasionali e naturali \textit{fughe} verso le ottave vicine. La deviazione standard diventa un parametro espressivo che controlla la \textit{disciplina} del materiale.
    \item Range Uniforme \texttt{range: [1, 10]}: Utile quando tutti i valori in un intervallo sono ugualmente possibili.
    \item Scelta Pesata : Permette di definire il \textit{colore} statistico di una sezione. Una dinamica specificata come \texttt{\{choices: ['p', 'mf', 'f'], weights: [0.6, 0.3, 0.1]\}} assicura una predominanza di eventi piano, pur mantenendo la varietà.
    \item Valore Fisso : Garantisce il determinismo assoluto, essenziale per parametri strutturali come il senso di movimento spaziale.
\end{enumerate}
\subsubsection{La Logica Gerarchica dei Ritmi}
La generazione dei ritmi è un esempio emblematico di come il sistema supporti molteplici livelli di astrazione, dal controllo totale alla delega generativa.

\begin{lstlisting}[language=Python]
if 'explicit_values' in rhythm_mask:
\section{Modalità 1: Controllo totale con una lista esplicita}
    params['ritmi'] = rhythm_mask['explicit_values']
elif 'choices' in rhythm_mask:
    choice = random.choices(rhythm_mask['choices'], ...)[0]
    if isinstance(choice, list):
\section{Modalità 2: Scelta tra pattern pre-composti}
        params['ritmi'] = choice
    else:
\section{Modalità 3: Astrazione massima tramite categorie ('piccoli', 'medi'...)}
        params['ritmi'] = self._generate_rhythm_pattern(choice)
\end{lstlisting}
Questa architettura permette al compositore di scegliere il livello di dettaglio più consono: specificare un pattern esatto, scegliere da una libreria di pattern, o semplicemente indicare una \textit{qualità} ritmica desiderata.
\subsection{L'Evoluzione nel Tempo: Interpolazione delle Maschere}
Se la generazione da una singola maschera crea eventi statici, l'interpolazione tra due maschere (\texttt{stato\_iniziale} e \texttt{stato\_finale}) dà vita a processi dinamici e trasformativi. Il metodo \texttt{\_interpolate\_mask()} implementa questa logica.
Un problema chiave nell'interpolazione è come gestire parametri che compaiono solo nello stato finale. Gamma adotta una strategia di \textit{riempimento} che ne aumenta la flessibilità.

\begin{lstlisting}[language=Python]
all_keys = set(start_mask.keys()) | set(end_mask.keys())
for key in all_keys:
    s_mask = start_mask.get(key)
    e_mask = end_mask.get(key)

if s_mask is None: s_mask = e_mask
    if e_mask is None: e_mask = s_mask
\end{lstlisting}
Se un parametro è definito solo alla fine, il sistema assume che fosse presente fin dall'inizio con lo stesso valore finale. Questo permette di \textit{introdurre} un nuovo processo (es. un glissando) senza doverne specificare un valore nullo all'inizio, semplificando la scrittura delle partiture YAML.
\subsubsection{Strategie di Interpolazione Differenziate}
L'interpolazione si adatta al tipo di parametro, producendo transizioni musicalmente significative:

\begin{itemize}
 \item Parametri Numerici (Range, Mean/Std) : I loro valori vengono interpolati linearmente. Questo permette di creare effetti come un \textit{restringimento} del campo sonoro (interpolando verso un range più piccolo) o una \textit{focalizzazione} (interpolando verso una deviazione standard minore).
\end{itemize}

\begin{itemize}
 \item Scelte Discrete (Choices) : Il sistema tenta un  cross-fade probabilistico . Se le scelte sono le stesse, i loro pesi (\texttt{weights}) vengono interpolati. Questo rea una transizione graduale nella probabilità di occorrenza, ad esempio passando da una predominanza di dinamiche piano a una di forte. Se le scelte sono diverse, il sistema effettua una transizione a scalino a metà del percorso.
\end{itemize}

È importante notare che questo processo di interpolazione non solo guida la generazione degli eventi sonori, ma fornisce anche i dati per la visualizzazione grafica. Le \textit{maschere di tendenza} visibili nel PDF generato da \texttt{CompositionDebugger} sono la rappresentazione visiva diretta dei valori interpolati in ogni punto del tempo. Questo crea una coerenza totale tra ciò che il compositore specifica, ciò che il sistema visualizza e ciò che l'orchestra Csound suona.
\subsection{Il Sistema Gerarchico del Glissando}
Il glissando in Gamma non è una semplice transizione di frequenza, ma un sistema gerarchico che illustra l'approccio progettuale del compositore: fornire opzioni potenti con priorità chiare.

\begin{enumerate}
    \item Modalità Offset (Priorità Massima): Specifica un intervallo di glissando relativo alla nota di partenza (es. \texttt{offset\_ottava: 2}). È ideale per creare pattern di movimento che mantengono la loro coerenza intervallare a diverse altezze.
    \item Modalità Assoluta (Priorità Media) : Specifica una destinazione fissa (es. \texttt{ottava\_arrivo: 8}). È utile per creare convergenze armoniche, dove più voci, partendo da punti diversi, si dirigono verso un'unica regione tonale.
    \item Default (Nessun Glissando) : In assenza di specifiche, la frequenza rimane statica.
\end{enumerate}
Questa gerarchia viene risolta in \texttt{\_generate\_params\_from\_mask()}, che calcola i parametri \texttt{ottava\_arrivo} e \texttt{registro\_arrivo} finali. Questi vengono poi passati a Csound, dove la funzione \texttt{calcFrequenza} è chiamata due volte, una per la frequenza di partenza e una per quella di arrivo, garantendo che entrambe rispettino la logica dell'intonazione pitagorica del sistema.  % Auto-generated: include sezione4.tex
% --- Contenuto LaTeX autogenerato da capitolo5.md (sezione 6) ---

\section{SCOLPIRE IL TEMPO: MODELLI TEMPORALI E MICRO-RITMICA}
La classe \texttt{TimeScheduler} incapsula la logica per la distribuzione temporale, offrendo un toolkit di modelli che corrispondono a gesti musicali archetipici. La scelta di isolare questa funzionalità in una classe dedicata sottolinea come il tempo musicale non sia un semplice parametro, ma una dimensione fondamentale che richiede un trattamento specializzato.

Il metodo \texttt{generate\_onsets} è il cuore di questa classe. La sua architettura è elegante e flessibile: parte sempre da una progressione lineare di base, che viene poi \textit{deformata} o \textit{rimappata} secondo il modello scelto nel file YAML.

\begin{lstlisting}[language=Python]
def generate_onsets(self, model, duration, num_events):
    base_progress = np.linspace(0, 1, num_events, endpoint=False)
    final_progress = np.zeros_like(base_progress)
    model_type = model.get('type', 'linear')
    return final_progress * duration
\end{lstlisting}

Questa architettura a due fasi (generazione di una progressione normalizzata e successiva trasformazione) permette di definire gesti temporali indipendentemente dalla durata effettiva, rendendoli riutilizzabili e scalabili.
\subsubsection{I Modelli Archetipici}
\textbf{Lineare (\texttt{type: linear})}

Il modello di default, che distribuisce gli eventi in modo equidistante. Sebbene semplice, è fondamentale per creare pulsazioni regolari, ostinati o griglie ritmiche stabili su cui altri layer possono costruire complessità.

\textbf{Ritardando (\texttt{type: ritardando})} 
\begin{lstlisting}[language=Python]
elif model_type == 'ritardando':
    shape = model.get('shape', 2.0)
    final_progress = base_progress ** shape
\end{lstlisting}
Applicando una funzione di potenza con esponente maggiore di 1, la curva di progressione si \textit{piega}, concentrando gli eventi all'inizio e diradandoli verso la fine. Il parametro \texttt{shape} controlla l'intensità del gesto: un valore più alto crea un ritardando pronunciato.

\textbf{Accelerando (\texttt{type: accelerando})}
\begin{lstlisting}[language=Python]
elif model_type == 'accelerando':
    shape = model.get('shape', 2.0)
    final_progress = 1 - (1 - base_progress) ** shape
\end{lstlisting}
La formula inverte la progressione, applica la potenza e la inverte nuovamente.
\textbf{Stocastico (\texttt{type: stochastic})}

\begin{lstlisting}[language=Python]
elif model_type == 'stochastic':
    final_progress = np.sort(np.random.rand(num_events))
\end{lstlisting}
Il metodo genera un set di istanti casuali e poi li ordina. Questo garantisce che, pur essendo irregolari, gli eventi mantengano una progressione temporale in avanti. È un modello efficace per rompere la rigidità della griglia metrica.
\subsubsection{Il Modello \texttt{breakpoint}: Curve Temporali su Misura}
Il modello più potente e flessibile è \texttt{breakpoint}. Permette al compositore di \textit{disegnare} una curva di distribuzione temporale definendo una serie di punti di controllo. Ogni segmento tra due punti può avere la propria curvatura, consentendo la creazione di profili complessi.

Consideriamo un esempio:
\begin{lstlisting}[language=Python]
timing_model:
  type: breakpoint
  points:
    - [0.0, 0.0]        # Inizio
    - [0.3, 0.7, 0.5]   # Il 70% degli eventi avviene nel primo 30% del tempo (curva concava, ease-out)
    - [1.0, 1.0, 3.0]   # Il restante 30% degli eventi si distribuisce nel 70% del tempo rimanente (curva convessa, ease-in)
\end{lstlisting}
Questo YAML descrive un gesto di \textit{esplosione e diradamento}: un'alta densità di eventi all'inizio, seguita da una lunga coda rarefatta.

L'implementazione gestisce questa complessità in modo modulare:
\begin{lstlisting}[language=Python]
time_in_segment = (segment_times - t_start) / (t_end - t_start)
shaped_time = time_in_segment ** shape
interpolated_values = v_start + (v_end - v_start) * shaped_time
final_progress[segment_mask] = interpolated_values
\end{lstlisting}
La logica chiave è la normalizzazione del tempo *all'interno di ogni segmento*. Questo permette di applicare una curva di \texttt{shape} locale senza che questa influenzi gli altri segmenti, rendendo il sistema potente e intuitivo.  % Auto-generated: include sezione5.tex
% --- Contenuto LaTeX autogenerato da capitolo6.md (sezione 7) ---

\section{SINTASSI E SEMANTICA COMPOSITIVA}
YAML (YAML Ain't Markup Language) emerge in Gamma non solo come formato di configurazione, ma come vero e proprio linguaggio di partitura per la composizione algoritmica. La scelta di YAML rispetto ad altri formati riflette la necessità di bilanciare leggibilità umana con precisione computazionale, creando un ponte tra l'intuizione compositiva e l'esecuzione algoritmica.
\subsection{Struttura Gerarchica}
La struttura compositiva in Gamma segue una gerarchia rigorosa che rispecchia l'organizzazione tradizionale della musica occidentale, adattandola alle esigenze della generazione algoritmica.
\subsubsection{La Gerarchia Fondamentale}
Al livello più alto, una composizione è una lista di sezioni:

\begin{lstlisting}[language=Python]
  nome_sezione: "Introduzione"
  durata: 30
  layers:
\begin{itemize}
    \item nome_layer: "Texture di base"
\end{itemize}
\section{parametri del layer}
\begin{itemize}
    \item nome_layer: "Eventi puntuali"
\end{itemize}
\section{parametri del layer}
  nome_sezione: "Sviluppo"
  durata: 60
  layers:
\section{altri layers}
\end{lstlisting}

Questa struttura apparentemente semplice nasconde una ricchezza semantica considerevole. Ogni livello gerarchico porta con sé un dominio di parametri specifico e regole di ereditarietà implicite.
\subsubsection{Parametri per Livello Gerarchico}
\textbf{Livello Sezione}: I parametri a questo livello influenzano tutti i layer contenuti:
\begin{itemize}
    \item \texttt{durata}: Definisce il contenitore temporale
    \item \texttt{ratio\_temporale}: Permette dilatazioni o compressioni senza modificare i valori numerici
    \item \texttt{inviluppo\_sezione}: Applica una modulazione globale d'ampiezza
\end{itemize}

\textbf{Livello Layer}: Qui si definisce l'identità del flusso sonoro:
\begin{itemize}
    \item \texttt{num\_attivazioni}: Controlla la densità eventi
    \item \texttt{timing\_model}: Determina la distribuzione temporale
    \item \texttt{lifespan}: Definisce quando il layer è attivo
    \item Stati (unico/iniziale/finale): Contengono le maschere di tendenza
\end{itemize}

La separazione dei domini parametrici non è arbitraria. Riflette una comprensione che certi aspetti musicali (come la durata totale di una sezione) sono strutturali, mentre altri (come la distribuzione delle altezze) sono texturali.
\subsubsection{Eredità e Override dei Valori}
Il sistema implementa un modello di eredità implicita dove i valori di default si propagano attraverso la gerarchia:

\begin{lstlisting}[language=Python]
\begin{itemize}
    \item nome_sezione: "Sezione con defaults"
\end{itemize}
  durata: 60
\section{ratio_temporale assume valore 1.0}
\section{inviluppo_sezione assume 'continua'}
  layers:
\begin{itemize}
    \item nome_layer: "Layer minimale"
\end{itemize}
\section{num_attivazioni assume 10}
\section{timing_model assume}
      stato_unico:
        ottava: {range: [4, 6]}
\section{tutti gli altri parametri assumono defaults}
\end{lstlisting}

Questa eredità permette specifiche concise quando i defaults sono appropriati, ma mantiene la possibilità di override granulare quando necessario. Il meccanismo di normalizzazione in Python garantisce che anche parametri specificati in forma abbreviata vengano espansi nella forma completa prima del processing.
\subsection{Definizione delle Maschere}
Le maschere di tendenza rappresentano il cuore semantico del sistema, trasformando il YAML da semplice formato di dati a linguaggio espressivo per la composizione.
\subsubsection{Sintassi delle Modalità di Generazione}
La sintassi per le maschere supporta quattro modalità principali, ciascuna con la propria semantica:

\textbf{Range}: Definisce un intervallo di valori equiprobabili:
\begin{lstlisting}[language=Python]
ottava: {range: [3, 7]}
registro: {range: [1.0, 10.0]}  # float permette microtonalità
\end{lstlisting}

\textbf{Choices}: Permette selezione da un insieme discreto:
\begin{lstlisting}[language=Python]
dinamica: {choices: ['p', 'mf', 'f']}
\section{Con pesi per distribuzione non uniforme}
tipo_ritmi: {choices: ['piccoli', 'medi'], weights: [0.3, 0.7]}
\end{lstlisting}

\textbf{Distribuzione Normale}: Per concentrazione attorno a un centro:
\begin{lstlisting}[language=Python]
durata_armonica: {mean: 2.0, std: 0.5}
\end{lstlisting}

\textbf{Valore Fisso}: Quando non si desidera variazione:
\begin{lstlisting}[language=Python]
senso_movimento: {value: -1}
\end{lstlisting}
\subsubsection{Normalizzazione Automatica}
Il sistema permette sintassi abbreviate che vengono espanse automaticamente:

\begin{lstlisting}[language=Python]
\section{Forma abbreviata}
dinamica: 'mf'
\section{Viene normalizzata in}
dinamica: {value: 'mf'}
\end{lstlisting}

Questa normalizzazione avviene nel metodo \texttt{\_normalize\_mask()} e permette di mantenere il YAML leggibile senza sacrificare la consistenza interna del sistema.
\subsubsection{Parametri Interpolabili vs Fissi}
Non tutti i parametri supportano l'interpolazione. La distinzione riflette la natura musicale dei parametri:

\textbf{Interpolabili}:
\begin{itemize}
    \item Parametri numerici continui (ottava, registro, durata)
    \item Distribuzioni (mean, std di una normale)
    \item Pesi di scelte discrete (quando le scelte sono identiche)
\end{itemize}

\textbf{Non Interpolabili}:
\begin{itemize}
    \item Stringhe che rappresentano categorie (\texttt{tipo\_ritmi} quando usa categorie)
    \item Liste di valori (\texttt{explicit\_values} per ritmi)
    \item Parametri strutturali (\texttt{timing\_model})
\end{itemize}

Questa distinzione è gestita automaticamente dal sistema di interpolazione, che applica strategie appropriate per ogni tipo.
\subsection{Controlli Avanzati}
Oltre ai parametri musicali di base, Gamma offre controlli avanzati che permettono di gestire aspetti sottili ma cruciali della generazione.
\subsubsection{Lifespan e Ciclo di Vita dei Layer}
Il parametro \texttt{lifespan} permette controllo fine su quando un layer è attivo:

\begin{lstlisting}[language=Python]
layers:
\begin{itemize}
    \item nome_layer: "Introduzione graduale"
\end{itemize}
    lifespan: [0.0, 0.3]  # Solo nel primo 30%

\begin{itemize}
    \item nome_layer: "Corpo principale"  
\end{itemize}
    lifespan: [0.2, 0.9]  # Dal 20% al 90%, sovrapposizione con intro

\begin{itemize}
    \item nome_layer: "Coda"
\end{itemize}
    lifespan: [0.8, 1.0]  # Ultimo 20%, sovrapposizione con corpo
\end{lstlisting}

Questa specifica crea una forma ad arco con sovrapposizioni controllate, impossibile da ottenere con semplice sequenzialità.
\subsubsection{Safety Buffer e Leeway}
Due meccanismi complementari gestiscono i bordi temporali:

\begin{lstlisting}[language=Python]
\begin{itemize}
    \item nome_layer: "Eventi lunghi"
\end{itemize}
  usa_safety_buffer: true  # Default, previene sconfinamenti
  leeway_fine_layer: 2.0   # Permette 2 secondi extra alla fine
\end{lstlisting}

Il safety buffer sottrae tempo dalla generazione per garantire che nessun evento ecceda i limiti. Il leeway aggiunge tempo extra alla fine per permettere code naturali. La combinazione permette controllo preciso del comportamento ai bordi mantenendo flessibilità espressiva.
\subsubsection{Modalità Solo e Veteran Mode}
Due modalità speciali facilitano il workflow compositivo:

\begin{lstlisting}[language=Python]
\begin{itemize}
    \item nome_layer: "Layer in sviluppo"
\end{itemize}
  solo: true  # Solo questo layer sarà renderizzato

\begin{itemize}
    \item nome_layer: "Layer completato"
\end{itemize}
  veteranMode: true  # Riusa il WAV esistente se presente
\end{lstlisting}

La modalità \texttt{solo} isola layer specifici per testing e rifinitura. Il \texttt{veteranMode} ottimizza i tempi di rendering riutilizzando materiale già generato. Entrambe le modalità operano a livello di orchestrazione Python senza influenzare la generazione Csound.
\subsubsection{Implicazioni Compositive della Sintassi}
La sintassi YAML di Gamma non è neutra - incorpora assunzioni e affordance che guidano il processo compositivo. La struttura gerarchica suggerisce di pensare in termini di sezioni e layer. La sintassi delle maschere incoraggia il pensiero probabilistico. I controlli avanzati permettono raffinamenti che sarebbero complessi in notazione tradizionale.

Questa non è semplicemente una questione di convenienza. Il linguaggio che usiamo per descrivere la musica influenza profondamente come pensiamo alla musica stessa. YAML in Gamma diventa così non solo un formato di dati, ma un medium compositivo che shapes il pensiero musicale verso paradigmi stocastici e stratificati, mantenendo al contempo una connessione con i concetti tradizionali di struttura e forma musicale.  % Auto-generated: include sezione6.tex
% --- Contenuto LaTeX autogenerato da conclusione.md (sezione 9) ---

\section{Conclusione: Un Ecosistema per la Composizione}
L'analisi ha dimostrato come Gamma realizzi con successo il suo paradigma centrale: quello delle maschere di tendenza. Il compositore definisce i confini, le probabilità, le traiettorie, e il sistema esplora lo spazio creativo così delineato.

Tuttavia l'analisi ha evidenziato anche le sfide intrinseche di questo approccio. La sintassi YAML, pur essendo leggibile, può diventare complessa e verbosa quando si definiscono evoluzioni parametriche sofisticate. Inoltre, il tempo di rendering di Csound rimane il principale collo di bottiglia, suggerendo che per un lavoro più agile potrebbero essere esplorate soluzioni di caching più avanzate o motori di sintesi alternativi per le anteprime.

Queste considerazioni non sminuiscono i risultati, ma anzi tracciano una rotta per il futuro. Il percorso da Gamma a un ipotetico sistema \textit{Delta} emerge quasi naturalmente dall'architettura esistente. Se Gamma è un sistema dove i layer operano in un elegante isolamento, Delta potrebbe esplorare l'interazione tra layer, trasformando la composizione da un insieme di processi paralleli a un vero e proprio ecosistema di agenti sonori che si ascoltano e si influenzano a vicenda. L'introduzione di una memoria a lungo termine e di metriche di valutazione del materiale generato potrebbe trasformare il sistema da un esecutore di istruzioni a un agente capace di auto-organizzazione.

In definitiva, Gamma si è dimostrato una piattaforma robusta e flessibile, un fondamento tecnico e concettuale solido. L'esperienza acquisita nella sua progettazione e nel suo utilizzo non è solo un traguardo, ma il trampolino di lancio per esplorare la prossima frontiera della musica generativa: non più solo creare suono secondo regole, ma progettare sistemi che scoprono le proprie regole attraverso l'esplorazione e l'interazione.  % Auto-generated: include conclusione.tex


% Aggiungi la bibliografia
\newpage % ---- Inizia una nuova pagina prima della bibliografia
\bibliographystyle{plain}
\bibliography{bibliography}

\end{document}